%%%%%%%%%%%%%%%%%%%%%%%%%%%%%%%%%%%%%%%%%%%%%%%%%%%%%%%%%%%%%%%%%%%%%%%%%%%%%%%%%%%%%%%%%%%%%%%%%%%%%%%
% Soutenance Finale
% Bitume Legends Project
% CarrEniX
% Juin 2022
%%%%%%%%%%%%%%%%%%%%%%%%%%%%%%%%%%%%%%%%%%%%%%%%%%%%%%%%%%%%%%%%%%%%%%%%%%%%%%%%%%%%%%%%%%%%%%%%%%%%%%%

<<<<<<< HEAD

=======
>>>>>>> 0737757309992d623e78eb6953edcad6021eead0
\documentclass[a4paper,12pt]{article}

\usepackage[utf8]{inputenc}
\usepackage[french]{babel}
\usepackage{geometry}
\usepackage{lipsum}
\usepackage{mathtools}
\usepackage[T1]{fontenc}
\usepackage{url}
\usepackage{longtable}
\usepackage[aboveskip=.5cm]{caption}
\usepackage{pdflscape}
\usepackage{xcolor}
\usepackage{graphicx,color}

\newcommand{\hsp}{\hspace{20pt}}
\newcommand{\HRule}{\rule{\linewidth}{0.5mm}}
\newcommand{\btmlgs}{\textsl{Bitume Legends}}
\newcommand{\AI}{Intelligence Artificielle}
\newcommand{\FL}{\textsl{FL Studio 20}}
\newcommand{\CEX}{\textsc{CarrEniX}}
\newcommand{\SITE}{\(\mathtt{btms.games}\)}
\renewcommand{\listfigurename}{Table des Annexes}
\renewcommand{\contentsname}{Table des Matières}
\newcommand\ytl[2]{\parbox[b]{10em}{\hfill{\color{cyan}\bfseries\sffamily 
    #1}~$\cdots\cdots$~}\makebox[0pt][c]{$\bullet$}\vrule\quad 
    \parbox[c]{5cm}{\vspace{7pt}\color{red!40!black!80}\raggedright\sffamily #2\\[7pt]}\\[-3pt]}


\usepackage{fancyhdr}
\pagestyle{fancy}
<<<<<<< HEAD
\lhead{Rapport de Projet}
=======
\lhead{Rapport de Soutenance Finale}
>>>>>>> 0737757309992d623e78eb6953edcad6021eead0
\rhead{{\CEX}\\{\btmlgs}}   

\begin{document}  

    \graphicspath{{../Medias}}

\begin{titlepage}
  \begin{sffamily}
  \begin{center}

  
    \includegraphics[scale=0.3]{epita.png}\\[1.5cm]

    \textsc{\huge EPITA Rennes}\\[0.5cm]

    % Titre
    \textsc{\Large Rapport de Soutenance Finale}\\[1.5cm]

    \HRule \\[0.4cm]
     { \LARGE \bfseries \btmlgs \\[0.4cm] }

    \HRule \\[2cm]
    \includegraphics[scale=0.5]{logo192.png}
     \\[0.5cm]
    \textsc{\Large \CEX}\\[1.5cm]

    % Membres
    \begin{minipage}{0.4\textwidth}
      \begin{flushleft} \large
        Anthony Caron\\
        Victorien Cambourian\\
      \end{flushleft}
    \end{minipage}
    \begin{minipage}{0.4\textwidth}
      \begin{flushright} \large
        Melvyn Delaroque\\
        Xavier de Place\\
      \end{flushright}
    \end{minipage}

    \vfill

    % Bottom of the page
    {\large 9 Juin 2022}

  \end{center}
  \end{sffamily}
\end{titlepage}


    \newpage

    \tableofcontents

    \newpage

    \addcontentsline{toc}{section}{Introduction}
    \section*{Introduction}
<<<<<<< HEAD
    Nous sommes le Studio \CEX\;et nous sommes fiers de vous présenter le rapport de notre projet, \btmlgs. 
    Dans ce document, vous découvrirez comment nous avons mené à bien notre projet, mais également comment de fil en aiguille nous avons abouti à ce résultat. Suivant un plan chronologique, nous aborderons les différents mode de jeu, notre façon de travailler, notre approche du jeu, mais également sa promotion et ce que nous avons tiré des différentes soutenances ainsi que ce que nous a apporté ce projet. Nous conclurons avec des remerciements aux différentes personnes qui ont contribué à faire avancer le projet \btmlgs.


    \section{Origine du projet}
        Dans notre groupe, nous aimons beaucoup les voitures. Nous regardons tous des émissions comme 
        \textit{Top Gear}, \textit{The Grand Tour} ou encore \textit{Vilebrequin}. Nous avons donc décider de faire
        un projet qui se rapporte à cet univers. Nous avons réfléchi à plusieurs possibilités de jeux, mais nous
        nous sommes mis d'accord sur un jeu de courses plutôt qu'un RPG où les joueurs incarneraient des voitures.
        De plus, un jeu de courses est facilement imaginable avec un mode multijoueur et un mode \AI. C'était donc un choix qui s'imposait pour nous.\\

        Le nom \textit{Bitume Legends} est une référence au jeu de Gameloft, \textit{Asphalt 9 Legends}.
        Nous connaissons tous ce jeu et nous avons décidé de nous inspirer grandement de son nom tout en y intégrant
        une de nos spécialités : les jeux de mots. Le nom de notre team \CEX\, est du même type. Nous avons 
        francisé le nom du studio de jeux vidéo Square Enix. En français, cela donne Carré Enix, réduit en
        \CEX.\\
    
    \section{Concept de base}
        Notre vision originale était de faire un jeu de course automobile en style \textit{Low Poly}.
        Nous voulions avoir un système d'expérience, permettant de débloquer de nouvelles voitures plus puissantes.
        Pour acquérir cette expérience, nous voulions que le joueur puisse faire des courses de plusieurs types.
        Le premier serait de se battre contre la montre, c'est à dire faire des courses en temps limité.
        Le second serait de se battre contre une \AI, de faire la course contre une voiture automatique.
        Et enfin, nous voulions avoir une option multijoueur, pour que nous puissions faire des soirées avec nos amis
        et jouer à plusieurs à notre jeu. Pour que le jeu ne soit pas ennuyant, nous voulions créer différents
        circuits, dans différents endroits et avec des atmosphères différentes. A cela, nous voulions ajouter
        des voitures aux caractéristiques réalistes, et aux designs variés. Enfin, nous voulions avoir notre propre
        bande son, composée par nos soins, et notre charte graphique.
    
    % \clearpage
    
    \section{Organisation interne chez \CEX}
    \subsection{Organisation Pratique}
        Nous avons mis en place une organisation particulière entre nous,
        car au début nous ne savions pas spécialement par où commencer.
        Chaque semaine, nous nous sommes réunis pour définir des 
        objectifs pour chacun, à faire durant la semaine suivante. Cela nous 
        a permis d’avoir des buts concrets sur le court terme et d'avancer plus 
        efficacement.

    \subsection{Répartition des Tâches}
        Nous nous sommes basés sur ce que nous avions annoncé dans notre Cahier des Charges :
        Melvyn s'est occupé des musiques d'ambiance, du \textit{Sound Design} de 
        notre jeu et du menu principal. Victorien a créé le premier circuit du jeu,
        il a implémenté le menu principal et a modélisé sur \textsl{Blender} une Formule 1.
        Anthony a géré le site Internet, les réseaux sociaux et le programme \(\beta\).
        Enfin, Xavier a implémenté le mode multijoueur, et le cœur du moteur de jeu,
        à savoir le système de direction des voitures.

    \subsubsection{Melvyn}
        Étant responsable de la musique, j'ai composé la 
        musique du jeu. J'ai utilisé le logiciel de MAO (Musique Assistée par Ordinateur) 
        \textsl{FL Studio 20}. J'ai composé deux musiques, une musique pour le menu et 
        une musique de course et j'ai demandé à un ami d'enregistrer sa voix pour la musique
        dans un studio d'enregistrement de son école. Le style choisi par mes pairs et moi-même
        au sein du groupe est la \textsl{Phonk}, style est souvent associé à la culture automobile
        \textit{underground} japonaise, d'où le désir d'utiliser de la \textsl{Phonk} pour notre jeu. 
        De plus, ce style est très énergique, assourdissant parfois et fait monter l'adrénaline
        dans le sang. Au début, j'ai hésité à faire de l'\textit{Eurobeat}, un genre de \textit{dance music}
        associé au milieu automobile, mais le groupe a tranché et nous avons choisi un genre qui
        nous est à la fois plus familier et bien plus facile à produire comparé à de la musique
        electronique. De plus je suis responsable du design et j'ai travaillé sur le design des
        menus, ainsi que sur l'identité graphique du jeu et l'identité sonore de \btmlgs.

    \subsubsection{Victorien}
        Étant donné que nous créons un jeu de voiture, il était 
        bienvenu de modéliser au moins une voiture par nos propres soins. 
        Ayant des connaissances dans \textsl{Blender}, un logiciel 
        de modélisation 3D, je m'en suis chargé. 
        Nous avons convenu d'une Formule 1, voiture considéré comme le "Graal" du jeux. 
        Les formes de la voiture ont toutes été créées à partir de formes basiques,
        telles que des rectangles, des cylindres, des sphères ou encore des cubes. 
        Ensuite, il ne restait qu'à jouer avec les formes : les étirer, 
        redimensionner une face pour que les formes correspondent avec une Formule 1. 
        Vu que le design est en \textit{low-poly}, nous avons rajouter des
        faces à certaines formes. Par exemple, avec 26 rectangles verticaux collés les 
        uns aux autres, on peut former un cylindre.
        Pour essayer d'être réaliste, nous avons regarder une 
        vidéo présentant une Formule 1 avec ses détails telles que le volant, 
        les suspensions, les ailerons et les moustaches de la voiture.

    \subsubsection{Anthony}
        Pour le site Web, nous avons décidé de reprendre l'esthétique du
        menu de notre jeu vidéo. Pour cela, nous avons divisé le site web 
        en quatre parties, accessibles par quatre boutons amenant aux quatre 
        sous-pages de notre site. La première est la présentation du projet, dans laquelle
        nous avons écrit une courte introduction, l'historique et les membres
        de l'équipe. La seconde concerne la réalisation du projet, qui contient
        la chronologie de la réalisation, les problèmes rencontrés et solutions
        envisagées pour contrer ceux-ci. Ensuite, une troisième page nous 
        offre des liens de téléchargement du jeu pour Windows et macOS, et
        le projet en version complète ou allégée.
        Enfin, la dernière page permet l'accès aux différentes ressources
        (liens des images, sons, logiciels...). C'est une sorte de bibliothèque 
        des sources de notre projet.
        Ainsi, il me manquera plus qu'à faire fonctionner certains boutons et 
        liens et rendre le site responsive, c'est a dire un site qui s'adpate en 
        fonction de l'appareil depuis lequel on le consulte, et 
        règler quelques bugs d'affichages pour finaliser le site Web. 


    \subsubsection{Xavier}
        Pour faire tourner un jeu, il faut que chaque élément soit relié avec
        les autres grâce à des liens dans le code. Dans notre cas, cela peut être
        appliqué aux voitures, qui doivent être des objets controllables et visbles.
        Pour cela, nous avons intégré les voitures dans le moteur de jeu \textsl{Unity}
        en rajoutant des composants tel que le \textit{Rigibody}, le centre de gravité,
        les \textit{Colliders}, et les \textit{Wheel Colliders}. Le premier est un élément
        permettant de simuler les propriétés physiques de la voiture, comme la masse ou la
        détection de collisions. Le second est un élément du \textit{Rigibody}, qui est sous
        la forme d'un point précis mis sur la voiture, et qui est le point d'application
        de la gravité. C'est un élément pratique, qui permet de définir comment la
        voiture va réagir dans les virages, ou comment celle-ci va \textit{drifter}.
        Le troisième élément à intégrer est en plusieurs parties, chacune pour un morceau
        de la voiture (capot, toit, ailes, etc.). Ce sont des zones virtuelles autour
        de la voiture qui permettent de la rendre solide, c'est à dire qu'on ne peut pas
        traverser, et qui ne peut pas traverser non plus les murs et autres éléments.
        Et enfin, les \textit{Wheel Colliders} sont des éléments appliqués sur chaque roue
        et qui jouent le rôle de moteur. Pour les roues arrières, ils transmettent l'action
        du joueur si il appuie sur les touches pour avancer et reculer, en entrainant les
        roues (et donc la voiture) vers la bonne direction. Et pour les roues avant,
        ils s'occupent de faire tourner la voiture en fonction des actions du joueur.
        Après cela, nous avons intégré le mode de jeu mutlijoueur. Nous avons décidé
        d'utiliser \textsl{Photon Cloud}, une solution gratuite, \textit{open source},
        et relativement "simple" à implémenter. Il y a une bonne documentation sur
        Internet, et de nombreux tutoriaux sur YouTube. Nous avons commencé par
        créer une scène pour la connection, qui, grâce à une interface graphique,
        permet de choisir un nom. Ensuite, nous pouvons créer une \textit{room}
        (instance de multijoueur), d'en chercher une déjà ouverte ou encore
        d'en joindre une directement si nous connaissons son nom. Enfin, une
        fois que la \textit{room} est créée, le jeu lance une scène comportant
        un circuit, et le jeu peut commencer.
\clearpage

    \section{Récit chronologique}
    \textbf{\textsc{Dans cette partie, nous vous présentons comment nous nous sommes organisés pour sortir notre jeu Bitume Legends, mais aussi comment chaque parties du jeu ont été réalisées}}
        \subsection{Janvier $\to$ Mars}
            \subsubsection{Multijoueur}
                Nous avons implémenté la base du multijoueur, non sans peines. Nous avons eu des problèmes de 
                synchronisation entre les joueurs mais également des problèmes de vues et de contrôle. C'est à dire 
                que nous contrôlions la mauvaise voiture et que nous ne voyions pas par notre caméra mais celle d'un 
                autre joueur. 
                Nous avons choisi de limiter la connexion à 4 personnes en simultané, pour éviter les 
                "embouteillages" dans le circuit tout en offrant un minimum de compétition. Il y a une interface de 
                connexion, avec différentes options (détaillées dans le manuel d'utilisation), nous permettant de 
                choisir un pseudo, de créer ou rejoindre une instance du jeu, et de commencer le jeu.
  
            \subsubsection{Programme Beta (\(\beta\))}
                Nous avons créé un questionnaire sur \textsl{Google Forms} afin que les premières personnes testant
                le jeu puisse nous signaler les bugs qu'elles ont rencontrés et nous les décrire le plus 
                précisément possible, dans l'optique de les corriger rapidement. Pour cela, ce questionnaire 
                possède plusieurs questions nous permettant de cibler le plus précisément possible le problème tels qu'un problème d'interface, un mode de jeu disfonctionnel mais encore un 
                potentiel cheater. 
                Toutefois, l'utilisateur peut nous contacter sur \textsl{Discord} ou \textsl{GitHub} si besoin.
        
            \subsubsection{Communication}
                Nous avons créé un compte \textsl{Instagram}\footnote{\(\mathtt{instagram.com/bitumelegends}\)}
                pour tenir informé nos potentiels premiers joueurs. 
                Ils pourront nous rejoindre sur \textsl{Discord} pour que nous puissions les inscrire dans le
                programme de tests. Nous avons également utilisé notre réseau d'amis mais également les différentes 
                présentation de ce que nous faisons dans notre lycée pour mettre en avant notre projet tout en leur 
                montrant un projet concret qui peut donner envie.
        
            \subsubsection{Site Internet}
                Le site Internet a pour but de présenter les membres du groupe, leur rôle et leur avis sur le projet.
                Il présente aussi l'historique et les idées de conception du jeu, du nom du jeu et celui du studio
                de développement. Ensuite, il nous permet d'expliquer comment nous avons réalisé le projet avec la
                chronologie de réalisation, c'est-à-dire les différentes étapes de réalisation du projet mais aussi
                les problèmes rencontrés et les solutions envisagées. Enfin, il contient aussi
                un espace dédié pour télécharger le projet et le jeu. Ce site est notre vitrine, il nous permet de 
                faire la promotion de \btmlgs.
=======
    \lipsum[1-2]


    \section{Reprise du Cahier des Charges}
    Le but ici c'est de réexpliquer le concept du jeu tel que présenté ds le cahier des charges. 
    On explique notre vision, ce que nous avons tenu à mettre ds le jeu et on parle pas trop des
    chose qu'on a dit mais pas fait.
    \clearpage
    \newpage


    \section{Chronologie temporelle du temps chronologique}
    \textbf\textsc{J'ai mis les points importants relatés ds les anciens rapports, le but va être de rédiger ça}
        \subsection{Janvier -> Mars}
            \subsubsection{Multijoueur}
                La base du multijoueur a été implémentée, non sans peines. 
                Nous pouvons nous connecter à 4 personnes (nous avons choisi ce nombre pour des questions 
                de simplicité) et commencer une partie. Il y a une interface de connection, avec différentes 
                options (détaillées plus tard dans ce document) puis le système de \textit{spawn} (apparition).
                Une musique de jeu a également été crée pour le multijoueur.
  
            \subsubsection{Programme Beta (\(\beta\))}
                Nous avons créé un questionnaire sur \textsl{Google Forms} afin que les premières personnes testant
                le jeu puisse nous signaler les bugs qu'ils ont rencontrés et nous les décrire le plus 
                précisément possible dans le but d'être corrigé rapidement. Pour cela, ce questionnaire 
                possède pour trois catégories. Tout d'abord, une question sur le mode de jeu où le bug 
                s'est produit. Puis une autre sur la nature du bug et enfin un paragraphe réservé
                pour nous décrire avec le plus de détails possible ce bug. Toutefois, si l'utilisateur
                trouve compliqué de décrire son bug, des liens pour nous contacter sur \textsl{Discord} 
                ou \textsl{GitHub} sont indiqués.
        
            \subsubsection{Communication}
                Au sujet de la communication, nous avons décidé de faire une publication sur \textsl{Instagram}
                \footnote{\(\mathtt{instagram.com/bitumelegends}\)}
                pour tenir informés nos potentiels premiers joueurs. Dans cette publication se trouve le lien 
                du site Internet où l'utilisateur pourra consulter se familiariser avec le menu de notre
                jeu, puisque le site reprend l'esthétique de ce dernier. Enfin l'onglet de la réalisation
                et celui de la présentation donnent l'occasion de nous connaître plus, de même pour le projet.
                Ils pourront enfin nous rejoindre sur \textsl{Discord} pour que nous puissions les inscrire dans le
                programme de tests.
        
            \subsubsection{Site Internet}
                Le site web a pour but de présenter les membres du groupe, leur rôle et leur avis sur le projet.
                Il présente aussi l'historique avec les idées de conception du jeu, du nom du jeu et celui du studio
                de développement. Ensuite, il nous permet d'expliquer comment nous avons réalisé le projet avec la
                chronologie de réalisation, c'est-à-dire les différentes étapes de réalisation du projet mais aussi
                les problèmes rencontrés et les solutions envisagées. Enfin, il contient aussi
                un espace dédié pour télécharger le projet et le jeu. Ce site est notre vitrine, il nous permet de faire
                la promotion de \btmlgs.
>>>>>>> 0737757309992d623e78eb6953edcad6021eead0
        
            \subsubsection{Menu}
                Le menu doit permettre d'accéder aux différents modes de jeux et fonctionnalités du jeu.
                Il doit servir de hub pour accéder au mode multijoueur, contre la montre et contre l'\AI.
                Il doit également permettre d'accéder au garage contenant les voitures du joueur. Un affichage
                du niveau du joueur, un bouton pour accéder à la remontée de bug et un menu d'options 
                sont également dans le menu du jeu. Au début du projet, le menu était basique. Il servait 
                simplement à pouvoir accéder aux différentes modes de jeux. Suite à cela, Melvyn et Victorien 
                ont commencé à travailler sur un design plus poussé que celui de base de \textsl{Unity}, 
<<<<<<< HEAD
                en adoptant notamment un thème de couleurs basé sur le logo du jeu. Tout d'abord, nous avons 
                réalisé une ébauche du design sur Canva, dans le but de partager des idées sur le look
                final du menu. Ensuite nous avons implémenté le design dans le jeu, en insérant un fond
                et des espaces pour les futures fonctionnalités, en implémentant les boutons et en changeant
                leur design. Nous avons également implémenté la charte graphique du jeu. Enfin, nous avons 
                ajouté des effets sonores et visuels aux boutons lorsqu'ils sont cliqués ou lorsque
                la souris passe dessus. Une musique composée par Melvyn tourne également en fond dans le menu.


        \subsection{Mars $\to$ Avril}
            Suite à la soutenance 1 du 10 mars, nous avons pris quelques jours de repos. Puis le 14, nous
=======
                en adoptant notamment un thème de couleurs basé sur le logo du jeu. Tout d'abord, une
                ébauche du design a été réalisée sur Canva, dans le but de partager des idées sur le look
                final du menu. Ensuite nous avons implémenté le design dans le jeu, en insérant un fond
                et des espaces pour les futures fonctionnalités, en implémentant les boutons et en changeant
                leur design. Nous avons également implémenté le code couleur du design. Enfin, nous avons 
                ajouté des effets sonores et de couleurs aux boutons lorsqu'ils sont cliques ou lorsque
                la souris passe dessus. Une musique tourne également en fond dans le menu.


        \subsection{Mars -> Avril}
            Suite à la soutenance 1 du 10 mars, nous avons prit quelques jours de repos. Puis le 14, nous
>>>>>>> 0737757309992d623e78eb6953edcad6021eead0
            avons mit en place nos objectifs pour la soutenance 2 vis-à-vis du cahier des charges, de
            l'état du jeu ainsi que des commentaires de la première soutenance.\\
            Nous avons travaillé en parallèle sur différents objectifs, certains plus longs que d'autres.
            Notamment l'\AI, qui a prit énormément de temps et son travail d'implémentation ayant
            commencé au début de cette période. Le 20 mars ce sont les mécaniques de conduite comme le
            drift ainsi que le \textit{sound design} des véhicules qui ont été implémentés. Puis le 25 la
            création de deux nouveaux circuits, \textsl{City} et \textsl{Port}. Le mois de mars s'est
<<<<<<< HEAD
            conclu par la création des menus des modes de jeu : \textsl{Timer} et \textsl{Solo}.\\
            Après 3 semaines, la première version de l'\AI était prête le 10 avril, ainsi que le
            comportement en fin et début de course le 15. Le 18, le garage et le choix de la voiture 
            sélectionnée, ainsi que la sélection de la difficulté et du circuit ont été mis en place avec
            succès avec le systeme de sauveguarde. Le 20, l'IA a connu sa version 2.0 avec une précision améliorée et fonctionnelle sur
=======
            conclu par la création des menus des modes de jeu \textsl{Timer} et \textsl{Solo}.\\
            Après 3 semaines, la première version de l'\AI était prête le 10 avril, ainsi que le
            comportement en fin et début de course le 15. Le 18, le garage et le choix de la voiture 
            sélectionnée, ainsi que la sélection de la difficulté et du circuit ont été mis en place avec
            succès. Le 20, l'IA a connu sa version 2.0 avec une précision améliorée et fonctionnelle sur
>>>>>>> 0737757309992d623e78eb6953edcad6021eead0
            les deux circuits.\\
            Durant la dernière semaine avant la soutenance, des implémentations mineures ou des
            corrections de bugs ont eu lieu. Le 22 avril le menu principal a été terminé et avec
            l'affichage de la voiture et le lien entre chaque sous-menus a été fait. Le 24 la musique a
            été correctement implémentée et le 25 le mode de jeu \textsl{Timer} a été corrigé pour enfin
<<<<<<< HEAD
            fonctionner correctement. Enfin, le 27 le site Internet est devenu \textit{responsive} et le
=======
            fonctionner correctement. Enfin, le 27 le site Web est enfin devenu \textit{responsive} et le
>>>>>>> 0737757309992d623e78eb6953edcad6021eead0
            28 le mode \textsl{Solo} est devenu fonctionnel.

            \subsubsection{Graphisme}
                Au lancement du projet, nous voulions créer nous même nos voitures et nos circuit à 
                l'aide du logiciel de modélisation 3D Blender. Toutefois, cela prends beaucoup de temps,
                temps qui n'est pas utilisé à coder le jeu, résoudre des bugs, etc. Nous avons donc fait
                un compromis entre notre volonté de créer nous même le design du jeu et la contrainte de
                temps. Nous nous sommes orientés alors sur l'ajout d'un \textit{asset} du \textsl{Unity
<<<<<<< HEAD
                Asset Store}, pour obtenir des voitures et bases de circuits. Ceci nous a permis de
=======
                Asset Store}, pour obtenir des voitures et des morceaux de pistes. Ceci nous a permis de
>>>>>>> 0737757309992d623e78eb6953edcad6021eead0
                créer directement dans \textsl{Unity} nos différents circuits et de gagner beaucoup de 
                temps. 
            
            \subsubsection{Sauvegarde}
<<<<<<< HEAD
                Pour que notre jeu ne soit pas une prise de tête à chaque lancement et devoir tout recommencer, nous 
                avons créé un système de sauvegarde. Il est composé de 2 principes : le \textsl{Save}, qui permet 
                d'enregistrer les données voulues et nécessaires au bon fonctionnement du jeu, et le 
                \textsl{Load}, qui permet de charger la sauvegarde lors du lancement du jeu. Ce système permet
                ainsi de sauvegarder la voiture sélectionnée et de la charger directement dans le menu.
                Il permet également de sauvegarder les paramètres sonores du jeu (volume de la musique) mais également 
                l'expérience du joueur, et ses voitures débloquées. À tout moment, le joueur peut réinitialiser sa 
                progression et recommencer le 
                jeu depuis le début.
=======
                Pour que notre jeu soit plus facile à appréhender, nous avons créé un système de 
                sauvegarde. Il est composé de 2 principe : le \textsl{Save}, qui permet d'enregistrer 
                les données voulues et nécessaires au bon fonctionnement du jeu, ainsi que le 
                \textsl{Load}, qui permet de charger la sauvegarde lors du lancement du jeu. Pour le 
                moment, il s'occupe uniquement de sauvegarder quelle voiture a été choisie dans le 
                garage et de l'enregistrer pour la charger lors du prochain lancement du jeu. À terme, 
                nous comptons sauvegarder l'expérience du joueur, son pseudo, ainsi que la possible 
                customisation de ses voitures.
>>>>>>> 0737757309992d623e78eb6953edcad6021eead0

            \subsubsection{Menu}
                Pour la première soutenance, nous avons fait une charte graphique et un design pour les 
                menus.
<<<<<<< HEAD
                Nous nous étions concentrés sur le menu principal. Pour la seconde soutenance, nous 
                avons travaillé les sous-menus des différents modes de jeu et les fonctionnalités internes.\\
                Nous avons implémenté le garage, permettant de sélectionner une voiture pour la course parmi 
                la liste de voitures disponibles, qui est ensuite sauvegardée. 
                Nous avons ensuite modifié le menu de telle façon que La voiture sélectionnée s'affiche 
                ensuite dans le menu principal à place du logo du jeu. Les menus des modes de
=======
                Nous nous étions concentrés sur le menu principal. Pour cette seconde soutenance, nous 
                avons travaillé les
                sous-menus des différents modes de jeu et les fonctionnalités internes.\\
                Nous avons implémenté le garage, permettant de
                sélectionner une voiture pour la course parmi la liste de voitures disponibles, qui
                est ensuite sauvegardée. La voiture sélectionnée s'affiche ensuite dans le menu principal
                à la place du logo du jeu. Les menus des modes de
>>>>>>> 0737757309992d623e78eb6953edcad6021eead0
                jeux \textsl{Timer} (course contre la montre) et \textsl{Solo} (contre une \AI) ont
                été implémentés de la même manière. Ils ont donc une structure similaire. Au sein de
                ces menus, nous retrouvons la sélection des circuits ainsi que la difficulté de la course et
                un moyen de revenir au menu précédent.\\
                Nous comptons ajouter un menu de réglages permettant de régler le volume du jeu, le volume
                des musiques, les touches permettant d'avancer ainsi que de changer de pseudo. Une 
                version simplifiée de ce menu, contenant la modification des touches de directions et le
                volume du jeu sera également disponible en course.

            \subsubsection{\AI}
                L'implémentation de l'\AI\, s'est bien déroulée. Nous avons pris du temps
                à nous décider sur quelle solution nous allons utiliser et sur comment
                l'\AI\, devrait se comporter une fois implémentée dans le jeu. Nous sommes partis sur une
                solution hybride entre celle intégrée dans \textsl{Unity} et une \textit{homemade}.
                Nous nous sommes basé sur le \textit{NavMesh} de \textsl{Unity}, puis nous avons
                travaillé à animer la voiture et à définir les points qu'elle devait franchir pour
                terminer le circuit. Ensuite est arrivée
                la (longue) partie de la calibration. Au début, notre IA se déplaçait aléatoirement,
                si bien qu'elle inventait le chemin à chaque fois sans prendre le circuit que nous
                avions dessiné. Après plusieurs jours de recherche, nous avons réussi à lui faire
                prendre uniquement le chemin prévu. Ensuite, il a fallu régler sa vitesse et sa
                précision, pour éviter qu'elle ne rentre dans chaque mur par souci de freinage en virage.\\
                Après ces quelques soucis, nous avons obtenu un mode \textsl{Solo} pratiquement
                fonctionnel. Nous avons rajouté à ceci les scripts qui nous permettent de gérer
<<<<<<< HEAD
                le départ et la fin de la course et nous étions bons. Il est à noté que l'\AI\ possede la même 
                physique que les autres voitures présentent dans le jeu.
=======
                le départ et la fin de la course et nous étions bons.
>>>>>>> 0737757309992d623e78eb6953edcad6021eead0

            \subsubsection{Gameplay}
                En début de course, un décompte avant le départ est donné, bloquant la voiture pour
                empêcher les faux départs. Tout au long du circuit, le joueur doit traverser une série de 
                balises, les \textit{checkpoints}, pour valider la course. Cela permet d'empêcher le 
                joueur de simplement faire demi-tour et de traverser la ligne d'arrivée pour gagner, ou
<<<<<<< HEAD
                de couper le circuit. Ici le joueur est forcé de passer par chaque \textit{checkpoints}
                dans le bon sens afin de pouvoir terminer la course,
                idem pour l'\AI dans le mode \textsl{Solo}.\\
=======
                de couper le circuit. Ici le joueur est forcé de passer par chaque \textit{checkpoints},
                dans le bon sens afin de pouvoir terminer la course.\\
>>>>>>> 0737757309992d623e78eb6953edcad6021eead0
                Chaque mode de jeu (hors multijoueur) comprend une sélection de difficulté.
                La difficulté pour le mode de jeu \textsl{Timer} est déterminée par le temps maximum pour 
                terminer la course. En revanche, pour le mode de jeu \textsl{Solo}, 
                elle réside dans la vitesse et la précision de l'\AI, ce qui permet d'affronter des 
                \textit{bots} plus ou moins forts. \\
                En mode \textsl{Solo}, la course est gagnée si l'on est arrivé en premier ou que l'on a 
                survécu au \textit{bot} qui peut nous infliger des dégâts. En mode \textsl{Timer}, elle 
                est gagnée lorsque l'on passe la ligne d'arrivée (et toutes les balises précédentes) avant
                la fin du temps imparti. En cas de défaite, pour chacun de ces deux modes de jeu la partie
                est terminée et l'on peut soit recommencer la course, soit revenir au menu pour changer de
                véhicule, de circuit ou de difficulté.\\
                Nous avons également mis en place un système de collision qui, en plus d'impacter la forme de 
                la voiture, peut aussi impacter son comportement (un mauvais choc sur une des roues avant ou 
                le coté de la voiture peut provoquer des soucis de direction).

            \subsubsection{Musiques}
                Notre jeu comporte des musiques originales composées sur \FL\, par Melvyn. Un objectif d'un 
                peu moins de 5 musiques différentes a été posé dans le cahier des charges. Suite aux 
                commentaires de la première soutenance où il nous a été demandé de plus se concentrer sur le
                jeu, seules une musique de menu et une musique de course ont été composées. Toutefois, nous 
                tenions à ce qu'il y ai des nouveautés sur l'aspect de l'ambiance du jeu. Alors un travail 
                sur l'implémentation de ces musiques a été faite, notamment lors du passage d'une scène à
                une autre sans que la musique ne cesse ou recommence depuis le début. Pour la troisième 
                soutenance, l'objectif des musiques sera rempli avec un système de sélection des musiques 
                avec la possibilité de modifier le volume de la musique dans un sous-menus de réglages comme
                dit ci-dessus dans la section menu.
            
<<<<<<< HEAD
            \subsubsection{Site Internet}
                Pour ce qui est du site Internet, la mise en page de ce dernier était déjà bien avancée. 
=======
            \subsubsection{Site Web}
                Pour ce qui est du site Web, la mise en page de ce dernier était déjà bien avancée. 
>>>>>>> 0737757309992d623e78eb6953edcad6021eead0
                Cependant, quelques ajouts ont été fait, notamment pour permettre la finalisation de la 
                page ressources, qui contient tous les outils que nous avons utilisé pour réaliser notre 
                jeu. De plus, la page réalisation a été terminée avec l'ajout de la partie 
                problèmes et solutions.
<<<<<<< HEAD
                Enfin, pour ce qui est de la partie la plus difficile du site Internet, le \textit{responsive 
                design}, (qui est de rendre le site Internet adaptable à toutes tailles d'écran), l'utilisation 
                du code HTML était indispensable. Nous avons donc cherché à comprendre comment le 
                \textit{responsive} fonctionnait à l'aide de tutoriels et implémenté chaque élément 
                pour les rendre indépendants entre eux et ainsi obtenir notre site Internet \textit{responsive}.
                
        \subsection{Avril $\to$ Juin}
            \subsubsection{Sauvegarde}
               Le système de sauvegarde est indispensable pour déterminer l'avancer d'un joueur dans le jeu. C'est 
               pourquoi nous avons décidé de mettre en place une manière de représenter cette avancement par un système 
               de gain et de niveau, qui est sauvegardé dans le jeu. L'utilisateur acquiert de l'expérience plus ou 
               moins vite en fonction de son niveau. Un joueur finissant souvent premier en multijoueur, battant l'IA et
               gagnant contre la montre en mode difficile se verra gagner beaucoup plus d'expérience qu'un joueur 
               arrivant souvent dernier en multijoueur mais aussi un autre, gagnant contre la montre en mode facile. 
               Toute cette expérience est nécessaire pour passer d'un niveau à l'autre. Un joueur démarre le jeu au 
               niveau 0 et grimpe les échelons de plus en plus difficilement (puisqu'il faut plus d'expérience entre 
               chaque niveau) jusqu'au niveau maximum, le niveau 4. A chaque niveau, le joueur débloque une nouvelle 
               voiture pour le récompenser de son activité dans le jeu.
               
            \subsubsection{Menu}
                Après avoir implémenté le système de gain et de niveau, il nous faut maintenant un moyen de représenter
                l'expérience et le niveau actuel du joueur. Pour cela, une barre représentant l'expérience du joueur est
                afficher dans le menu ainsi que son niveau. Bien évidemment, ces deux éléments sont mis à jour après 
                chaque partie du joueur. De plus, le bouton "Option" a été rajouté pour régler le son et la musique du 
                jeu, mais aussi pour permettre à l'utilisateur de choisir sous quelle résolution il veut jouer et avec 
                quelle qualité de texture.  
                
            \subsubsection{Multijoueur}
            \lipsum[33]
            
            \subsubsection{\AI}
            
            Nous avons totalement refait notre système d'\AI, car la précédente n'était pas vraiment fonctionnelle. Auparavant l'ancienne IA "glissait sur le sol" sans utiliser les contrôles du véhicule. Désormais le nouveau système d'\AI prend en compte le système de physique du jeu. Ce nouveau système fonctionne par l'utilisation d'une série de \textsl{Waypoints} et de \textsl{Breakzones}. Ces \textsl{Waypoints} sont des zones circulaires d'une certaine taille que la voiture doit traverser dans l'ordre. Elle établit un chemin le plus court en fonction du placement des points et de son placement et le calcule en prenant en compte les obstacles sur le chemin. Le calcul se fait en direct afin de lui permettre de retracer un nouvel itinéraire jusqu'au prochain \textsl{Waypoints} dans le cas où elle aurait été déviée de sa course, par un joueur plus particulièrement. Les \textsl{Breakzones} sont des zones de taille modifiable qui forcera le bot à atteindre une vitesse définie au préalable afin d'anticiper des virages. Ce système permet à la voiture d'avoir le meilleur comportement possible et laisse beaucoup de place à l'optimisation, afin d'avoir le meilleur parcous possible. Sur chaque circuit nous comptons 3 {\AI}s, une pour chaque mode de difficulté. Elles se différencient par le véhicule utilisé, plus on augmente en difficulté, plus le véhicule sera sportif, nerveux et rapide. Bien que toutes les IA malgré la difficulté aient les mêmes \textsl{Waypoints} et \textsl{Breakzones}, ils ont été optimisés afin de pouvoir fonctionner pour tous les bots, et même d'être compétitifs.
            
            \subsubsection{Gameplay}
            
            Pour avoir un gameplay plaisant et attirant, nous avons ajoutés des objectifs ainsi qu'une interface simpliste, mais qui permet de donner un minimum de sensation au joueur. Dans le jeu, vous aurez la possibilités de mettre pause, et ainsi de relancer le jeu ou alors de quitter et revenir au menu. Une fois la partie terminée, vous pourrez encore une fois, soit relancer, soit quitter. Nous avons ajouté un compteur de vitesse, ainsi qu'un décor assez dense, de telle manière à ce que je le joueur ressente un certain dynamisme ainsi qu'une sensation de vitesse Nous avons également adapté les circuits de façon à ce que les joueurs ne soient pas perdus dans le circuit et sachent où aller instinctivement. Pour ce qui est des objectifs, il y a la volonté de gagné la course, mais également le déblocage de voitures dans votre garage. Vous commencez donc niveau 0, et si vous voulez jouer avec l'une des voitures les plus puissante jeu, vous devez gagner plusieurs courses, ce qui vous permet de connaître mieux les différents circuits, mais également les voitures que vous conduisez   
            
            Suite aux remarques de la deuxième soutenance, nous nous sommes rendus compte du fait que nous avions un trop grand nombre de véhicules. Bien que la diversité de choix nous tenaient à coeur, la grande maorité des voitures n'avait rien de particulier à offrir, nous avions simplement réutilisé des paramètres fonctionnels à toutes les voitures sans trop nous y attarder. Cependant afin d'avoir un jeu misant sur la qualité et non pas sur la quantité, nous avons opté pour 5 véhicules différents, qui cette fois-ci possèdent des caractéristiques leur étant propres et correspondant au type de véhicule. Chaque véhicule a des performances différentes, plus sportives à hauts niveaux.
            
            \subsubsection{Sound Design et Musiques}
            
            A l'origine on prévoyait une demi-douzaine de musiques pour le jeu, plusieurs en menu et plusieurs en course. Cependant le temps dédié à la composition musicale s'est avéré trop grand comparé au temps dédié à la création du jeu. Melvyn a donc décidé de ne pas en composer d'autre et s'est concentré sur le sound-design du jeu. Un des objectifs pour la troisième soutenance était également d'avoir un son caractéristique pour chaque véhicule car comme en vrai, en fonction du véhicule et de ses caractéristiques, le son émis est différents. Il a fallu récupérer les sons de chaque modèle utilisé, les traiter et gérer le comportement en volume et en hauteur en fonction du nombre de t/min. Les voitures modifiées peuvent également avoir un bruit de turbocompresseur ou de supercompresseur.
            
            \subsubsection{Site Internet}
            \lipsum[33]
            
            

    \section{Partie Technique}
    Pour la réalisation de ce projet, certaines étapes étaient incontournables. C'est pourquoi nous allons vous les expliquer en passant par les nombreux outils que nous avons utilisé et en quoi ils nous ont servi.
        \subsection{Collaboration}
            La création de jeux en solo ne demande pas de partage de données, contrairement 
            à celle en équipe. Nous avions besoin d'un système de collaboration et de partage des données où nous 
            pourrions nous échanger les fichiers relatifs au jeu. 
            Nous avons donc créé un ensemble de \textit{repositories} (ou \textit{repo}) sur \textsl{Github}
            \footnote{\(\mathtt{github.com/Bitume-Legends-Crew}\)}, répartis dans une organisation,
            chacun pour un usage bien spécifique. Le premier est donc le \textit{repo} de
            notre jeu, nommé \textit{game}, qui contient le projet au format \(\mathtt{.unity}\)
            et des dossiers contenants les différents assets et autres ressources,
            nécessaires à l'exécution du jeu. 
            Le second est un \textit{repo} consacré exclusivement aux différents rapports ou au cahier des charges
            que nous devons fournir. Il est composé à 99\% de .\TeX\, et 1\% de \(\mathtt{.pdf}\).
            Enfin, le dernier \textit{repo} est celui dédié à notre site Internet, qui
            possède une double fonction : il nous permet de collaborer sur le site
            mais aussi de l'héberger grâce à \textsl{GitHub Pages}.\\
            \indent Ainsi, nous pouvons toujours être à jour sur la bonne version du jeu,
            du site ou des rapports, tout en étant géographiquement à distance les uns 
            des autres.

        \subsection{Communication}
            Pour communiquer entre nous et avec notre équipe de \(\beta\)-testeurs, nous
            avons créé un serveur Discord \footnote{\(\mathtt{discord.gg/5NR43GHUBD}\)}
            découpé en multiples 
            \textit{channels}, ayant chacun une mission précise pour ne pas mélanger les
            informations. Ce serveur est aussi le lieu de nos réunions hebdomadaires 
            (ou plus fréquemment en cas de soutenance). De plus, depuis le site Internet, nous
            avons mis un lien vers un questionnaire de remontée de bugs, via un \textsl{Google Forms}.
            Nous avons également mis en place un calendrier collectif lors de la
            dernière semaine avant la soutenance. Cela nous a offert une vision plus
            claire des tâches individuelles et de les découper en plages horaires afin de 
            respecter les \textit{deadlines} et de ne pas avoir d'erreurs de versions
            entre nous.

        \subsection{3D}
            Comme moteur de jeu, nous avons utilisé \textsl{Unity}. Comme Xavier utilise un Mac 
            et que le reste du groupe est sous Windows, nous utilisons la version 2021.2.7f1 qui
            fonctionne sur les deux OS. Nous avons décidé de geler la version pour limiter au
            maximum les problèmes d'incompatibilité entre nous. Nous avons choisi \textsl{Unity}
            pour sa simplicité de prise en main et une fonctionnalité très utile :
            l'\textsl{Asset Store}. C'est une plateforme où nous pouvons acheter ou
            utiliser gratuitement des ressources telles que des bâtiments, des voitures ainsi que
            des personnages. En plus de l'utilisation de l'\textsl{Asset Store} pour notre première
            voiture, nous avons aussi utilisé \textsl{Blender}, logiciel de modélisation 3D. Il
            nous a permis entre autres de créer une Formule 1 de 2021 qui sera utilisée et
            implémentée plus tard dans le jeu.

        \subsection{IDE}
            Pour écrire notre code en C\#, nous utilisons l'IDE \textsl{Rider} de 
            \textsl{JetBrains}. Il possède une bonne intégration de Unity, et nous
            y sommes bien habitué, c'est celui que nous utilisons au quotidien pour
            nos TPs de programmation. Pour faire certains tests, très précis et qui
            ne nécessitent pas de beaucoup de ressources, nous utilisons \textsl{Vim}
            directement dans notre terminal.
            Pour collaborer sur le rapport et les autres documents en \LaTeX\, 
            pour éviter de désigner un "esclave \LaTeX", nous utilisons le site 
            \textsl{Overleaf}\footnote{\(\mathtt{overleaf.com}\)}. 
            À la manière d'un document en ligne, comme 
            un \textsl{Google Docs} ou autre, nous pouvons écrire en
            simultané et compléter à quatre cerveaux les documents demandés.
            
        \subsection{Installateur} 
         
        Pour pouvoir installer notre jeu sur Windows et Mac, il nous a fallu créer un fichier \textsl{.exe} permettant d'installer tous les fichiers nécessaires pour faire fonctionner \btmlgs.
        Pour ce faire, nous avons utilisé \textsl{Inno Setup}, un logiciel libre et gratuit. Il nous a permit d'ajouter un fichier sauvegarde spécial pour certains \(\beta-\)testeurs qui nous ont demandé d'avoir un profil au niveau maximum pour pouvoir testé le jeu dans son intégralité.


    
    
    
    \section{Notre ressenti au long du projet}
    \subsection{Janvier $\to$ Mars}
        \subsubsection{Anthony}
        \textit{L’organisation du travail au sein du groupe était plutôt bonne.
        Dès le début, nous avons créé un repo \textsl{GitHub} afin de partager notre travail 
        effectué ainsi qu’un \textsl{Discord} pour se mettre d’accord sur les différentes 
        deadlines à propos du travail à effectué chaque semaine. Ainsi nous 
        effectuons toutes les semaines, des réunions en vocal, dans le but 
        d’expliquer aux autres membres du groupe ce que l’on a fait dans la 
        semaine. A propos du site web, j’ai eu du mal à manipuler \textsl{Bootstrap Studio} 
        du fait du manque de tutoriaux pour apprendre à utiliser \textsl{Bootstrap Studio}
        sans de développement en HTML. J’ai donc dû apprendre comment fonctionne 
        ce logiciel et après avoir réalisé que le principe était de mettre tous 
        les éléments liés dans une seule colonne, \textsl{Bootstrap} est devenu toute de 
        suite très facile et m’a permis de réaliser rapidement le site web.}
  
        \subsubsection{Melvyn}
        \textit{Entre la première présentation du projet et le rapport de soutenance
        j'ai découvert les différentes difficultés liées à la réalisation d'un projet.
        J'ai eu du mal avec l'investissement personnel et la gestion du temps durant 
        ce mois et demi. Je compte améliorer ma manière de travailler et changer mon
        processus de composition afin de pouvoir être plus investi dans le projet et 
        produire plus de contenu. Je suis heureux de pouvoir travailler sur un style de 
        jeu que j'aime. Je suis également ravi de pouvoir à nouveau travailler sur des
        projets musicaux et de pouvoir mettre ma passion au service du projet.
        Un aspect intéressant du projet a également été de pouvoir implémenter nos idées
        en utilisant des méthodes qui m'étaient dès lors inconnues. Ce fut un challenge 
        mais j'ai trouvé cela amusant et suis prêt pour la suite.}
 
        \subsubsection{Victorien}
        \textit{Quelques années auparavant, j’avais déjà créé un jeux-vidéo 
        sur un autre moteur de jeux, \textsl{Unreal Engine} de \textsl{Epic Games}. 
        Je n’avait cependant pas eu à coder car il suffisait de créer des
        \textsl{Blueprints} (élements de code préfabriqués qu’il faut relier 
        entre eux). De plus, c’était un jeu développé en solo, pas en équipe.
        Je ne savais pas spécialement par quoi commencer au vu de ce que l’on
        avait prévu. Après avoir fixé des objectifs hebdomadaires grâce aux 
        réunions, les choses étaient cadrées et j’ai pu être plus efficace sur
        le développement du jeu. Il a fallu également faire attention à notre 
        emploi du temps. Bien qu’il y eût les vacances pour avancer, il y avait 
        à la rentrée les Midterms et 5 jours plus tard première soutenance. 
        Nous devions être organisé et voir plus loin que la semaine suivante.}
  
        \subsubsection{Xavier}
        \textit{La gestion du projet s'est très bien passée. Nous nous sommes
        rapidement mis d'accord sur une organisation qui nous convenait. Cela a
        permit d'avancer en restant tous sur la même longueur d'onde. Tout le
        groupe est très motivé, ce qui est nécéssaire pour un projet de ce type.
        J'ai eu cependant beaucoup plus de mal à créer le mode multijoueur. J'ai
        dû m'y reprendre à trois fois en tout, car chaque version antérieure
        ne marchait pas ou n'acceptait pas les ajouts de fonctionnalités. Cela a
        occupé une bonne partie de mes soirées depuis mi-Janvier. Malgré cela, 
        j'ai pu apprendre beaucoup de choses dans pleins de domaines, de
        l'hébergement Web en reprogammant les \textit{DNS} de \textsl{GitHub}, 
        ou encore en recherchant comment créer une organisation \textsl{GitHub} 
        pour gérer le projet.}\\\\
    
     \subsection{Mars $\to$ Avril}
         \subsubsection{Anthony}
                \textit{Au sujet du site Web, j'ai eu beaucoup de mal à implémenter le responsive design
                puisque l'application Bootstrap Studio 5 ne suffisait pas, et l'usage du code HTML était
                obligatoire. Ayant peu de connaissance dans ce langage, j'ai donc demandé
                de l'aide à mon groupe pour m'aider à le mettre en place. Ainsi après avoir compris le
                code généré par Bootstrap pour transformer chaque élément en responsive, d'autre
                problèmes de mise en forme ont complexifié la tâche entre les éléments responsive et
                non. De plus, la cohésion d'équipe a été fortement accentuée, ce qui a permis de régler 
                les problèmes beaucoup plus rapidement. C'est pourquoi une grande avancée dans le 
                jeu vidéo s'est faite ressentir par rapport à la soutenance 1. Du fait de cette avancée, 
                ma motivation et l'envie de faire de Bitume Legends un bon jeu de voiture est encore 
                plus forte que précédemment.}

            \subsubsection{Melvyn}
                \textit{J'ai remarqué une légère baisse d'implication de ma part entre la 
                première et la deuxième soutenance, notamment une semaine où le 
                travail était plus que minime de ma part. J'ai également dû plus 
                travailler avec les autres, comparé à la première soutenance où l'on
                travaillait un peu plus dans notre "domaine d'expertise" à chacun 
                au lieu de nous entraider. Je me suis reprit quelques temps avant la
                soutenance et je suis fier de l'avancée du projet. Je pense que ce projet
                a beaucoup de potentiel. Il y a eu de grandes améliorations graphiques et
                techniques en ces quelques semaines et notre jeu ressemble enfin à un jeu.
                J'ai beaucoup d'espoirs pour la suite. Là où je m'étais trop concentré sur
                la musique lors de la première soutenance, je me suis plus tourné vers
                le graphisme, le gameplay et la physique du jeu, il était temps de vraiment
                faire du code...}

            \subsubsection{Victorien}
                \textit{Suite à la première soutenance, la première idée que j'ai eu a été
                de vouloir développer le jeu et s'amuser dessus. Le but étant
                d'avoir un jeu plaisant, joli et agréable à jouer. J'ai donc passé
                de nombreuses heures à implémenter les différents modes de jeu, résoudre les 
                \textsl{bugs}. 
                Je suis très satisfait de mon travail. De plus, ceci m'a permit de m'améliorer en 
                \textsl{C\#} ainsi qu'en programmation orientée objet. C'est un vrai plaisir de coder 
                le jeu et de voir notre travail
                porter ses fruits. Là où au départ \btmlgs\, n'avait pas forcément de style et
                n'attirait pas l'œil, il est maintenant beaucoup plus attractif suite à
                la refonte graphique du jeu ainsi que notre avancée.}

            \subsubsection{Xavier}
                \textit{Depuis la dernière soutenance, je me suis bien amusé à faire l'\AI.
                Cela était sympa au début puis plus le temps avançait, plus les problèmes arrivaient.
                Cela m'a fait passé par tous les états possibles, de la joie intense à la dépression
                profonde. Malgré cela, l'\AI a sûrement été la partie que j'ai préféré faire.
                Pour le reste, je suis très fier de l'avancée que nous avons, nous sommes à jour sur 
                notre planning et le jeu est très plaisant à jouer. Nous sommes très content de ce que 
                rendent les graphiques et les voitures, ce qui était le point noir de la dernière 
                soutenance. Bref, le jeu va vraiment être super sympa et cela nous rend heureux ! }
                
     \subsection{Avril $\to$ Juin}
         \subsubsection{Anthony}
         \textit{Ce projet m'a servit à découvrir le travail de groupe sur une assez longue période, avec tous ses inconvénient et ses avantages. Il m'a était très enrichissant, vis à vis de régler les bugs en cherchant des solutions le plus vite possible. De plus, j'ai ressenti un vrai travail d'entraide entre nous mais aussi en terme de répartition des tâches ce qui a considérablement mis un boost dans l'avancer de notre jeu. Au final, je suis plutôt très content du résultat du jeu, nous avons était optimiste au début du jeu sur ce qu'on voulait à la fin et nous avons pratiquement remplit tous nos objectifs. }
             
        
        \subsubsection{Melvyn}
        \textit{Suite à la deuxième soutenance, je me suis mieux investi dans le projet, notamment dans le travail de l'\AI. Je me suis amusé à faire de l'optimisation du système et le rendre compétitif. Je me suis vraiment épanoui dans le projet à partir du moment ou nous avons véritablement travaillé tous ensemble. Ce projet a été très enrichissant à titre personnel et également professionnellement. C'est le premier pavé sur la route des futurs travaux de groupes en tant qu'ingénieur. Bien que le fait que ce projet devait être travaillé sur notre temps personnel, je n'ai pas eu l'impression d'avoir manqué ce temps. Je me réjouis de travailler à nouveau sur un projet d'une telle envergure, surtout avec l'expérience acquise ce semmestre.}
        
             
        
        \subsubsection{Victorien}
             \textit{Ce projet a été très enrichissant. Il m'a permit de comprendre certains concept que je n'avais pas spécialement compris en C\#, mais également sur le plan humain où il nous a fallu rester soudé et donc trouver des moyens de rester motivé ou d'être motivé, et de rester une équipe. J'ai pu également remarqué que l'on était bien plus efficace lorsque nous étions tous en présentiel à quatre autour d'une table et l'ambiance était bien plus plaisante, ce qui sera utile lors des futurs projets. Poser également les bases avec des réunion et une manière de travailler en début de projet nous a également été favorable et nous a permit d'être plus efficace rapidement. Les synthèses à la fin de chaque soutenance nous ont également permit de savoir sur quoi accès nos prochains objectif. Cette méthode nous sera, comme dit plus, certainement tous utile lors de nos prochains projet. Enfin, je suis très contents de voir ce projet aboutir, de savoir que nous sommes parti de rien et que nous avons un jeu fonctionnel et sur lequel chacun des membres des groupes mais également les joueurs se plaisent.}
        
        \subsubsection{Xavier}
             \textit{Je suis très content de notre projet. Nous avons su aller jusqu'au bout de nos idées phares, fait le tri qui s'imposait parmi les non essentielles, et créé un jeu qui nous donne vraiment envie de jouer. Malgré les contretemps, et les soucis d'organisations, propres à chaque travail de groupe, nous avons pu avancer selon notre timeline définie en début de projet, dans notre cahier des charges.}
        
        
        
    \section{Ressentis de certains testeurs}
        Le programme de \(\beta-\)testing a permis que chaque utilisateurs, ou testeurs, puissent nous remonter les bugs éventuels lorsqu'ils jouent au jeu. De plus, ce dernier sert à remonter ce que le joueur à le plus aimer, le moins et même ce qu'il aimerait voir. Ce programme à démontrer une très grande efficacité dans l'originalité du jeu, notamment grâce aux remontées d'idées des testeurs mais nous a permis aussi d'avoir un jeu très solide, avec très peu de bugs grâce aux nombreux testeurs. Cependant, deux endroits étaient réservés pour le programme de \(\beta-\)testing, un sur le site Internet avec le Google Form et un autre sur notre discord. Nous avons ainsi remarqué que les testeurs ont pour, quasi de la totalité, utilisés le serveur discord.
        Maintenant, nous avons décidé de vous transmettre quelque retours, avis du jeu de certains testeurs.
        
        
    
    
    
    \section{Problèmes rencontrés et solutions}
        \subsection{Janvier $\to$ Avril}
        
         \subsubsection{Implémentation des voitures}
                L'implémentation de la physique de voitures fut complexe, en particulier au vu des 
                nombreuses variables impactant le comportement d'une voiture. Il a fallut gérer le 
                poids, la vitesse, la puissance en chevaux du moteur, le couple maximum, l'angle et la 
                vitesse de braquage pour la direction, la force et la vitesse de freinage ou encore 
                l'inertie du moteur et la hauteur du centre de gravité.\\
                Suite à l'implémentation de la physique des voitures, nous nous sommes rendu compte que 
                les modèles des voitures que nous avions ne permettaient pas son bon fonctionnement. Il 
                a fallu dans un premier temps modifier les \textit{prefabs} des voitures, entre autres 
                leurs \textit{rigibody} qui empêchaient les roues de tourner et de considérer qu'elles 
                touchaient le sol. Suite à cela, il a fallu également modifier les roues des voitures en
                unifiant la jante ainsi que le pneu, chose qui n'était pas faite, avant pour pouvoir 
                entraîner l'essieu et synchroniser la direction. Après avoir résolu ce problème sur une 
                voiture, il a fallu appliquer la solution aux autres, ce qui explique également pourquoi les 
                voitures ont pour l'instant la même physique. Toutefois, les voitures ainsi que les 
                collisions sont fonctionnelles.\\
                Un autre problème fût aussi celui du sound-design de la voiture. Plusieurs éléments dans
                une voiture produisent du son et n'ont pas le même comportement en fonction du poids et
                la vitesse du véhicule ou encore de la puissance du moteur. Il a fallut gérer le passage
                des rapports, le bruit du moteur dont le \textit{pitch} et le volume changeait selon la
                vitesse ou le type du véhicule (un pick-up a un bruit différent d'une Supercar). Avec le
                passage de rapport il y avait également le bruit du \textsl{turbocompresseur} ou du
                \textsl{supercharger} à gérer.\\
                Au final nous avons trouvé certains paramètres permettant à la voiture d'avoir un bon
                comportement et d'avoir un bon son. Il ne nous reste plus qu'à appliquer ces paramètres
                aux autres voitures.

            \subsubsection{\AI}
                En implémentant notre IA, nous avons fait face à de nombreux problèmes divers et variés. Pour 
                commencer, les voitures autonomes ne voulaient pas passer par les bons endroits. Elles 
                faisaient demi-tour sur la ligne de départ et passaient à travers les murs ou entre les 
                plots et se stoppaient sans raison au milieu du circuit. Pour résoudre ce problème, nous
                avons compris comment fonctionnait le \textit{NavMesh} puis appris à séparer les 
                différentes couches de circuit : ce sur quoi la voiture peut rouler et les obstacles. Le 
                \textit{NavMesh} est un calque qui est apposé au dessus du circuit et qui permet de
                définir une zone sur laquelle les voitures autonomes peuvent rouler et calculer leur
                itinéraire. Après cela, nous avons recalculé le \textit{NavMesh} et remis des plots
                ou des barrières là où c'était nécessaire.\\
                Le deuxième problème consistait en un arrêt inattendu et systématique au milieu du 
                circuit. Pour trouver comment résoudre ce problème, nous avons exploré plusieurs pistes. Premièrement vérifier si la direction de la voiture était la bonne, puis si les 
                \textit{checkpoints} fonctionnaient correctement. Après de longues recherches 
                infructueuses, nous avons trouvé que le problème était lié à l'absence de composants 
                dans la voiture, permettant de comptabiliser les collisions avec les 
                \textit{checkpoints}. Nous avons donc rajouté ce composant et changé la façon de définir
                une trajectoire. Et après ce travail, les voitures autonomes arrivent à la fin du 
                circuit.

            \subsubsection{Implémentation des musiques}
                Pour les menus nous avons préféré créer plusieurs scènes pour les sous-menus accessibles
                par le menu principal., plutôt qu'une seule scène qui se transforme à l'appui d'un 
                bouton sur le menu.
                Bien que pratique pour les scripts individuels à chaque modes de jeux et sous-menus,
                cela compliquait l'implémentation de la musique. Il a fallu trouver un moyen d'empêcher
                les musiques de s'arrêter à chaque chargement d'un nouveau menu. Pour cela il
                fallait faire en sorte que la musique fasse partie des éléments
                \textit{DontDestroyOnLoad}, qui permet qu'une fois la musique chargé, elle restera 
                possible d'accès
                même après changement de scène. Il nous a ensuite fallu dans le script s'occupant de la 
                musique, de lui faire référence et de lui dire de ne pas s'arrêter lors d'un changement
                de scène.\\
                Un problème qui a suivi est le fait que bien que la musique ne s'arrête plus, à chaque
                ouverture d'un menu une autre instance de la musique se lançait. Pour remédier à cela il
                a fallut préciser dans le script que toute nouvelles musique dans ne scène de menu ne
                devait pas se lancer si une autre se jouait déjà.\\
                Enfin un dernier problème a eu lieu lors de l'implémentation des musiques de courses. La
                musique de menu continuait de jouer par dessus celle de course. Il a alors fallut
                ajouter dans le script le cas où lorsque l'on était en course, il fallait mettre la 
                propriété \textit{SetActive}, qui permet d'activer ou désactiver un élément. Par 
                exemple, si l'on entrait en course, il a fallu désactiver la musique de menu et activer 
                la musique de course.
        
        
        
        
        
        \subsection{Avril $\to$ Juin}
    
            \subsubsection{Implémentation des voitures}
                L'implémentation de la physique de voitures fut complexe, en particulier au vu des 
                nombreuses variables impactant le comportement d'une voiture. Il a fallut gérer le 
                poids, la vitesse, la puissance en chevaux du moteur, le couple maximum, l'angle et la 
                vitesse de braquage pour la direction, la force et la vitesse de freinage ou encore 
                l'inertie du moteur et la hauteur du centre de gravité.\\
                Suite à l'implémentation de la physique des voitures, nous nous sommes rendu compte que 
                les modèles des voitures que nous avions ne permettaient pas son bon fonctionnement. Il 
                a fallu dans un premier temps modifier les \textit{prefabs} des voitures, entre autres 
                leurs \textit{rigibody} qui empêchaient les roues de tourner et de considérer qu'elles 
                touchaient le sol. Suite à cela, il a fallu également modifier les roues des voitures en
                unifiant la jante ainsi que le pneu, chose qui n'était pas faite, avant pour pouvoir 
                entraîner l'essieu et synchroniser la direction. Après avoir résolu ce problème sur une 
                voiture, il a fallu appliquer la solution aux autres, ce qui explique également pourquoi les 
                voitures ont pour l'instant la même physique. Toutefois, les voitures ainsi que les 
                collisions sont fonctionnelles.\\
                Un autre problème fût aussi celui du sound-design de la voiture. Plusieurs éléments dans
                une voiture produisent du son et n'ont pas le même comportement en fonction du poids et
                la vitesse du véhicule ou encore de la puissance du moteur. Il a fallut gérer le passage
                des rapports, le bruit du moteur dont le \textit{pitch} et le volume changeait selon la
                vitesse ou le type du véhicule (un pick-up a un bruit différent d'une Supercar). Avec le
                passage de rapport il y avait également le bruit du \textsl{turbocompresseur} ou du
                \textsl{supercharger} à gérer.\\
                Au final nous avons trouvé certains paramètres permettant à la voiture d'avoir un bon
                comportement et d'avoir un bon son. Il ne nous reste plus qu'à appliquer ces paramètres
                aux autres voitures.
                
            \subsubsection{Implémentation des musiques}
                Pour les menus nous avons préféré créer plusieurs scènes pour les sous-menus accessibles
                par le menu principal., plutôt qu'une seule scène qui se transforme à l'appui d'un 
                bouton sur le menu.
                Bien que pratique pour les scripts individuels à chaque modes de jeux et sous-menus,
                cela compliquait l'implémentation de la musique. Il a fallu trouver un moyen d'empêcher
                les musiques de s'arrêter à chaque chargement d'un nouveau menu. Pour cela il
                fallait faire en sorte que la musique fasse partie des éléments
                \textit{DontDestroyOnLoad}, qui permet qu'une fois la musique chargé, elle restera 
                possible d'accès
                même après changement de scène. Il nous a ensuite fallu dans le script s'occupant de la 
                musique, de lui faire référence et de lui dire de ne pas s'arrêter lors d'un changement
                de scène.\\
                Un problème qui a suivi est le fait que bien que la musique ne s'arrête plus, à chaque
                ouverture d'un menu une autre instance de la musique se lançait. Pour remédier à cela il
                a fallut préciser dans le script que toute nouvelles musique dans ne scène de menu ne
                devait pas se lancer si une autre se jouait déjà.\\
                Enfin un dernier problème a eu lieu lors de l'implémentation des musiques de courses. La
                musique de menu continuait de jouer par dessus celle de course. Il a alors fallut
                ajouter dans le script le cas où lorsque l'on était en course, il fallait mettre la 
                propriété \textit{SetActive}, qui permet d'activer ou désactiver un élément. Par 
                exemple, si l'on entrait en course, il a fallu désactiver la musique de menu et activer 
                la musique de course.

            \subsubsection{\AI}
                En implémentant notre IA, nous avons fait face à de nombreux problèmes divers et variés. Pour 
                commencer, les voitures autonomes ne voulaient pas passer par les bons endroits. Elles 
                faisaient demi-tour sur la ligne de départ et passaient à travers les murs ou entre les 
                plots et se stoppaient sans raison au milieu du circuit. Pour résoudre ce problème, nous
                avons compris comment fonctionnait le \textit{NavMesh} puis appris à séparer les 
                différentes couches de circuit : ce sur quoi la voiture peut rouler et les obstacles. Le 
                \textit{NavMesh} est un calque qui est apposé au dessus du circuit et qui permet de
                définir une zone sur laquelle les voitures autonomes peuvent rouler et calculer leur
                itinéraire. Après cela, nous avons recalculé le \textit{NavMesh} et remis des plots
                ou des barrières là où c'était nécessaire.\\
                Le deuxième problème consistait en un arrêt inattendu et systématique au milieu du 
                circuit. Pour trouver comment résoudre ce problème, nous avons exploré plusieurs pistes. Premièrement vérifier si la direction de la voiture était la bonne, puis si les 
                \textit{checkpoints} fonctionnaient correctement. Après de longues recherches 
                infructueuses, nous avons trouvé que le problème était lié à l'absence de composants 
                dans la voiture, permettant de comptabiliser les collisions avec les 
                \textit{checkpoints}. Nous avons donc rajouté ce composant et changé la façon de définir
                une trajectoire. Et après ce travail, les voitures autonomes arrivent à la fin du 
                circuit.

       
        
            \subsubsection{Multijoueur}
                 \lipsum[37-38]
        
        
    
    


    \addcontentsline{toc}{section}{Conclusion}
    \section*{Conclusion du projet}
        \subsection{Objectif atteint ?}
        Par nécessité nous avions créé un compte Instagram pour notre groupe afin d'informer et de discuter avec la communauté sur les avancements du jeu. Cependant notre serveur Discord a prit une certaine ampleur et la majorité des annonces et des discussions vis-à-vis du jeu avec nos \textsl{Beta-testeurs}. Nous avons donc choisis d'abandonner l'idée de maintenir le compte Instagram aux profits de Discord.\\
        Lors de la deuxième soutenance, une remarque a été faite concernant notre trop grand nombre de voitures, ce qui nous a amené à revoir nos objectifs, notamment vis-à-vis de la customisation des véhicules. L'optimisation de chaque véhicule même lorsque leur nombre était réduit nous a prit déjà un temps considérable, alors si l'on devait permettre une modification des statistiques du véhicule tout en les gardant fonctionnelles et équilibrées tout le reste du projet aurait été impacté. De plus, un système de customisation aurait impliqué un système de gain d'argent virtuel pour acheter ces modifications en plus du gain d'expérience. Finalement nous nous sommes concentrés sur la poignée de véhicules déblocables afin de leur donner une certaine âme et une identité via leur caractéristiques et leur son.\\
        Un nombre d'une demi-douzaine de musique avait également été prévue à l'origine, encore une fois le problème venait du temps que cela aurait nécessité par rapport aux fonctionnalités importantes du jeu.
        Si l'on se réfere à notre "RAPPORT TEMPS PROJET"
        \\\lipsum[36]
    \subsection{Avenir de Bitume Legends}
        + projet Hopital à la toussaint 2022\\
        Entre la première et la deuxième soutenance notre jeu a gagné en popularité au sein des autre groupes et même au sein de l'école. En effet la refonte graphique du jeu ainsi que la création des modes de jeu \textsl{Solo} et \textsl{Timer} a plu à certains qui aiment également les jeux de voiture. De plus le lancement de la \textsl{Beta} ouverte a fait découvrir le projet zen dehors de l'école. Les retours étaient globalement très positifs et nous permettaient une découverte et une résolution plus facile des bugs. En restant proche de nos testeurs nous avons pu améliorer la qualité du jeu et développer une communauté de passionnés qui subsiste en dehors de l'école.
        
    \subsection{Remerciements}
        Nous remercions les membres de notre communauté de \(\beta-\)testeurs, qui nous ont beaucoup aidé dans la
        correction de nombreux bugs. Nous remercions pareillement \@Mace24 qui nous a conseillé tout au long du projet sur la gestion de notre temps et le déroulement des soutenances.
    
    
    \begin{center}
        Made with $\heartsuit$ by \CEX\,on \LaTeX.\\
        \textcopyright\,2021-2022, \btmlgs\\
        \SITE
    \end{center}

    
    
\section*{Références}
        \begin{itemize}
            \item \(\mathtt{blender.org}\)
            \item \(\mathtt{bootstrapstudio.io}\)
            \item \(\mathtt{discord.com}\)
            \item \(\mathtt{unity.com}\)
            \item \(\mathtt{photonengine.com/pun}\)
            \item \(\mathtt{overleaf.com}\)
            \item \(\mathtt{jetbrains.com/rider}\)
            \item \(\mathtt{assetstore.unity.com}\)
            \item \(\mathtt{image-line.com/fl-studio}\)
            \item \(\mathtt{youtube.com}\)
            \item \(\mathtt{github.com}\)
        \end{itemize}
=======
                Enfin, pour ce qui est de la partie la plus difficile du site Web, le \textit{responsive 
                design}, (qui est de rendre le site Web adaptable à toutes tailles d'écran), l'utilisation 
                du code HTML était indispensable. Nous avons donc cherché à comprendre comment le 
                \textit{responsive} fonctionnait à l'aide de tutoriels et implémenté chaque élément 
                pour les rendre indépendants entre eux et ainsi obtenir notre site Web \textit{responsive}.


    \section{Partie Technique}
    \lipsum[28-32]


    \section{Ressentis personnels}
    \lipsum[33-36]


    \addcontentsline{toc}{section}{Conclusion}
    \section*{Conclusion}
    \lipsum[36-37]

>>>>>>> 0737757309992d623e78eb6953edcad6021eead0

    \clearpage
    \addcontentsline{toc}{section}{Annexes}
    \section*{Annexes}
        \listoftables
        \listoffigures

        \begin{table}
            \caption{Chronologie du Projet}
            \centering
            \begin{minipage}[t]{.7\linewidth}
                \color{gray}
                \rule{\linewidth}{1pt}
                \ytl{Before}{To Complete}
                \ytl{10 mars}{Soutenance 1}
                \ytl{14 mars}{Mise en place des objectifs pour la soutenance 2}
                \ytl{20 mars}{Implémentation du drift et du sound design}
                \ytl{25 mars}{Création de deux nouveaux circuits}
                \ytl{30 mars}{Implémentation des menus \textsl{Timer} et \textsl{Solo}}
                \ytl{10 avril}{\AI\, version 1.0}
                \ytl{15 avril}{Implémentation du début et de la fin des courses}
                \ytl{17 avril}{Implémentation du garage}
                \ytl{18 avril}{Implémentation de la sauvegarde}
                \ytl{20 avril}{\AI\, version 2.0}
                \ytl{22 avril}{Finition du menu principal}
                \ytl{24 avril}{Implémentation de la musique}
                \ytl{25 avril}{Mode Timer fonctionnel}
<<<<<<< HEAD
                \ytl{27 avril}{Site Internet responsive}
=======
                \ytl{27 avril}{Site Web responsive}
>>>>>>> 0737757309992d623e78eb6953edcad6021eead0
                \ytl{28 avril}{Mode Solo fonctionnel}
                \ytl{29 avril}{Soutenance 2}
                \ytl{After}{To Complete}
                \rule{\linewidth}{1pt}
            \end{minipage}
        \end{table}


\end{document}
