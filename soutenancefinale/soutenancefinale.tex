%%%%%%%%%%%%%%%%%%%%%%%%%%%%%%%%%%%%%%%%%%%%%%%%%%%%%%%%%%%%%%%%%%%%%%%%%%%%%%%%%%%%%%%%%%%%%%%%%%%%%%%
% Soutenance Finale
% Bitume Legends Project
% CarrEniX
% Juin 2022
%%%%%%%%%%%%%%%%%%%%%%%%%%%%%%%%%%%%%%%%%%%%%%%%%%%%%%%%%%%%%%%%%%%%%%%%%%%%%%%%%%%%%%%%%%%%%%%%%%%%%%%

\documentclass[a4paper,12pt]{article}

\usepackage[utf8]{inputenc}
\usepackage[french]{babel}
\usepackage{geometry}
\usepackage{lipsum}
\usepackage{mathtools}
\usepackage[T1]{fontenc}
\usepackage{url}
\usepackage{longtable}
\usepackage[aboveskip=.5cm]{caption}
\usepackage{pdflscape}
\usepackage{xcolor}
\usepackage{graphicx,color}

\newcommand{\hsp}{\hspace{20pt}}
\newcommand{\HRule}{\rule{\linewidth}{0.5mm}}
\newcommand{\btmlgs}{\textsl{Bitume Legends}}
\newcommand{\AI}{Intelligence Artificielle}
\newcommand{\FL}{\textsl{FL Studio 20}}
\newcommand{\CEX}{\textsc{CarrEniX}}
\newcommand{\SITE}{\(\mathtt{btms.games}\)}
\renewcommand{\listfigurename}{Table des Annexes}
\renewcommand{\contentsname}{Table des Matières}
\newcommand\ytl[2]{\parbox[b]{10em}{\hfill{\color{cyan}\bfseries\sffamily 
    #1}~$\cdots\cdots$~}\makebox[0pt][c]{$\bullet$}\vrule\quad 
    \parbox[c]{5cm}{\vspace{7pt}\color{red!40!black!80}\raggedright\sffamily #2\\[7pt]}\\[-3pt]}


\usepackage{fancyhdr}
\pagestyle{fancy}
\lhead{Rapport de Soutenance Finale}
\rhead{{\CEX}\\{\btmlgs}}   

\begin{document}  

    \graphicspath{{../Medias}}

\begin{titlepage}
  \begin{sffamily}
  \begin{center}

  
    \includegraphics[scale=0.3]{epita.png}\\[1.5cm]

    \textsc{\huge EPITA Rennes}\\[0.5cm]

    % Titre
    \textsc{\Large Rapport de Soutenance Finale}\\[1.5cm]

    \HRule \\[0.4cm]
     { \LARGE \bfseries \btmlgs \\[0.4cm] }

    \HRule \\[2cm]
    \includegraphics[scale=0.5]{logo192.png}
     \\[0.5cm]
    \textsc{\Large \CEX}\\[1.5cm]

    % Membres
    \begin{minipage}{0.4\textwidth}
      \begin{flushleft} \large
        Anthony Caron\\
        Victorien Cambourian\\
      \end{flushleft}
    \end{minipage}
    \begin{minipage}{0.4\textwidth}
      \begin{flushright} \large
        Melvyn Delaroque\\
        Xavier de Place\\
      \end{flushright}
    \end{minipage}

    \vfill

    % Bottom of the page
    {\large 9 Juin 2022}

  \end{center}
  \end{sffamily}
\end{titlepage}


    \newpage

    \tableofcontents

    \newpage

    \addcontentsline{toc}{section}{Introduction}
    \section*{Introduction}
    \lipsum[1-2]


    \section{Reprise du Cahier des Charges}
    Le but ici c'est de réexpliquer le concept du jeu tel que présenté ds le cahier des charges. 
    On explique notre vision, ce que nous avons tenu à mettre ds le jeu et on parle pas trop des
    chose qu'on a dit mais pas fait.
    \clearpage
    \newpage


    \section{Chronologie temporelle du temps chronologique}
    \textbf\textsc{J'ai mis les points importants relatés ds les anciens rapports, le but va être de rédiger ça}
        \subsection{Janvier -> Mars}
            \subsubsection{Multijoueur}
                La base du multijoueur a été implémentée, non sans peines. 
                Nous pouvons nous connecter à 4 personnes (nous avons choisi ce nombre pour des questions 
                de simplicité) et commencer une partie. Il y a une interface de connection, avec différentes 
                options (détaillées plus tard dans ce document) puis le système de \textit{spawn} (apparition).
                Une musique de jeu a également été crée pour le multijoueur.
  
            \subsubsection{Programme Beta (\(\beta\))}
                Nous avons créé un questionnaire sur \textsl{Google Forms} afin que les premières personnes testant
                le jeu puisse nous signaler les bugs qu'ils ont rencontrés et nous les décrire le plus 
                précisément possible dans le but d'être corrigé rapidement. Pour cela, ce questionnaire 
                possède pour trois catégories. Tout d'abord, une question sur le mode de jeu où le bug 
                s'est produit. Puis une autre sur la nature du bug et enfin un paragraphe réservé
                pour nous décrire avec le plus de détails possible ce bug. Toutefois, si l'utilisateur
                trouve compliqué de décrire son bug, des liens pour nous contacter sur \textsl{Discord} 
                ou \textsl{GitHub} sont indiqués.
        
            \subsubsection{Communication}
                Au sujet de la communication, nous avons décidé de faire une publication sur \textsl{Instagram}
                \footnote{\(\mathtt{instagram.com/bitumelegends}\)}
                pour tenir informés nos potentiels premiers joueurs. Dans cette publication se trouve le lien 
                du site Internet où l'utilisateur pourra consulter se familiariser avec le menu de notre
                jeu, puisque le site reprend l'esthétique de ce dernier. Enfin l'onglet de la réalisation
                et celui de la présentation donnent l'occasion de nous connaître plus, de même pour le projet.
                Ils pourront enfin nous rejoindre sur \textsl{Discord} pour que nous puissions les inscrire dans le
                programme de tests.
        
            \subsubsection{Site Internet}
                Le site web a pour but de présenter les membres du groupe, leur rôle et leur avis sur le projet.
                Il présente aussi l'historique avec les idées de conception du jeu, du nom du jeu et celui du studio
                de développement. Ensuite, il nous permet d'expliquer comment nous avons réalisé le projet avec la
                chronologie de réalisation, c'est-à-dire les différentes étapes de réalisation du projet mais aussi
                les problèmes rencontrés et les solutions envisagées. Enfin, il contient aussi
                un espace dédié pour télécharger le projet et le jeu. Ce site est notre vitrine, il nous permet de faire
                la promotion de \btmlgs.
        
            \subsubsection{Menu}
                Le menu doit permettre d'accéder aux différents modes de jeux et fonctionnalités du jeu.
                Il doit servir de hub pour accéder au mode multijoueur, contre la montre et contre l'\AI.
                Il doit également permettre d'accéder au garage contenant les voitures du joueur. Un affichage
                du niveau du joueur, un bouton pour accéder à la remontée de bug et un menu d'options 
                sont également dans le menu du jeu. Au début du projet, le menu était basique. Il servait 
                simplement à pouvoir accéder aux différentes modes de jeux. Suite à cela, Melvyn et Victorien 
                ont commencé à travailler sur un design plus poussé que celui de base de \textsl{Unity}, 
                en adoptant notamment un thème de couleurs basé sur le logo du jeu. Tout d'abord, une
                ébauche du design a été réalisée sur Canva, dans le but de partager des idées sur le look
                final du menu. Ensuite nous avons implémenté le design dans le jeu, en insérant un fond
                et des espaces pour les futures fonctionnalités, en implémentant les boutons et en changeant
                leur design. Nous avons également implémenté le code couleur du design. Enfin, nous avons 
                ajouté des effets sonores et de couleurs aux boutons lorsqu'ils sont cliques ou lorsque
                la souris passe dessus. Une musique tourne également en fond dans le menu.


        \subsection{Mars -> Avril}
            Suite à la soutenance 1 du 10 mars, nous avons prit quelques jours de repos. Puis le 14, nous
            avons mit en place nos objectifs pour la soutenance 2 vis-à-vis du cahier des charges, de
            l'état du jeu ainsi que des commentaires de la première soutenance.\\
            Nous avons travaillé en parallèle sur différents objectifs, certains plus longs que d'autres.
            Notamment l'\AI, qui a prit énormément de temps et son travail d'implémentation ayant
            commencé au début de cette période. Le 20 mars ce sont les mécaniques de conduite comme le
            drift ainsi que le \textit{sound design} des véhicules qui ont été implémentés. Puis le 25 la
            création de deux nouveaux circuits, \textsl{City} et \textsl{Port}. Le mois de mars s'est
            conclu par la création des menus des modes de jeu \textsl{Timer} et \textsl{Solo}.\\
            Après 3 semaines, la première version de l'\AI était prête le 10 avril, ainsi que le
            comportement en fin et début de course le 15. Le 18, le garage et le choix de la voiture 
            sélectionnée, ainsi que la sélection de la difficulté et du circuit ont été mis en place avec
            succès. Le 20, l'IA a connu sa version 2.0 avec une précision améliorée et fonctionnelle sur
            les deux circuits.\\
            Durant la dernière semaine avant la soutenance, des implémentations mineures ou des
            corrections de bugs ont eu lieu. Le 22 avril le menu principal a été terminé et avec
            l'affichage de la voiture et le lien entre chaque sous-menus a été fait. Le 24 la musique a
            été correctement implémentée et le 25 le mode de jeu \textsl{Timer} a été corrigé pour enfin
            fonctionner correctement. Enfin, le 27 le site Web est enfin devenu \textit{responsive} et le
            28 le mode \textsl{Solo} est devenu fonctionnel.

            \subsubsection{Graphisme}
                Au lancement du projet, nous voulions créer nous même nos voitures et nos circuit à 
                l'aide du logiciel de modélisation 3D Blender. Toutefois, cela prends beaucoup de temps,
                temps qui n'est pas utilisé à coder le jeu, résoudre des bugs, etc. Nous avons donc fait
                un compromis entre notre volonté de créer nous même le design du jeu et la contrainte de
                temps. Nous nous sommes orientés alors sur l'ajout d'un \textit{asset} du \textsl{Unity
                Asset Store}, pour obtenir des voitures et des morceaux de pistes. Ceci nous a permis de
                créer directement dans \textsl{Unity} nos différents circuits et de gagner beaucoup de 
                temps. 
            
            \subsubsection{Sauvegarde}
                Pour que notre jeu soit plus facile à appréhender, nous avons créé un système de 
                sauvegarde. Il est composé de 2 principe : le \textsl{Save}, qui permet d'enregistrer 
                les données voulues et nécessaires au bon fonctionnement du jeu, ainsi que le 
                \textsl{Load}, qui permet de charger la sauvegarde lors du lancement du jeu. Pour le 
                moment, il s'occupe uniquement de sauvegarder quelle voiture a été choisie dans le 
                garage et de l'enregistrer pour la charger lors du prochain lancement du jeu. À terme, 
                nous comptons sauvegarder l'expérience du joueur, son pseudo, ainsi que la possible 
                customisation de ses voitures.

            \subsubsection{Menu}
                Pour la première soutenance, nous avons fait une charte graphique et un design pour les 
                menus.
                Nous nous étions concentrés sur le menu principal. Pour cette seconde soutenance, nous 
                avons travaillé les
                sous-menus des différents modes de jeu et les fonctionnalités internes.\\
                Nous avons implémenté le garage, permettant de
                sélectionner une voiture pour la course parmi la liste de voitures disponibles, qui
                est ensuite sauvegardée. La voiture sélectionnée s'affiche ensuite dans le menu principal
                à la place du logo du jeu. Les menus des modes de
                jeux \textsl{Timer} (course contre la montre) et \textsl{Solo} (contre une \AI) ont
                été implémentés de la même manière. Ils ont donc une structure similaire. Au sein de
                ces menus, nous retrouvons la sélection des circuits ainsi que la difficulté de la course et
                un moyen de revenir au menu précédent.\\
                Nous comptons ajouter un menu de réglages permettant de régler le volume du jeu, le volume
                des musiques, les touches permettant d'avancer ainsi que de changer de pseudo. Une 
                version simplifiée de ce menu, contenant la modification des touches de directions et le
                volume du jeu sera également disponible en course.

            \subsubsection{\AI}
                L'implémentation de l'\AI\, s'est bien déroulée. Nous avons pris du temps
                à nous décider sur quelle solution nous allons utiliser et sur comment
                l'\AI\, devrait se comporter une fois implémentée dans le jeu. Nous sommes partis sur une
                solution hybride entre celle intégrée dans \textsl{Unity} et une \textit{homemade}.
                Nous nous sommes basé sur le \textit{NavMesh} de \textsl{Unity}, puis nous avons
                travaillé à animer la voiture et à définir les points qu'elle devait franchir pour
                terminer le circuit. Ensuite est arrivée
                la (longue) partie de la calibration. Au début, notre IA se déplaçait aléatoirement,
                si bien qu'elle inventait le chemin à chaque fois sans prendre le circuit que nous
                avions dessiné. Après plusieurs jours de recherche, nous avons réussi à lui faire
                prendre uniquement le chemin prévu. Ensuite, il a fallu régler sa vitesse et sa
                précision, pour éviter qu'elle ne rentre dans chaque mur par souci de freinage en virage.\\
                Après ces quelques soucis, nous avons obtenu un mode \textsl{Solo} pratiquement
                fonctionnel. Nous avons rajouté à ceci les scripts qui nous permettent de gérer
                le départ et la fin de la course et nous étions bons.

            \subsubsection{Gameplay}
                En début de course, un décompte avant le départ est donné, bloquant la voiture pour
                empêcher les faux départs. Tout au long du circuit, le joueur doit traverser une série de 
                balises, les \textit{checkpoints}, pour valider la course. Cela permet d'empêcher le 
                joueur de simplement faire demi-tour et de traverser la ligne d'arrivée pour gagner, ou
                de couper le circuit. Ici le joueur est forcé de passer par chaque \textit{checkpoints},
                dans le bon sens afin de pouvoir terminer la course.\\
                Chaque mode de jeu (hors multijoueur) comprend une sélection de difficulté.
                La difficulté pour le mode de jeu \textsl{Timer} est déterminée par le temps maximum pour 
                terminer la course. En revanche, pour le mode de jeu \textsl{Solo}, 
                elle réside dans la vitesse et la précision de l'\AI, ce qui permet d'affronter des 
                \textit{bots} plus ou moins forts. \\
                En mode \textsl{Solo}, la course est gagnée si l'on est arrivé en premier ou que l'on a 
                survécu au \textit{bot} qui peut nous infliger des dégâts. En mode \textsl{Timer}, elle 
                est gagnée lorsque l'on passe la ligne d'arrivée (et toutes les balises précédentes) avant
                la fin du temps imparti. En cas de défaite, pour chacun de ces deux modes de jeu la partie
                est terminée et l'on peut soit recommencer la course, soit revenir au menu pour changer de
                véhicule, de circuit ou de difficulté.\\
                Nous avons également mis en place un système de collision qui, en plus d'impacter la forme de 
                la voiture, peut aussi impacter son comportement (un mauvais choc sur une des roues avant ou 
                le coté de la voiture peut provoquer des soucis de direction).

            \subsubsection{Musiques}
                Notre jeu comporte des musiques originales composées sur \FL\, par Melvyn. Un objectif d'un 
                peu moins de 5 musiques différentes a été posé dans le cahier des charges. Suite aux 
                commentaires de la première soutenance où il nous a été demandé de plus se concentrer sur le
                jeu, seules une musique de menu et une musique de course ont été composées. Toutefois, nous 
                tenions à ce qu'il y ai des nouveautés sur l'aspect de l'ambiance du jeu. Alors un travail 
                sur l'implémentation de ces musiques a été faite, notamment lors du passage d'une scène à
                une autre sans que la musique ne cesse ou recommence depuis le début. Pour la troisième 
                soutenance, l'objectif des musiques sera rempli avec un système de sélection des musiques 
                avec la possibilité de modifier le volume de la musique dans un sous-menus de réglages comme
                dit ci-dessus dans la section menu.
            
            \subsubsection{Site Web}
                Pour ce qui est du site Web, la mise en page de ce dernier était déjà bien avancée. 
                Cependant, quelques ajouts ont été fait, notamment pour permettre la finalisation de la 
                page ressources, qui contient tous les outils que nous avons utilisé pour réaliser notre 
                jeu. De plus, la page réalisation a été terminée avec l'ajout de la partie 
                problèmes et solutions.
                Enfin, pour ce qui est de la partie la plus difficile du site Web, le \textit{responsive 
                design}, (qui est de rendre le site Web adaptable à toutes tailles d'écran), l'utilisation 
                du code HTML était indispensable. Nous avons donc cherché à comprendre comment le 
                \textit{responsive} fonctionnait à l'aide de tutoriels et implémenté chaque élément 
                pour les rendre indépendants entre eux et ainsi obtenir notre site Web \textit{responsive}.


    \section{Partie Technique}
    \lipsum[28-32]


    \section{Ressentis personnels}
    \lipsum[33-36]


    \addcontentsline{toc}{section}{Conclusion}
    \section*{Conclusion}
    \lipsum[36-37]


    \clearpage
    \addcontentsline{toc}{section}{Annexes}
    \section*{Annexes}
        \listoftables
        \listoffigures

        \begin{table}
            \caption{Chronologie du Projet}
            \centering
            \begin{minipage}[t]{.7\linewidth}
                \color{gray}
                \rule{\linewidth}{1pt}
                \ytl{Before}{To Complete}
                \ytl{10 mars}{Soutenance 1}
                \ytl{14 mars}{Mise en place des objectifs pour la soutenance 2}
                \ytl{20 mars}{Implémentation du drift et du sound design}
                \ytl{25 mars}{Création de deux nouveaux circuits}
                \ytl{30 mars}{Implémentation des menus \textsl{Timer} et \textsl{Solo}}
                \ytl{10 avril}{\AI\, version 1.0}
                \ytl{15 avril}{Implémentation du début et de la fin des courses}
                \ytl{17 avril}{Implémentation du garage}
                \ytl{18 avril}{Implémentation de la sauvegarde}
                \ytl{20 avril}{\AI\, version 2.0}
                \ytl{22 avril}{Finition du menu principal}
                \ytl{24 avril}{Implémentation de la musique}
                \ytl{25 avril}{Mode Timer fonctionnel}
                \ytl{27 avril}{Site Web responsive}
                \ytl{28 avril}{Mode Solo fonctionnel}
                \ytl{29 avril}{Soutenance 2}
                \ytl{After}{To Complete}
                \rule{\linewidth}{1pt}
            \end{minipage}
        \end{table}


\end{document}
