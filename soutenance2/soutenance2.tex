%%%%%%%%%%%%%%%%%%%%%%%%%%%%%%%%%%%%%%%%%%%%%%%%%%%%%%%%%%%%%%%%%%%%%%%%%%%%%%%%%%%%%%%%%%%%%%%%%%%%%%%
% Soutenance 2
% Bitume Legends Project
% CarrEniX
% Avril 2022
%%%%%%%%%%%%%%%%%%%%%%%%%%%%%%%%%%%%%%%%%%%%%%%%%%%%%%%%%%%%%%%%%%%%%%%%%%%%%%%%%%%%%%%%%%%%%%%%%%%%%%%

% Packages
\documentclass[12pt,a4paper]{article}
\usepackage{mathtools}
\usepackage[utf8]{inputenc}
\usepackage{graphicx}
\usepackage[french]{babel}
\usepackage[T1]{fontenc}
\usepackage{url}
\usepackage{fancyhdr}
\usepackage{longtable}
\usepackage[aboveskip=.5cm]{caption}
\usepackage{pdflscape}
\usepackage{geometry}

% Global settings of document
\setlength\parindent{20pt}
\newcommand{\btmlgs}{\textsl{Bitume Legends}}
\newcommand{\AI}{Intelligence Artificielle}
\newcommand{\CEX}{\textsc{CarrEniX}}
\pagestyle{fancy}
\rhead{\btmlgs}
\lhead{Rapport de Soutenance Intermédiaire}
\rfoot{Page \thepage}
\lfoot{\CEX}
\fancyfoot[C]{}
\setlength{\headheight}{15pt}
\geometry{textheight=600pt}
\renewcommand{\listfigurename}{Table des Annexes}
\renewcommand{\contentsname}{Table des Matières}
\graphicspath{{../Medias}}

% Begin of the document
\begin{document}

    \begin{titlepage}
        \newcommand{\HRule}{\rule{\linewidth}{0.5mm}}
        \center
        \text{\LARGE Projet \btmlgs}\\[1cm]
        \includegraphics[scale=0.7]{logo192.png} \\[1cm]
        \HRule \\[0.4cm]
        { \huge \bfseries Rapport de Soutenance Intermédiaire\\[0.3cm] }
        \HRule \\[1.5cm]
        {\large \bfseries \CEX} \\[0.3cm]
        Anthony \textsc{Caron}\;--\;Melvyn \textsc{Delaroque}\\
        Victorien \textsc{Cambourian}\;--\;Xavier \textsc{de Place}
        \\ [4.5cm]
        EPITA INFOSUP 2026\\Année 2021 - 2022\\
        \(\mathtt{btms.games}\)
    \end{titlepage}

    \tableofcontents
    \clearpage

    \addcontentsline{toc}{section}{Introduction}
    \section*{Introduction}
        Nous sommes le studio \CEX\, qui développe \btmlgs, un jeu de course automobile 3D en
        \textit{low-poly}. Après avoir commencé le projet début janvier, et présenté notre avancée 
        mi-mars, nous publions notre second rapport. Il permet de faire le point sur ce que nous avons
        fait, les objectifs atteints, les priorités ainsi que les ressources utilisés durant ce projet.
        
        \textbf{\(\Downarrow\) PLAN À CORRIGER SI BESOIN \(\Downarrow\)}\\
        Dans ce rapport, nous allons aborder l'avancée de notre projet, l'organisation de notre groupe,
        les ressources et créations de chacun des membres du groupe ainsi que leurs valeurs 
        ajoutées, le ressenti des membres durant cette première période et nos objectifs.

    \section{Avancée par rapport aux attentes}
        \textbf{Nous reprenons point par point le précédent rapport et nous vérifions que chaque objectif
        posé dans celui-ci est bien accompli.}


        \subsection{Finir le menu}
            \textbf{\textsc{Todo Here @Victorien / @Melvyn}}\\
            Pour la première soutenance, nous avons fait une charte graphique et un design pour les menus.
            Nous nous étions concentrés sur le menu principal. Pour cette seconde soutenance, nous avons travaillé les
            sous-menus des différents modes de jeu et les fonctionnalités internes.\\
            % Dans le menus principal nous avons ajouté la barre d'expérience (voir \textbf{1.5 Gameplay}).
            Nous avons également implémenté le garage, permettant de
            sélectionner une voiture pour la course parmi une liste de voitures disponibles, qui
            est ensuite sauvegardée. La voiture sélectionnée s'affiche ensuite dans le menu principal
            à la place du logo du jeu. Les menus des modes de
            jeux \textsl{Timer} (course contre la montre) et \textsl{Solo} (contre une \AI) ont
            été implémentés de la même manière. Ils ont donc une structure similaire. Au sein de
            ces menus, nous retrouvons la sélection des circuits et de la difficulté de la course.\\
            Nous avons également mis en place un système de gestion des musiques faisant en sorte
            que la musique continue sans interruption tout en se baladant dans les menus et
            effectuant la transition entre eux.\\
            |\dots\\|\_\_\_\\


        \subsection{Finir le site Web}
            \textbf{\textsc{Todo Here @Anthony}}\\
            |\dots\\|\dots\\|\dots\\|\dots\\|\dots\\|\dots\\|\dots\\|\dots\\|\_\_\_\\


        \subsection{Continuer les musiques}
            \textbf{\textsc{Todo Here @Melvyn}}\\
            |\dots\\|\dots\\|\dots\\|\dots\\|\dots\\|\dots\\|\dots\\|\dots\\|\_\_\_\\


        \subsection{Implémentation de l'\AI}
            L'implémentation de l'\AI\, s'est bien déroulée. Nous avons pris du temps
            à nous décider sur quelle solution nous allons utiliser et sur comment
            l'\AI\, devrait se comporter une fois implémentée dans le jeu. Nous sommes partis sur une
            solution hybride entre celle intégrée dans \textsl{Unity} et une \textsl{Homemade}.
            Nous nous sommes basé sur le \textsl{NavMesh} de \textsl{Unity}, puis nous avons
            travailler à animer la voiture et à définir les points qu'elle devait franchir pour
            terminer le circuit.  Ensuite est arrivée
            la (longue) partie de la calibration. Au début, notre IA se déplaçait aléatoirement,
            si bien qu'elle inventait le chemin à chaque fois sans prendre le circuit que nous
            avions dessiné. Après plusieurs jours de recherche, nous avons réussi à la faire
            prendre uniquement le chemin prévu. Ensuite, il a fallu régler sa vitesse et sa
            précision, pour éviter qu'elle rentre dans chaque mur par souci de freinage en virage.\\
            Après ces quelques soucis, nous avons obtenu un mode \textsl{Solo} pratiquement
            fonctionnel. Nous avons rajouté à ceci les scripts qui nous permettent de gérer
            le départ et la fin de la course et nous étions bons.\\


        \subsection{Gameplay}
            \textbf{On parle ici de tt ce qui touche la course, donc le start, le win, les visuels et tt
            le reste avec l'expérience, on dit que conformément au CdC, on le fait en sprint 4.}\\
            En début de course, un décompte avant le départ est donné, bloquant la voiture pour 
            empêcher les faux départs. Tout au long du circuit, le joueur doit traverser une série de balises,
            les \textsl{checkpoints}, pour valider la course. Cela permet d'empêcher le joueur
            de simplement faire demi-tour et de traverser la ligne d'arrivée pour gagner, ou
            de couper le circuit. Ici le joueur est forcé de passer par chaque \textsl{checkpoints},
            dans le bon sens afin de pouvoir terminer la course.\\
            Chaque mode de jeu (hors multijoueur) comprend une sélection de difficulté.
            La difficulté pour le mode de jeu \textsl{Timer} est déterminée par le temps
            donné pour terminer la course. En revanche, pour le mode de jeu \textsl{Solo}, 
            elle réside dans la vitesse et la précision de l'\AI. \\
            Grâce aux résultats de la course, le joueur gagne de l'expérience
            en fonction de la difficulté de celle-ci. Cette expérience permet de débloquer
            de nouvelles voitures ou de les améliorer. Le niveau relatif à la quantité d'expérience
            du joueur est déterminé par une fonction mathématique, afin que l'on ne monte pas
            trop rapidement en niveau. Bien que répétitif, ce système permet au joueur de
            s'améliorer et de se familiariser avec les contrôles du jeu et des voitures,
            ce qui est idéal pour battre ses amis en course.


    \clearpage
    \section{Récit de la réalisation}
        \textbf{On raconte ici les grandes lignes de la réalisation, et on présente rapidement les parties
        qu'on va aborder.\\}
        |\dots\\|\dots\\|\dots\\|\dots\\|\dots\\|\dots\\|\dots\\|\dots\\|\dots\\
        |\dots\\|\dots\\|\dots\\|\dots\\|\dots\\|\dots\\|\dots\\|\dots\\|\_\_\_\\


        \subsection{Chronologie}
            \textbf{l'idée de ce paragraphe est de montrer comment nous avons réalisé notre projet en 
            fonction du temps. il faudrait en plus faire une sorte de chronologie imagée, a mettre en 
            annexe ou directement ici.}\\
            |\dots\\|\dots\\|\dots\\|\dots\\|\dots\\|\dots\\|\dots\\|\dots\\|\dots\\|\dots\\|\dots\\
            |\_\_\_ + schema

        \subsection{Point de vue de chacun}
            \subsubsection{Anthony}
                \textit{\bfseries TODO @ANTHONY}\\
                |\dots\\|\dots\\|\dots\\|\dots\\|\dots\\|\dots\\|\dots\\|\dots\\|\dots\\|\_\_\_\\

            \subsubsection{Melvyn}
                \textit{J'ai remarqué une légère baisse d'implication de ma part entre la 
                première et la deuxième soutenance, notamment une semaine où le 
                travail était plus que minime de ma part. J'ai également dû plus 
                travailler avec les autres, comparé à la première soutenance où l'on
                travaillait un peu plus dans notre "domaine d'expertise" à chacun 
                au lieu de nous entraider. Je me suis reprit quelques temps avant la
                soutenance et je suis fier de l'avancée du projet. Je pense que ce projet
                a beaucoup de potentiel. Il y a eu de grandes améliorations graphiques et
                techniques en ces quelques semaines et notre jeu ressemble enfin à un jeu.
                J'ai beaucoup d'espoirs pour la suite. Là où je m'étais trop concentré sur
                la musique lors de la première soutenance, je me suis plus tourné vers
                le graphisme, le \textsl{gameplay} et la physique du jeu, il était temps de vraiment
                faire du code...}

            \subsubsection{Victorien}
                \textit{Suite à la première soutenance, la première idée que j'ai eu a été
                de vouloir pousser le jeu, le développer et s'amuser dessus. Le but étant
                d'avoir un jeu plaisant, joli et agréable à jouer. J'ai donc passer
                de nombreuses heures (entre autres durant les vacances) à implémenter
                les différents modes de jeu, résoudre les \textsl{bug}. Je suis très satisfait
                de mon travail. C'est un vrai plaisir de coder le jeu et de voir notre
                avancée. Là où au départ \btmlgs n'avait pas forcément de style et
                n'attirait pas l'œil, il est maintenant beaucoup plus attractif suite à
                la refonte graphique du jeu ainsi que notre avancée. }

            \subsubsection{Xavier}
                \textit{Depuis la dernière soutenance, je me suis bien amusé à faire l'\AI.
                Cela était sympa au début puis plus le temps avançait, plus les problèmes arrivaient.
                Cela m'a fait passé par tous les états possibles, de la joie intense à la dépression
                profonde. Malgré cela, l'IA a sûrement été la partie que j'ai préféré faire.
                Pour le reste, je suis très fier de l'avancée que nous avons, nous sommes à jour sur notre
                planning et le jeu est très plaisant à jouer. Nous sommes très content de ce que rendent
                les graphiques et les voitures, ce qui était le point noir de la dernière soutenance.
                Bref, le jeu va vraiment être super sympa et cela nous rend heureux ! }
                


        \clearpage
        \subsection{Problèmes / Solutions}

            \subsubsection{Problème 1 + solution: Les voitures ne voulaient pas avancer}
                On parle ici du pb des voitures qui ne voulaient psa avancer, avec le nvmh.\\
                |\dots\\|\dots\\|\dots\\|\dots\\|\dots\\|\dots\\|\dots\\|\dots\\|\dots\\
                |\dots\\|\dots\\|\dots\\|\dots\\|\dots\\|\dots\\|\dots\\|\dots\\|\_\_\_\\

            \subsubsection{Problème 2 + solution: L'\AI\, s'arrêtait inopinément au milieu du parcours}
                On parle ici du pb de l'ia qui s'arrete au milieu du parcours.\\
                |\dots\\|\dots\\|\dots\\|\dots\\|\dots\\|\dots\\|\dots\\|\dots\\|\dots\\
                |\dots\\|\dots\\|\dots\\|\dots\\|\dots\\|\dots\\|\dots\\|\dots\\|\_\_\_\\

            \subsubsection{Problème 3 + solution: IDEE???}
                Il faut trouver un autre pb qu'on a eu sur le jeu\\
                |\dots\\|\dots\\|\dots\\|\dots\\|\dots\\|\dots\\|\dots\\|\dots\\|\dots\\
                |\dots\\|\dots\\|\dots\\|\dots\\|\dots\\|\dots\\|\dots\\|\dots\\|\_\_\_\\

    \clearpage
    \section{Objectifs pour la fin du projet}
        \subsection{Finir le garage}
            La sélection des voitures étant implémentée, il nous reste des fonctionnalités
            d'achats et de customisation pour les voitures du joueur.\\
            Pour l'instant, le garage est un simple défilement de voitures sur un fond uni.
            Nous voulons créer une scène 3D ressemblant à un véritable garage automobile.
            Au sein de ce garage, un menu défilant permettra de choisir sa voiture.
            En revanche nous comptons implémenter un système d'achat des voitures débloquées
            à l'aide de l'expérience et de l'argent du joueur, les voitures non débloquées
            seront grisées, celles achetables auront un cadenas et une option d'achat.
            Nous pourrons également voir les statistiques des voitures telles que leur vitesse,
            leur masse, leur accélération, etc. De plus à le joueur pourra également décider
            d'améliorer ces statistiques ainsi que l'aspect visuel de sa voiture avec l'argent
            du jeu. En effet, il est logique qu'un jeu de voiture comporte des options de
            \textsl{tuning}.
        

        \subsection{Faire le système de niveaux}
            \textbf{\textsc{Todo Here >\_}}\\
            |\dots\\|\dots\\|\dots\\|\dots\\|\dots\\|\dots\\|\dots\\|\dots\\|\dots\\|\dots\\|\_\_\_\\
            
        \subsection{Finir l'implémentation des voitures}
           Pour l'instant, les voitures ont toutes la même physique. C'est à dire le même
           centre de gravité, quasiment la même masse, puissance, inertie moteur, adhérence sur le sol.
           Étant donné que nous voulons utilisé des données réalistes pour chaque voiture,
           nous allons devoir nous occuper de gérer la physique pour les vingt-cinq voitures,
           car deux seulement sont prêtes actuellement.
            
        \subsection{Correction des divers bugs}
            \textbf{\textsc{Todo Here >\_}}\\
            |\dots\\|\dots\\|\dots\\|\dots\\|\dots\\|\dots\\|\dots\\|\dots\\|\dots\\|\dots\\|\_\_\_\\
            
        \subsection{Lancer les essais de \(\beta\)\textsl{-testing}}
            Suite à la première soutenance, le jeu n'était selon nous pas assez concluant
            pour lancer le système de \(\beta\)\textsl{-testing} au public. Nous avons
            décidé de garder le système de \(\beta\)\textsl{-testing} privé jusqu'à maintenant,
            puisqu'il n'y avait qu'un mode de jeu disponible. Grâce à notre réseau d'amis, 
            nous allons avoir des retours positifs et négatifs sur le jeu afin de débusquer
            chaque bug et pouvoir appliquer chaque amélioration demandée.


    \addcontentsline{toc}{section}{Conclusion}
    \section*{Conclusion}
        En conclusion, voici là où nous en sommes dans notre projet.
        Nous sommes contents de notre avancée, et nous savons comment
        nous allons continuer notre projet. 
        Nous serons ravi de vous revoir mi-juin pour vous présenter le jeu en entier !

    \section*{Références}
        \begin{itemize}
            \item \(\mathtt{blender.org}\)
            \item \(\mathtt{bootstrapstudio.io}\)
            \item \(\mathtt{discord.com}\)
            \item \(\mathtt{unity.com}\)
            \item \(\mathtt{photonengine.com/pun}\)
            \item \(\mathtt{overleaf.com}\)
            \item \(\mathtt{jetbrains.com/rider}\)
            \item \(\mathtt{assetstore.unity.com}\)
            \item \(\mathtt{image-line.com/fl-studio}\)
            \item \(\mathtt{youtube.com}\)  % \\[8cm]
        \end{itemize}

        \begin{center}
            Made with $\heartsuit$ by \CEX\, on \LaTeX.\\
            \textcopyright\, 2021-2022, \btmlgs\\
            \(\mathtt{btms.games}\)
        \end{center}

    \addcontentsline{toc}{section}{Annexes}
    \section*{Annexes}
    \listoffigures

\end{document}
