%%%%%%%%%%%%%%%%%%%%%%%%%%%%%%%%%%%%%%%%%%%%%%%%%%%%%%%%%%%%%%%%%%%%%%%%%%%%%%%%%%%%%%%%%%%%%%%%%%%%%%%
% Soutenance 2
% Bitume Legends Project
% CarrEniX
% Avril 2022
%%%%%%%%%%%%%%%%%%%%%%%%%%%%%%%%%%%%%%%%%%%%%%%%%%%%%%%%%%%%%%%%%%%%%%%%%%%%%%%%%%%%%%%%%%%%%%%%%%%%%%%

% Packages
\documentclass[12pt,a4paper]{article}
\usepackage{mathtools}
\usepackage[utf8]{inputenc}
\usepackage{graphicx}
\usepackage[french]{babel}
\usepackage[T1]{fontenc}
\usepackage{url}
\usepackage{fancyhdr}
\usepackage{longtable}
\usepackage[aboveskip=.5cm]{caption}
\usepackage{pdflscape}
\usepackage{geometry}

% Global settings of document
\setlength\parindent{20pt}
\newcommand{\btmlgs}{\textsl{Bitume Legends}}
\newcommand{\AI}{Intelligence Artificielle}
\newcommand{\FL}{\textsl{FL Studio 20}}
\newcommand{\CEX}{\textsc{CarrEniX}}
\newcommand{\SITE}{\(\mathtt{btms.games}\)}
\pagestyle{fancy}
\rhead{\btmlgs}
\lhead{Rapport de Soutenance Intermédiaire}
\rfoot{Page \thepage}
\lfoot{\CEX}
\fancyfoot[C]{}
\setlength{\headheight}{15pt}
\geometry{textheight=600pt}
\renewcommand{\listfigurename}{Table des Annexes}
\renewcommand{\contentsname}{Table des Matières}
\graphicspath{{../Medias}}

% Begin of the document
\begin{document}

    \begin{titlepage}
        \newcommand{\HRule}{\rule{\linewidth}{0.5mm}}
        \center
        \text{\LARGE Projet \btmlgs}\\[1cm]
        \includegraphics[scale=0.7]{logo192.png} \\[1cm]
        \HRule \\[0.4cm]
        { \huge \bfseries Rapport de Soutenance Intermédiaire\\[0.3cm] }
        \HRule \\[1.5cm]
        {\large \bfseries \CEX} \\[0.3cm]
        Anthony \textsc{Caron}\;--\;Melvyn \textsc{Delaroque}\\
        Victorien \textsc{Cambourian}\;--\;Xavier \textsc{de Place}
        \\ [4.5cm]
        EPITA INFOSUP 2026\\Année 2021 - 2022\\
        \SITE
    \end{titlepage}

    \tableofcontents
    \clearpage

    \addcontentsline{toc}{section}{Introduction}
    \section*{Introduction}
        Nous sommes le studio \CEX\, qui développe \btmlgs, un jeu de course automobile 3D en
        \textsl{low-poly}. Après avoir commencé le projet début janvier, et présenté notre avancée 
        mi-mars, nous publions notre second rapport. Il permet de faire le point sur ce que nous avons
        fait, les objectifs atteints, les priorités ainsi que les ressources utilisés durant ce 
        projet.\\
        Dans ce second rapport, nous allons aborder l'avancée de notre projet, les modifications que 
        nous avons du apporter suite à la première soutenance, les problèmes et difficultés auxquels 
        nous nous sommes confrontés, comment nous les avons les réglés, le ressenti des membres durant 
        cette deuxième période ainsi que nos objectifs pour la troisième et dernière soutenance.

    \section{Avancées}
        Pour récapituler les avancées du projet, nous avons repris point par point les objectifs de 
        notre rapport de première soutenance et nous avons comparé notre point d'avancement et nos prédictions.
        Et le résultat est concluant:
        
        \subsection{Site Web}
            Pour ce qui est du site Web, la mise en page de ce dernier était déjà bien avancée. 
            Cependant, quelques ajouts ont été fait, notamment pour permettre la finalisation de la 
            page ressources, qui contient tous les outils que nous avons utilisé pour réaliser notre 
            jeu. De plus, la page réalisation a été terminé avec l'ajout de la partie 
            problèmes et solutions.
            Enfin, pour ce qui est de la partie la plus difficile du site Web, le \textsl{responsive 
            design}, (qui est de rendre le site Web adaptable à toutes tailles d'écran), l'utilisation 
            du code HTML était indispensable. Nous avons donc cherché à comprendre comment le 
            \textsl{responsive} fonctionnait à l'aide de tutoriels et implémenté chaque élément 
            pour les rendre indépendants entre eux et ainsi obtenir notre site Web \textsl{responsive}.
        
        \subsection{Menu}
            Pour la première soutenance, nous avons fait une charte graphique et un design pour les 
            menus.
            Nous nous étions concentrés sur le menu principal. Pour cette seconde soutenance, nous 
            avons travaillé les
            sous-menus des différents modes de jeu et les fonctionnalités internes.\\
            Nous avons implémenté le garage, permettant de
            sélectionner une voiture pour la course parmi la liste de voitures disponibles, qui
            est ensuite sauvegardée. La voiture sélectionnée s'affiche ensuite dans le menu principal
            à la place du logo du jeu. Les menus des modes de
            jeux \textsl{Timer} (course contre la montre) et \textsl{Solo} (contre une \AI) ont
            été implémentés de la même manière. Ils ont donc une structure similaire. Au sein de
            ces menus, nous retrouvons la sélection des circuits ainsi que la difficulté de la course et
            un moyen de revenir au menu précédent.\\
            Nous comptons ajouté un menu de réglages permettant de régler le volume du jeu, le volume
            des musiques, les touches permettant d'avancer ainsi que de changer de pseudo. Une 
            version simplifiée de ce menu, contenant la modification des touches de directions et le
            volume du jeu sera également disponible en course.


        \subsection{Musiques}
            Notre jeu comporte des musiques originales composées sur \FL\, par Melvyn. Un objectif d'un 
            peu moins de 5 musiques différentes a été posé dans le cahier des charges. Suite aux 
            commentaires de la première soutenance où il nous a été demandé de plus se concentrer sur le
            jeu, seules une musique de menu et une musique de course ont été composées. Toutefois, nous 
            tenions à ce qu'il y ai des nouveautés sur l'aspect de l'ambiance du jeu. Alors un travail 
            sur l'implémentation de ces musiques a été faite, notamment lors du passage d'une scène à
            une autre sans que la musique ne cesse ou recommence depuis le début. Pour la troisième 
            soutenance, l'objectif des musiques sera rempli avec un système de sélection des musiques 
            avec la possibilité de modifier le volume de la musique dans un sous-menus de réglages comme
            dit ci-dessus dans la section menu.

        \subsection{\AI}
            L'implémentation de l'\AI\, s'est bien déroulée. Nous avons pris du temps
            à nous décider sur quelle solution nous allons utiliser et sur comment
            l'\AI\, devrait se comporter une fois implémentée dans le jeu. Nous sommes partis sur une
            solution hybride entre celle intégrée dans \textsl{Unity} et une \textsl{Homemade}.
            Nous nous sommes basé sur le \textsl{NavMesh} de \textsl{Unity}, puis nous avons
            travailler à animer la voiture et à définir les points qu'elle devait franchir pour
            terminer le circuit.  Ensuite est arrivée
            la (longue) partie de la calibration. Au début, notre IA se déplaçait aléatoirement,
            si bien qu'elle inventait le chemin à chaque fois sans prendre le circuit que nous
            avions dessiné. Après plusieurs jours de recherche, nous avons réussi à la faire
            prendre uniquement le chemin prévu. Ensuite, il a fallu régler sa vitesse et sa
            précision, pour éviter qu'elle rentre dans chaque mur par souci de freinage en virage.\\
            Après ces quelques soucis, nous avons obtenu un mode \textsl{Solo} pratiquement
            fonctionnel. Nous avons rajouté à ceci les scripts qui nous permettent de gérer
            le départ et la fin de la course et nous étions bons.


        \subsection{Gameplay}
            En début de course, un décompte avant le départ est donné, bloquant la voiture pour 
            empêcher les faux départs. Tout au long du circuit, le joueur doit traverser une série de 
            balises, les \textsl{check-points}, pour valider la course. Cela permet d'empêcher le 
            joueur de simplement faire demi-tour et de traverser la ligne d'arrivée pour gagner, ou
            de couper le circuit. Ici le joueur est forcé de passer par chaque \textsl{check-points},
            dans le bon sens afin de pouvoir terminer la course.\\
            Chaque mode de jeu (hors multijoueur) comprend une sélection de difficulté.
            La difficulté pour le mode de jeu \textsl{Timer} est déterminée par le temps maximum pour 
            terminer la course. En revanche, pour le mode de jeu \textsl{Solo}, 
            elle réside dans la vitesse et la précision de l'\AI, ce qui permet d'affronter des 
            \textsl{bots} plus
            ou moins forts. \\
            En mode \textsl{Solo}, la course est gagnée si l'on est arrivé en premier ou que l'on a 
            survécu au \textsl{Bot} qui peut nous infliger des dégâts. En mode \textsl{Timer}, elle 
            est gagnée lorsque l'on passe la ligne d'arrivée (et toutes les balises précédentes) avant
            la fin du temps imparti. En cas de défaite, pour chacun de ces deux modes de jeu la partie
            est terminée et l'on peut soit recommencer la course, soit revenir au menu pour changer de
            véhicule, de circuit ou de difficulté.\\


    \clearpage
    \section{Récit de la réalisation}
        \textbf{On raconte ici les grandes lignes de la réalisation, et on présente rapidement les 
        parties qu'on va aborder.\\}


        \subsection{Point de vue de chacun}
            \subsubsection{Anthony}
                \textit{Au sujet du site Web, j'ai eu beaucoup de mal à implémenter le responsive design
                puisque l'application Bootstrap Studio 5 ne suffisait pas, et l'usage du code HTML était
                obligatoire. Ayant peu de connaissance dans ce dernier langage, j'ai donc demandé
                de l'aide à mon groupe pour m'aider à le mettre en place. Ainsi après avoir compris le
                code généré par Bootstrap pour transformer chaque élément en responsive, d'autre
                problèmes de mise en forme ont complexifié la tâche entre les éléments responsive et
                non. De plus, la cohésion d'équipe a été fortement accentué, ce qui a permis de régler 
                les problèmes beaucoup plus rapidement. C'est pourquoi une grande avancée dans le 
                jeu vidéo s'est fait ressentir par rapport à la soutenance 1. Du fait de cette avancée, 
                ma motivation et l'envie de faire de Bitume Legends un bon jeu de voiture sera plus 
                forte pour la soutenance 3.}

            \subsubsection{Melvyn}
                \textit{J'ai remarqué une légère baisse d'implication de ma part entre la 
                première et la deuxième soutenance, notamment une semaine où le 
                travail était plus que minime de ma part. J'ai également dû plus 
                travailler avec les autres, comparé à la première soutenance où l'on
                travaillait un peu plus dans notre "domaine d'expertise" à chacun 
                au lieu de nous entraider. Je me suis reprit quelques temps avant la
                soutenance et je suis fier de l'avancée du projet. Je pense que ce projet
                a beaucoup de potentiel. Il y a eu de grandes améliorations graphiques et
                techniques en ces quelques semaines et notre jeu ressemble enfin à un jeu.
                J'ai beaucoup d'espoirs pour la suite. Là où je m'étais trop concentré sur
                la musique lors de la première soutenance, je me suis plus tourné vers
                le graphisme, le \textsl{gameplay} et la physique du jeu, il était temps de vraiment
                faire du code...}

            \subsubsection{Victorien}
                \textit{Suite à la première soutenance, la première idée que j'ai eu a été
                de vouloir développer le jeu et s'amuser dessus. Le but étant
                d'avoir un jeu plaisant, joli et agréable à jouer. J'ai donc passer
                de nombreuses heures à implémenter les différents modes de jeu, résoudre les 
                \textsl{bugs}. 
                Je suis très satisfait de mon travail. De plus, ceci m'a permit de m'améliorer en 
                \textsl{C\#} ainsi qu'en programmation orientée objet. C'est un vrai plaisir de coder 
                le jeu et de voir notre travail
                porter ses fruits. Là où au départ \btmlgs\, n'avait pas forcément de style et
                n'attirait pas l'œil, il est maintenant beaucoup plus attractif suite à
                la refonte graphique du jeu ainsi que notre avancée.}

            \subsubsection{Xavier}
                \textit{Depuis la dernière soutenance, je me suis bien amusé à faire l'\AI.
                Cela était sympa au début puis plus le temps avançait, plus les problèmes arrivaient.
                Cela m'a fait passé par tous les états possibles, de la joie intense à la dépression
                profonde. Malgré cela, l'\AI a sûrement été la partie que j'ai préféré faire.
                Pour le reste, je suis très fier de l'avancée que nous avons, nous sommes à jour sur 
                notre planning et le jeu est très plaisant à jouer. Nous sommes très content de ce que 
                rendent les graphiques et les voitures, ce qui était le point noir de la dernière 
                soutenance. Bref, le jeu va vraiment être super sympa et cela nous rend heureux ! }


        \clearpage
        \subsection{Problèmes et Solutions}
            \subsubsection{Implémentation des voitures}
                L'implémentation de la physique de voitures fut complexe, en particulier au vu des 
                nombreuses variables impactant le comportement d'une voiture. Il a fallut gérer le 
                poids, la vitesse, la puissance en chevaux du moteur, le couple maximum, l'angle et la 
                vitesse de braquage pour la direction, la force et la vitesse de freinage ou encore 
                l'inertie du moteur et la hauteur du centre de gravité.\\
                Suite à l'implémentation de la physique des voitures, nous nous sommes rendu compte que 
                les modèles des voitures que nous avions ne permettaient pas son bon fonctionnement. Il 
                a fallu dans un premier temps modifier les \textsl{prefabs} des voitures, entre autres 
                leurs \textsl{rigidify} qui empêchaient les roues de tourner et de considérer qu'elles 
                touchaient le sol. Suite à cela, il a fallu également modifier les roues des voitures en
                unifiant la jante ainsi que le pneu, chose qui n'était pas faite avant pour pouvoir 
                entraîner l'essieu et synchroniser la direction. Après avoir trouvé ce problème sur une 
                voiture, il a fallu l'appliquer aux autres, ce qui explique également pourquoi les 
                voitures ont pour l'instant la même physique. Toutefois, les voitures ainsi que les 
                collisions sont fonctionnelles.\\
                Un autre problème fût aussi celui du sound-design de la voiture. Plusieurs éléments dans
                une voiture produisent du son et n'ont pas le même comportement en fonction du poids et
                la vitesse du véhicule ou encore de la puissance du moteur. Il a fallut gérer le passage
                des rapports, le bruit du moteur dont le \textsl{pitch} et le volume changeait selon la
                vitesse ou le type du véhicule (un pick-up a un bruit différent d'une Supercar). Avec le
                passage de rapport il y avait également le bruit du turbocompresseur ou du
                supercompresseur à gérer.\\
                Au final nous avons trouvé certains paramètres permettant à la voiture d'avoir un bon
                comportement et d'avoir un bon son. Il ne nous reste plus qu'à appliquer ces paramètres
                aux autres voitures.

            \subsubsection{\AI}
                On parle ici du pb de l'ia qui s'arrête au milieu du parcours.\\
                |\dots\\|\dots\\|\dots\\|\dots\\|\dots\\|\dots\\|\dots\\|\dots\\|\dots\\
                |\dots\\|\dots\\|\dots\\|\dots\\|\dots\\|\dots\\|\dots\\|\dots\\|\_\_\_\\

            \subsubsection{Implémentation des musiques}
                Pour les menus nous avons préféré créer plusieurs scènes pour les sous-menus accessibles
                par le menu principal. Plutôt qu'une seule scène se transformant à l'appui d'un bouton.
                Bien que pratique pour les scripts individuels à chaque modes de jeux et sous-menus,
                cela compliquait l'implémentation de la musique. Il a fallu trouver un moyen d'empêcher
                les musiques de s'arrêter à chaque entrée dans un menu ou sous-menus. Pour cela il
                fallait faire en sorte que la musique fasse partie des éléments
                \textsl{DontDestroyOnLoad}.\\
                Un problème qui a suivi est le fait que bien que la musique ne s'arrête plus, à chaque
                ouverture d'un menu une autre instance de la musique se lançait. Pour remédier à cela il
                a fallut préciser dans le script que toute nouvelles musique dans ne scène de menu ne
                devait pas se lancer si une autre se jouait déjà.\\
                Enfin un dernier problème a eu lieu lors de l'implémentation des musiques de courses. La
                musique de menu continuait de jouer par dessus celle de course. Il a alors fallut
                ajouter dans le script le cas où lorsque l'on était en course, il fallait mettre un
                \textsl{SetActive} sur la musique en train de jouer et la désactiver si l'on entrait en
                course, la musique précédente étant celle des menus.

    \clearpage
    \section{Objectifs pour la fin du projet}
        \subsection{Finir le garage}
            La sélection des voitures étant implémentée, il nous reste des fonctionnalités
            d'achats et de customisation pour les voitures du joueur.\\
            Pour l'instant, le garage est un simple défilement de voitures sur un fond uni.
            Nous voulons créer une scène 3D ressemblant à un véritable garage automobile.
            Au sein de ce garage, un menu défilant permettra de choisir sa voiture.
            En revanche nous comptons implémenter un système d'achat des voitures débloquées
            à l'aide de l'expérience et de l'argent du joueur, les voitures non débloquées
            seront grisées, celles achetables auront un cadenas et une option d'achat.
            Nous pourrons également voir les statistiques des voitures telles que leur vitesse,
            leur masse, leur accélération, etc. De plus à le joueur pourra également décider
            d'améliorer les statistiques de ses voitures ainsi que leur aspect visuel avec l'argent
            du jeu. En effet, il est logique qu'un jeu de voiture comporte des options de
            \textsl{tuning}.
        
        \subsection{Faire le système de niveaux}
            Grâce aux résultats de la course, le joueur gagne de l'expérience en fonction de la
            difficulté de celle-ci. Cette expérience permet de débloquer de nouvelles voitures
            ou de les améliorer. La quantité d'expérience nécessaire du joueur pour passer un palier
            sera déterminé par une fonction exponentielle, afin que l'on ne monte pas trop rapidement 
            en niveau. Bien que répétitif, ce système permet au joueur 
            de s'améliorer et de se familiariser avec les contrôles du 
            jeu et des voitures, ce qui selon nous est idéal pour battre ses amis en course.
            
        \subsection{Finir l'implémentation des voitures}
           Pour l'instant, les voitures ont toutes la même physique. C'est à dire le même
           centre de gravité, quasiment la même masse, puissance, inertie moteur, adhérence sur le 
           sol.
           Étant donné que nous voulons utilisé des données réalistes pour chaque voiture,
           nous allons devoir nous occuper de gérer la physique pour les autres voitures,
           car deux seulement sont prêtes actuellement.
            
        \subsection{Lancer les essais de \(\beta\)\textsl{-testing}}
            Suite à la première soutenance, le jeu n'était selon nous pas assez concluant
            pour lancer le système de \(\beta\)\textsl{-testing} au public. Nous avons donc
            décidé de garder ce système privé puisqu'il n'y avait qu'un mode de jeu terminé de
            disponible, les autres modes de jeu n'ayant qu'une base permettant de les faire fonctionner.
            Bien que nous avons inviter chacun un proche à rejoindre ce système, nous comptons lancer le
            \(\beta\)\textsl{-testing} au grand public début mai grâce à notre réseau d'amis ainsi que
            notre compte Instagram. Les avis seront récoltés sur notre site Web, directement dans le jeu
            depuis le menu mais encore sur notre serveur Discord, accessible depuis notre site internet
            et notre compte Instagram. Avec ces retours positifs et négatifs sur le jeu nous comptons
            débusquer chaque bug et pouvoir appliquer chaque amélioration demandée.
        
        \subsection{Correction des divers bug}
            Comme tout programme informatique, notre jeu comporte divers problèmes, que nous n'avons pas
            tous identifié. Pour les trouver, nous avons nos tests internes (tester les parties du jeu
            au fur et à mesure de l'implémentation) mais aussi les tests externes, à savoir le
            \(\beta\)\textsl{-testing} précédemment expliqué.
            Cet objectif sera de corriger le maximum de bugs le mieux possible sans pour autant ne plus
            faire fonctionner le reste du jeu et d'implémenter ceux qui restent comme de véritables
            caractéristiques du jeu. (\textit{ce n'est pas un bug, c'est une feature.})


    \addcontentsline{toc}{section}{Conclusion}
    \section*{Conclusion}
        Une grande avancée en matière de graphisme et surtout de \textsl{Gameplay} a été faite depuis la
        première soutenance. Nous comptons continuer dans cette voie pour la troisième en ajoutant des 
        détails afin que le jeu puisse gagner son identité en matière de \textsl{Gameplay}, de graphisme, 
        de musique et de sa communauté.\\
        Voici là où nous en sommes dans notre projet.
        Nous sommes contents de notre avancée, et nous savons comment
        nous allons continuer notre projet. 
        Nous serons ravi de vous revoir mi-juin pour vous présenter le jeu complet !

    \section*{Références}
        \begin{itemize}
            \item \(\mathtt{blender.org}\)
            \item \(\mathtt{bootstrapstudio.io}\)
            \item \(\mathtt{discord.com}\)
            \item \(\mathtt{unity.com}\)
            \item \(\mathtt{photonengine.com/pun}\)
            \item \(\mathtt{overleaf.com}\)
            \item \(\mathtt{jetbrains.com/rider}\)
            \item \(\mathtt{assetstore.unity.com}\)
            \item \(\mathtt{image-line.com/fl-studio}\)
            \item \(\mathtt{youtube.com}\)
            \item \(\mathtt{github.com}\) % \\[8cm]
        \end{itemize}

        \begin{center}
            Made with $\heartsuit$ by \CEX\, on \LaTeX.\\
            \textcopyright\, 2021-2022, \btmlgs\\
            \SITE
        \end{center}

    \addcontentsline{toc}{section}{Annexes}
    \section*{Annexes}
    \listoffigures

\end{document}
