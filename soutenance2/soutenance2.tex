%%%%%%%%%%%%%%%%%%%%%%%%%%%%%%%%%%%%%%%%%%%%%%%%%%%%%%%%%%%%%%%%%%%%%%%%%%%%%%%%%%%%%%%%%%%%%%%%%%%%%%%
% Soutenance 2
% Bitume Legends Project
% CarrEniX
% Avril 2022
%%%%%%%%%%%%%%%%%%%%%%%%%%%%%%%%%%%%%%%%%%%%%%%%%%%%%%%%%%%%%%%%%%%%%%%%%%%%%%%%%%%%%%%%%%%%%%%%%%%%%%%

% Packages
\documentclass[12pt,a4paper]{article}
\usepackage{mathtools}
\usepackage[utf8]{inputenc}
\usepackage{graphicx}
\usepackage[french]{babel}
\usepackage[T1]{fontenc}
\usepackage{url}
\usepackage{fancyhdr}
\usepackage{longtable}
\usepackage[aboveskip=.5cm]{caption}
\usepackage{pdflscape}
\usepackage{geometry}

% Global settings of document
\setlength\parindent{20pt}
\newcommand{\btmlgs}{\textsl{Bitume Legends}}
\newcommand{\AI}{Intelligence Artificielle}
\newcommand{\CEX}{\textsc{CarrEniX}}
\pagestyle{fancy}
\rhead{\btmlgs}
\lhead{Rapport de Soutenance Intermédiaire}
\rfoot{Page \thepage}
\lfoot{\CEX}
\fancyfoot[C]{}
\setlength{\headheight}{15pt}
\geometry{textheight=600pt}
\renewcommand{\listfigurename}{Table des Annexes}
\renewcommand{\contentsname}{Table des Matières}
\graphicspath{{../Medias}}

% Begin of the document
\begin{document}

    \begin{titlepage}
        \newcommand{\HRule}{\rule{\linewidth}{0.5mm}}
        \center
        \text{\LARGE Projet \btmlgs}\\[1cm]
        \includegraphics[scale=0.7]{logo192.png} \\[1cm]
        \HRule \\[0.4cm]
        { \huge \bfseries Rapport de Soutenance Intermédiaire\\[0.3cm] }
        \HRule \\[1.5cm]
        {\large \bfseries \CEX} \\[0.3cm]
        Anthony \textsc{Caron}\;--\;Melvyn \textsc{Delaroque}\\
        Victorien \textsc{Cambourian}\;--\;Xavier \textsc{de Place}
        \\ [4.5cm]
        EPITA INFOSUP 2026\\Année 2021 - 2022\\
        \(\mathtt{btms.games}\)
    \end{titlepage}

    \tableofcontents
    \clearpage

    \addcontentsline{toc}{section}{Introduction}
    \section*{Introduction}
        Nous sommes le studio \CEX\, qui développe \btmlgs, un jeu de course automobile 3D en
        \textit{low-poly}. Après avoir commencé le projet début janvier, et présenté notre avancée 
        mi-mars, nous publions notre second rapport. Il permet de faire le point sur ce que nous avons
        fait, les objectifs atteints, les priorités ainsi que les ressources utilisés durant ce projet.
        
        \textbf{\(\Downarrow\) PLAN À CORRIGER SI BESOIN \(\Downarrow\)}\\
        Dans ce rapport, nous allons aborder l'avancée de notre projet, l'organisation de notre groupe,
        les ressources et créations de chacun des membres du groupe ainsi que leurs valeurs 
        ajoutées, le ressenti des membres durant cette première période et nos objectifs.

    \section{Avancée par rapport aux attentes}
        Nous reprenons point par point le précédent rapport et nous vérifions que chaque objectif
        posé dans celui-ci est bien accompli.
        \subsection{Finir le menu}
            \textbf{\textsc{Todo Here @Victorien / @Melvyn}}\\
            |\dots\\|\dots\\|\dots\\|\dots\\|\dots\\|\dots\\|\dots\\|\dots\\|\_\_\_\\
        \subsection{Finir le site Web}
            \textbf{\textsc{Todo Here @Anthony}}\\
            |\dots\\|\dots\\|\dots\\|\dots\\|\dots\\|\dots\\|\dots\\|\dots\\|\_\_\_\\
        \subsection{Continuer les musiques}
            \textbf{\textsc{Todo Here @Melvyn}}\\
            |\dots\\|\dots\\|\dots\\|\dots\\|\dots\\|\dots\\|\dots\\|\dots\\|\_\_\_\\
        \subsection{Implémentation de l'\AI}
            \textbf{\textsc{Todo Here @Xavier}}\\
            |\dots\\|\dots\\|\dots\\|\dots\\|\dots\\|\dots\\|\dots\\|\dots\\|\_\_\_\\
        \subsection{Gameplay}
            On parle ici de tt ce qui touche la course, donc le start, le win, les visuels et tt
            le reste avec l'expérience, on dit que conformément au CdC, on le fait en sprint 4.


    \clearpage
    \section{Récit de la réalisation}
        On raconte ici les grandes lignes de la réalisation, et on présente rapidement les parties
        qu'on va aborder.\\
        |\dots\\|\dots\\|\dots\\|\dots\\|\dots\\|\dots\\|\dots\\|\dots\\|\dots\\
        |\dots\\|\dots\\|\dots\\|\dots\\|\dots\\|\dots\\|\dots\\|\dots\\|\_\_\_\\


        \subsection{Chronologie}
            \textbf{l'idée de ce paragraphe est de montrer comment nous avons réalisé notre projet en 
            fonction du temps. il faudrait en plus faire une sorte de chronologie imagée, a mettre en 
            annexe ou directement ici.}\\
            |\dots\\|\dots\\|\dots\\|\dots\\|\dots\\|\dots\\|\dots\\|\dots\\|\dots\\|\dots\\|\dots\\
            |\_\_\_ + schema

        \subsection{Récit de chacun}
            \subsubsection{Anthony}
                \textit{\bfseries TODO @ANTHONY}\\
                |\dots\\|\dots\\|\dots\\|\dots\\|\dots\\|\dots\\|\dots\\|\dots\\|\dots\\|\dots\\|\_\_\_\\

            \subsubsection{Melvyn}
                \textit{J'ai remarqué une légère baisse d'implication de ma part entre la 
                première et la deuxième soutenance, notamment une semaine où le 
                travail était plus que minime de ma part. J'ai également dû plus 
                travailler avec les autres, comparé à la première soutenance où l'on
                travaillait un peu plus dans notre "domaine d'expertise" à chacun 
                au lieu de nous entraider. Je me suis reprit quelques temps avant la
                soutenance et je suis fier de l'avancée du projet. Je pense que ce projet
                a beaucoup de potentiel. Il y a eu de grandes améliorations graphiques et
                techniques en ces quelques semaines et notre jeu ressemble enfin à un jeu.
                J'ai beaucoup d'espoirs pour la suite. Là où je m'étais trop concentré sur
                la musique lors de la première soutenance, je me suis plus tourné vers
                le graphisme, le gameplay et la physique du jeu, il était temps de vraiment
                faire du code...}

            \subsubsection{Victorien}
                \textit{\bfseries TODO @VICTORIEN}\\
                |\dots\\|\dots\\|\dots\\|\dots\\|\dots\\|\dots\\|\dots\\|\dots\\|\dots\\|\dots\\|\_\_\_\\

            \subsubsection{Xavier}
                \textit{\bfseries TODO @XAVIER}\\
                |\dots\\|\dots\\|\dots\\|\dots\\|\dots\\|\dots\\|\dots\\|\dots\\|\dots\\|\dots\\|\_\_\_\\


        \clearpage
        \subsection{Problèmes / Solutions}

            \subsubsection{Problème 1 + solution: Les voitures ne voulaient pas avancer}
                On parle ici du pb des voitures qui ne voulaient psa avancer, avec le nvmh.\\
                |\dots\\|\dots\\|\dots\\|\dots\\|\dots\\|\dots\\|\dots\\|\dots\\|\dots\\
                |\dots\\|\dots\\|\dots\\|\dots\\|\dots\\|\dots\\|\dots\\|\dots\\|\_\_\_\\

            \subsubsection{Problème 2 + solution: L'\AI\, s'arrêtait inopinément au milieu du parcours}
                On parle ici du pb de l'ia qui s'arrete au milieu du parcours.\\
                |\dots\\|\dots\\|\dots\\|\dots\\|\dots\\|\dots\\|\dots\\|\dots\\|\dots\\
                |\dots\\|\dots\\|\dots\\|\dots\\|\dots\\|\dots\\|\dots\\|\dots\\|\_\_\_\\

            \subsubsection{Problème 3 + solution: IDEE???}
                Il faut trouver un autre pb qu'on a eu sur le jeu\\
                |\dots\\|\dots\\|\dots\\|\dots\\|\dots\\|\dots\\|\dots\\|\dots\\|\dots\\
                |\dots\\|\dots\\|\dots\\|\dots\\|\dots\\|\dots\\|\dots\\|\dots\\|\_\_\_\\

    \clearpage
    \section{Objectifs pour la fin du projet}
        \subsection{Finir le garage}
            \textbf{\textsc{Todo Here >\_}}\\
            |\dots\\|\dots\\|\dots\\|\dots\\|\dots\\|\dots\\|\dots\\|\dots\\|\dots\\|\dots\\|\_\_\_\\
        \subsection{Faire le système de niveaux}
            \textbf{\textsc{Todo Here >\_}}\\
            |\dots\\|\dots\\|\dots\\|\dots\\|\dots\\|\dots\\|\dots\\|\dots\\|\dots\\|\dots\\|\_\_\_\\
        \subsection{Finir l'implémentation des voitures}
            \textbf{\textsc{Todo Here >\_}}\\
            |\dots\\|\dots\\|\dots\\|\dots\\|\dots\\|\dots\\|\dots\\|\dots\\|\dots\\|\dots\\|\_\_\_\\
        \subsection{Correction des divers bugs}
            \textbf{\textsc{Todo Here >\_}}\\
            |\dots\\|\dots\\|\dots\\|\dots\\|\dots\\|\dots\\|\dots\\|\dots\\|\dots\\|\dots\\|\_\_\_\\
        \subsection{Lancer les essais de \(\beta\)\textsl{-testing}}
            \textbf{\textsc{Todo Here >\_}}\\
            |\dots\\|\dots\\|\dots\\|\dots\\|\dots\\|\dots\\|\dots\\|\dots\\|\dots\\|\dots\\|\_\_\_\\


    \addcontentsline{toc}{section}{Conclusion}
    \section*{Conclusion}
        En conclusion, voici là où nous en sommes dans notre projet.
        Nous sommes contents de notre avancée, et nous savons comment
        nous allons continuer notre projet. 
        Nous serons ravi de vous revoir mi-juin pour vous présenter le jeu en entier !

    \section*{Références}
        \begin{itemize}
            \item \(\mathtt{blender.org}\)
            \item \(\mathtt{bootstrapstudio.io}\)
            \item \(\mathtt{discord.com}\)
            \item \(\mathtt{unity.com}\)
            \item \(\mathtt{photonengine.com/pun}\)
            \item \(\mathtt{overleaf.com}\)
            \item \(\mathtt{jetbrains.com/rider}\)
            \item \(\mathtt{assetstore.unity.com}\)
            \item \(\mathtt{image-line.com/fl-studio}\)
            \item \(\mathtt{youtube.com}\)  % \\[8cm]
        \end{itemize}

        \begin{center}
            Made with $\heartsuit$ by \CEX\, on \LaTeX.\\
            \textcopyright\, 2021-2022, \btmlgs\\
            \(\mathtt{btms.games}\)
        \end{center}

    \addcontentsline{toc}{section}{Annexes}
    \section*{Annexes}
    \listoffigures

\end{document}
