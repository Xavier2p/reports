%%%%%%%%%%%%%%%%%%%%%%%%%%%%%%%%%%%%%%%%%%%%%%%%%%%%%%%%%%%%%%%%%%%%%%%%%%%%%%%%
% Soutenance 1
% Bitume Legends Project
% CarrEniX
% Mars 2022
%%%%%%%%%%%%%%%%%%%%%%%%%%%%%%%%%%%%%%%%%%%%%%%%%%%%%%%%%%%%%%%%%%%%%%%%%%%%%%%%

% Packages
\documentclass[12pt,a4paper]{article}
\usepackage{mathtools}
\usepackage[utf8]{inputenc}
\usepackage{graphicx}
\usepackage[french]{babel}
\usepackage[T1]{fontenc}
\usepackage{url}
\usepackage{fancyhdr}
\usepackage{longtable}
\usepackage[aboveskip=.5cm]{caption}
\usepackage{pdflscape}
\usepackage{geometry}
% this package can be remove if the frames are on export.
\usepackage{hyperref}

% Global settings of document
\setlength\parindent{20pt}
\newcommand{\btmlgs}{\textit{Bitume Legends}}
\newcommand{\AI}{Intelligence Artificielle}
\newcommand{\CEX}{\textsc{CarrEniX}}
\pagestyle{fancy}
\rhead{\btmlgs}
\lhead{Rapport de Première Soutenance}
\rfoot{Page \thepage}
\fancyfoot[C]{}
\setlength{\headheight}{15pt}

% Begin of the document
\begin{document}

\begin{titlepage}
    \newcommand{\HRule}{\rule{\linewidth}{0.5mm}}
    \center
    \text{\LARGE Projet \btmlgs}\\[1cm]
    add image
    %   \includegraphics[scale=0.7]{../Medias/logo192.png} \\[1cm]
    \HRule \\[0.4cm]
    { \huge \bfseries Rapport de Première Soutenance\\[0.3cm] }
    \HRule \\[1.5cm]
    {\large \bfseries \CEX} \\[0.3cm]
    Anthony CARON\;--\;Melvyn DELAROQUE\\ Victorien CAMBOURIAN\;--\;Xavier de PLACE
    \\ [6cm]
    EPITA INFOSUP 2026\\Année 2021 - 2022\\
    \(\mathtt{\href{https://btms.games/}{btms.games}}\)
\end{titlepage}

\tableofcontents
\clearpage

\section{Introduction}
Nous sommes le studio CarrEnix qui développe \btmlgs, un jeu de course 3D réaliste en
\textit{low-poly}. Après avoir commencé le projet début janvier, nous publions notre 
premier rapport. Il permet de faire le point sur ce que nous avons fait, les objectifs
atteints, les priorités ainsi que les ressources utilisés durant ce projet. Dans ce rapport,
nous allons aborder l'avancée de notre projet, l'organisation de notre groupe, les 
ressources et créations de chacun des membres du groupe ainsi que leurs valeurs 
ajoutées, le ressenti des membres durant cette première période ainsi que nos objectifs.



\section{Avancée du Projet}
    \subsection{Multijoueur}
    \subsection{Programme Beta}
    \subsection{Site Internet}
    \subsection{Menu}

        (fonctions principales)\\
        (design)
        Là, Melvyn et Victorien ont commencé à travailler un design plus poussé que celui de base d'unity, en adoptant notamment un thème de couleur basé sur le logo du jeu.
        Tout d'abord une ébauche du design a été réalisée sur Canva, dans le but de partager des idées sur le look final du menu.
        Ensuite nous avons implémenté le design dans le jeu, en repositionnant les boutons, insérant un fond et des espaces pour les futures fonctionnalités. Nous avons également implémenté le code couleur du design.
        Nous avons également ajouté des effets sonores et de couleurs aux boutons.

\clearpage

\section{Organisation de notre Groupe}
    \subsection{Organisation pratique}
    Ensuite, il nous a fallu mettre en place une organisation particulière 
    entre nous, car au début nous ne savions pas spécialement par où commencer.
    Chaque semaine, Nous nous sommes réunis des réunions, définissant des 
    objectifs pour chacun, à faire durant la semaine suivante. Cela nous 
    a permis d’avoir des buts sur le court terme et d'avancer plus 
    efficacement.

    \subsection{Organisation personnelle}
    Pour les objectifs, nous nous 
    sommes basés sur ce que nous avions annoncé dans notre Cahier des 
    Charges :
    Melvyn s'est occupé des musiques d'ambiance, du \textit{Sound Design} de 
    notre jeu et du menu principal. Victorien a créé le premier circuit du jeu,
    il a implémenté le menu principal et a modélisé sur Blender la F1. Anthony a
    géré le site Internet, les réseaux sociaux et le programme \(\beta\).
    Enfin, Xavier a implémenté le mode multijoueur, et le cœur du moteur de jeu,
    à savoir le système de direction des voitures.\\
    {\bfseries Ajouter ici ce que chacun a fait en perso}

\subsubsection{Melvyn}
Étant responsable du \textit{sound-design} et de la musique, j'ai composé la 
musique du jeu. J'ai utilisé le logiciel de MAO (Musique Assistée par Ordinateur) 
\textit{FL Studio 20}. J'ai composé deux musiques, une musique pour le menu et 
une musique de course et j'ai demandé à un ami d'enregistrer sa voix pour la musique dans 
un studio d'enregistrement de son école. Le style choisi par mes pairs et 
moi-même au sein du groupe est la Phonk.\\
La phonk est un genre de Trap rapide de 160 battements par minutes et très énergique. 
Ce genre est connu pour ses samples (extraits musicaux) de voix tirés de morceaux du
rap de Memphis. Des basses et des percussions puissantes sont également nécessaires, 
ainsi que l'usage de la \textit{cowbell} (percussion reprenant les cloches 
autour du cou des vaches) venant de la \textit{Roland TR-808}, 
une boîte à rythme mémorable de la culture hip-hop. 
Ce style est souvent associé à la culture automobile \textit{underground} 
japonaise, d'où le désir d'utiliser de la phonk pour notre jeu.

\subsubsection{Victorien}
Étant donné que nous créons un jeu de voiture, il était 
bienvenu de modéliser au moins une voiture par nos propres soins. 
Victorien, ayant des connaissances dans \textit{Blender}, un logiciel 
de modélisation 3D, s'en est donc chargé. 
Nous avons convenu d'une formule 1, voiture considéré comme "Graal" du jeux. 
Les formes de la voiture ont toutes été créées à partir de formes basiques,
telles que des rectangles, des cylindres, des sphères ou encore des cubes. 
Ensuite, il ne restait qu'à jouer avec les formes : les étirer, 
redimensionner une face pour que les formes correspondent avec une Formule 1. 
Etant donné que le design est en \textit{low-poly}, nous avons rajouter des. % répétition de étant donné
faces à certaines formes. Par exemple, avec 26 rectangles verticaux collés les 
uns aux autres, on peut former un cylindre.
Pour essayer d'être réaliste, nous avons regarder une 
vidéo présentant une formule 1 avec ses détails telles que le volant, 
les suspensions, les ailerons et moustaches de la voiture.

\subsubsection{Anthony}
\textbf{TODO}


\subsubsection{Xavier}
\textbf{TODO}



\clearpage
\section{Ressources Utilisées}
\subsection{Collaboration}
La création de jeux en solo ne demande pas de partage de données, contrairement 
à celle en équipe. Nous avions besoin d'un endroit de collaboration où nous 
pourrions nous échanger les fichiers relatifs au jeu. 
Nous avons donc créé un ensemble de \textit{repositeries} sur \textit{Github}
\footnote{\(\mathtt{\href{https://github.com/Bitume-Legends-Crew}{github.com/Bitume-Legends-Crew}}\)},
répartis dans une organisation, chacun pour un usage bien spécifique.
Le premier est donc le \textit{repo} de notre jeu, nommé \textit{game}, et qui
contient le projet au format Unity et des dossiers contenants les différents 
assets et autres ressources, nécessaires à l'execution du jeu. 
Le second est un \textit{repo} consacré exclusivement aux différents rapports 
que nous devons fournir. Il est composé à 99\% de .\TeX\, et 1\% de \(\mathtt{.pdf}\).
Et le dernier est celui dédié à notre site Internet, qui posssède une double
fonction : il nous permet de collaborer sur le site mais aussi de l'héberger 
grâce à \textit{GitHub Pages}.\\
Ainsi, nous pouvons toujours être à jour sur la bonne version du jeu,
du site ou des rapports, tout en étant géographiquement à distance les uns 
des autres.

\subsection{Communication}
Pour communiquer entre nous et avec notre public de \(\beta\)-testeurs, nous
avons créé un serveur Discord \footnote{\(\mathtt{\href{https://discord.
gg/5NR43GHUBD}{discord.gg/5NR43GHUBD}}\)} découpé en multiples 
\textit{channels}, ayant chacun une mission précise pour ne pas mélanger les
informations. Ce serveur est aussi le lieu de nos réunions hebdomadaires 
(ou plus fréquemment en cas de soutenance).

\subsection{3D}
Comme moteur de jeu, nous avons utilisé \textit{Unity} pour sa simplicité de prise en main 
et une fonctionnalité très utile : l'\textit{Asset Store}. C'est une plateforme où nous pouvons 
acheter ou utiliser gratuitement des ressources telles que des bâtiments, des voitures ainsi
que des personnages. En plus de l'utilisation de l'\textit{Asset Store} pour 
notre première voiture, 
nous avons aussi utilisé \textit{Blender}, logiciel de modélisation 3D. 
Il nous a permi entre autres de créer une voiture qui sera utilisée et 
implémentée plus tard dans le jeu. 

\clearpage

\subsection{Musique}
La musique a été composée sur \textit{FL Studio 20} logiciel de MAO
très ludique et permissif. Le processus d'élaboration a commencé par
l'écriture de la mélodie principale. Ensuite vient le travail des percussions
(cruciales pour la phonk) représentant 70\% du travail total.
Nous avons travaillé sur le \textit{mixing} et le \textit{leveling},
procédés de travail du volume, de la stéréo et d'effets de spacialisation 
du son. Cette étape représente la deuxième partie la plus importante de la 
composition, afin de donner une bonne qualité sonore à la musique.
Enfin, l'arrangement et le mastering permettant pour l'un, une 
structure cohérente de la musique et pour l'autre, de légères retouches
sonores faisant toute la différence. Cette dernière partie du processus 
est la plus lente mais celle sur laquelle on peut exprimer sa créativité,
car c'est ici que l'on peut l'on peut travailler sur des variations, 
des transitions et ajouter du punch.

\subsection{Site Internet}
Pour réaliser le site Internet, nous avons choisi d'utiliser \textit{Bootstrap 
Studio 5} afin de limiter le développement en HTML et CSS. \textit{Bootstrap 
Studio} est une application de conception de site Web très puissante,
nous pouvons choisir des structures déja prédéfinies comme des colonnes
et des lignes puis y ajouter ce que l'on souhaite, notamment des boutons, 
des images, des liens, du texte... Enfin, \textit{Bootstrap Studio} nous 
propose de personnaliser ces derniers éléments en modifiant leurs styles 
et leur place afin de limiter l'usage du CSS. Nous sommes quand même passé 
par le HTML pour customiser plus profondément certains éléments de notre site,
comme par exemple les boutons. C'est alors qu'intervient l'éditeur de texte
\textit{Visual Studio Code} dans lequel nous avons exporté le code généré 
par Bootstrap et intégré les propriétés manquantes à chaque bouton. 
Pour mettre en ligne le site une fois terminé, nous avos utilisé
la solution \textit{GitHub Pages} pour l'hébergement. Nous avons raccordé
le nom de domaine à notre \textit{repository} puis nous avons configuré
les \textit{DNS} (le système qui permet à un site web d'être visible sur
Internet) pour que tout fonctionne bien.

\subsection{IDE}
Pour écrire notre code en C\#, nous utilisons l'IDE \textit{Rider} de 
\textit{JetBrains}. Il possède une bonne intégration de Unity, et nous
y sommes bien habitué, c'est celui que nous utilisons au quotidien pour
nos TP de programmation. Pour faire certains tests, très précis et qui
ne nécessitent pas de beaucoup de ressources, nous utilisons \textit{Vim}
directement dans notre terminal.
Pour collaborer sur le rapport et les autres documents en \LaTeX\, 
pour éviter de désigner un "esclave \LaTeX", nous utilisons le site 
\textit{Overleaf}\footnote{\(\mathtt{\href{https://www.overleaf.com/}
{overleaf.com}}\)}. À la manière d'un document en ligne, un Google Docs
ou autre, nous pouvons écrire en simultané et compléter à quatre cerveaux
les documents demandés.

\clearpage

\section{Ressenti de chacun}
\subsection{Anthony}
\textit{L’organisation du travail au sein du groupe était plutôt bonne.
    Dès le début, nous avons créé un GitHub afin de partager notre travail 
    effectué ainsi qu’un Discord pour se mettre d’accord sur les différentes 
    deadlines à propos du travail à effectué chaque semaine. Ainsi nous 
    effectuons toutes les semaines, des réunions en vocal, dans le but 
    d’expliquer aux autres membres du groupe ce que l’on a fait dans la 
    semaine. A propos du site web, j’ai eu du mal à manipuler Bootstrap Studio 
    du fait du manque de tutoriaux pour apprendre à utiliser Bootstrap Studio
    sans de développement en HTML. J’ai donc dû apprendre comment fonctionne 
    ce logiciel et après avoir réalisé que le principe était de mettre tous 
    les éléments liés dans une seule colonne, Bootstrap est devenu toute de 
suite très facile et m’a permis de réaliser rapidement le site web.}

\subsection{Melvyn}
\textbf{TODO}
\textit{L'intérêt personnel que j'ai envers le développement d'un jeu 
    de course automobile vient de ma passion depuis l'enfance pour ce genre de 
    jeux. En effet, plus jeune et encore aujourd'hui, je joue à des jeux de 
    voitures (Need For Speed, Mario Kart ou Asphalt) et je regarde beaucoup 
    de contenu automobile à la télévision (Top Gear UK) et sur Youtube 
    (Vilebrequin). C'est un milieu qui mêle à la fois passion et mentalité 
    d'ingénieur, vu qu'elle permet de répondre à la question : comment ces 
    jeux que j'adore ont été fait ? L'expérience acquise en travaillant 
    directement sur un projet de développement de jeux vidéoludique 
    permettrait de développer mes compétences à la fois techniques et 
    artistiques et d'apprendre concrètement ce qu'est le travail de groupe en 
milieu professionnel}

\subsection{Victorien}
\textit{Quelques années auparavant, j’avais déjà créé un jeux-vidéo 
    sur un autre moteur de jeux, \textit{Unreal Engine} de \textit{Epic Games}. 
    Je n’avait cependant pas eu à coder car il suffisait de créer des
    \textit{Blueprints} (élements de code préfabriqués qu’il faut relier 
    entre eux). De plus, c’était un jeu développé en solo, pas en équipe.
    Je ne savais pas spécialement par quoi commencer au vu de ce que l’on
    avait prévu. Après avoir fixé des objectifs hebdomadaires grâce aux 
    réunions, les choses étaient cadrées et j’ai pu être plus efficace sur
    le développement du jeu. Il a fallu également faire attention à notre 
    emploi du temps. Bien qu’il y eût les vacances pour avancer, il y avait 
    à la rentrée les Midterms et 5 jours plus tard première soutenance. 
Nous devions être organisé et voir plus loin que la semaine suivante.}

\subsection{Xavier}
\textbf{TODO}
\textit{C'est un projet qui me tient à coeur personnellement. Il 
    va m'apporter de la discipline pour le travail en équipe. Il va aussi me 
    permettre de découvrir un monde que je ne connais pas très bien, celui du 
    jeu vidéo. Enfin, Bitume Legends est une occasion unique pour faire un projet 
presque entièrement libre, donc quelque chose dont on peut vraiment être fier.}
\clearpage

\section{Objectifs pour la suite}
\subsection{Finir le menu}
Le menu principal du jeu est presque fini, il nous manque un onglet 
pour les paramètres du jeu et il faut que nous relions les autres
fonctions du jeu tel que le mode \textit{timer} ou le mode \textit{vs \AI}.
Ensuite, nous devrons mettre au point le menu \textit{Player}, en rajoutant 
la possibilité de voir son niveau, de modifier son nom, etc.

\subsection{Finir le site Web}
Le site web est bien avancé, mais il nous manque certains éléments 
comme la fin de la chronologie, la page de téléchargement avec les liens,
ou encore deux trois bugs concernant l'affichage.

\subsection{Continuer les musiques}

\subsection{Implem l'\AI}
\subsection{Gamplay}

Logoden biniou degemer mat an penn ar bed kenavo, kilpenn kador feiz gwengolo 
mel goulenn kaer c’higer etre, bezh benn arc’h outo amzer divjod vouezh. Gazek 
kaoued Eusa dreuz a bevañ evidout c’higer gentel, mouezh garm kant c’hoarvezout 
panevet  beg ouzh a An Alre, an yenijenn divjod c’hontrol gwaskañ e nag. Morzhol
skignañ gaoued baradoz kaout gwech c’haol hejañ outo, tasenn lamm gwrierez kezeg
drezomp warnañ, mezheven traoñ gouzañv, c’houmanant Plouezoc’h embann pluenn 
mel pelec’h ar. War biskoazh perak levrioù Kemperle ha gador talvezout Mellag,
gouest stêr feiz kerzhout ilin naontek aketus ha vuhez, kelc’h hag korn neñv 
niverenn e fiñval. Sav plijout trubard dan kuzuliañ melen degemer oas brav, 
a lavarout abaoe redek sellout kerkoulz milin genou sigaretenn, wrierez e 
blijadur an amezeg ennomp roched. Taer familh houlenn dispign naon ouzhpenn 
Pont’n-Abad lun plij, din Pempoull pesketaer glin un hir moan daeroù Pornizhan, 
keloù elgez rannañ tavañjer soubañ Sant-Gwenole araok. Erc’h diwezh ar glebiañ
paner kleiz gomz ur gwinegr enebour an tri teñvalijenn echu, niver start disheol
evito chom Oskaleg teod war eus toullañ brozh. Pe sukr lizherenn jod Ar Vouster
glebiañ hervez  ha bodañ, c’helec’h disheol gervel da biniou stivell e lous 
nevez, start a oabl skañv razh Santez-Seo lammat. Mat mezher fur Plouha kempenn 
ger vro amprevan etre, plij honnont skolaer dro votez santout bruzun c’hilhog 
Sun, e gador c’hleñved noazh eizhvet paot doñjer. Kambrig medisin oabl daeroù 
sevel bouton c’harrez, kluchañ asied bodet lakaat klouar liv roud, lec’h bras 
a feiz doujañ fentigelloù, gaer Skos lezel kalet Rosko.

\section{Conclusion}
En conclusion, ..........SUITE A FAIRE
paner kleiz gomz ur gwinegr enebour an tri teñvalijenn echu, niver start disheol
evito chom Oskaleg teod war eus toullañ brozh. Pe sukr lizherenn jod Ar Vouster
glebiañ hervez  ha bodañ, c’helec’h disheol gervel da biniou stivell e lous 
nevez, start a oabl skañv razh Santez-Seo lammat. Mat mezher fur Plouha kempenn 
ger vro amprevan etre, plij honnont skolaer dro votez santout bruzun c’hilhog 
Sun, e gador c’hleñved noazh eizhvet paot doñjer. Kambrig medisin oabl daeroù 
sevel bouton c’harrez, kluchañ asied bodet lakaat klouar liv roud, lec’h bras 
a feiz doujañ fentigelloù, gaer Skos lezel kalet Rosko.

\begin{thebibliography}{9}

\end{thebibliography}

\begin{center}
    Made with $\heartsuit$ by \CEX\, on \LaTeX.\\
    \textcopyright\, 2021-2022 \btmlgs\\
    \(\mathtt{\href{https://btms.games/}{btms.games}}\)
\end{center}
\end{document}
