%%%%%%%%%%%%%%%%%%%%%%%%%%%%%%%%%%%%%%%%%%%%%%%%%%%%%%%%%%%%%%%%%%%%%%%%%%%%%%%%
% Soutenance 1
% Bitume Legends Project
% CarrEniX
% Mars 2022
%%%%%%%%%%%%%%%%%%%%%%%%%%%%%%%%%%%%%%%%%%%%%%%%%%%%%%%%%%%%%%%%%%%%%%%%%%%%%%%%

% Packages
\documentclass[12pt,a4paper]{article}
\usepackage{mathtools}
\usepackage[utf8]{inputenc}
\usepackage{graphicx}
\usepackage[french]{babel}
\usepackage[T1]{fontenc}
\usepackage{url}
\usepackage{fancyhdr}
\usepackage{longtable}
\usepackage[aboveskip=.5cm]{caption}
\usepackage{pdflscape}
\usepackage{geometry}

% Global settings of document
\setlength\parindent{20pt}
\newcommand{\btmlgs}{\textit{Bitume Legends}}
\newcommand{\AI}{Intelligence Artificielle}
\newcommand{\CEX}{CarrEniX}
\pagestyle{fancy}
\rhead{\btmlgs}
\lhead{Rapport de Soutenance 1}
\rfoot{Page \thepage}
\fancyfoot[C]{}
\setlength{\headheight}{15pt}

% Begin of the document
\begin{document}

\begin{titlepage}
  \newcommand{\HRule}{\rule{\linewidth}{0.5mm}}
  \center
  \text{\LARGE Projet \btmlgs}\\[1cm]
  \includegraphics[scale=0.7]{../Medias/logo192.png} \\[1cm]
  \HRule \\[0.4cm]
  { \huge \bfseries Rapport de Soutenance 1\\[0.15cm] }
  \HRule \\[1.5cm]
  {\large \bfseries \CEX} \\[0.3cm]
  Anthony CARON\;--\;Melvyn DELAROQUE\\ Victorien CAMBOURIAN\;--\;Xavier de PLACE
   \\ [6cm]
  EPITA INFOSUP 2026\\Année 2021 - 2022
  \end{titlepage}

\tableofcontents
\clearpage

\section{Introduction}
  Logoden biniou degemer mat an penn ar bed kenavo, kilpenn kador feiz gwengolo 
  mel goulenn kaer c’higer etre, bezh benn arc’h outo amzer divjod vouezh. Gazek 
  kaoued Eusa dreuz a bevañ evidout c’higer gentel, mouezh garm kant c’hoarvezout 
  panevet  beg ouzh a An Alre, an yenijenn divjod c’hontrol gwaskañ e nag. Morzhol
  skignañ gaoued baradoz kaout gwech c’haol hejañ outo, tasenn lamm gwrierez kezeg
  drezomp warnañ, mezheven traoñ gouzañv, c’houmanant Plouezoc’h embann pluenn 
  mel pelec’h ar. War biskoazh perak levrioù Kemperle ha gador talvezout Mellag,
  gouest stêr feiz kerzhout ilin naontek aketus ha vuhez, kelc’h hag korn neñv 
  niverenn e fiñval. Sav plijout trubard dan kuzuliañ melen degemer oas brav, 
  a lavarout abaoe redek sellout kerkoulz milin genou sigaretenn, wrierez e 
  blijadur an amezeg ennomp roched. Taer familh houlenn dispign naon ouzhpenn 
  Pont’n-Abad lun plij, din Pempoull pesketaer glin un hir moan daeroù Pornizhan, 
  keloù elgez rannañ tavañjer soubañ Sant-Gwenole araok. Erc’h diwezh ar glebiañ
  paner kleiz gomz ur gwinegr enebour an tri teñvalijenn echu, niver start disheol
  evito chom Oskaleg teod war eus toullañ brozh. Pe sukr lizherenn jod Ar Vouster
  glebiañ hervez  ha bodañ, c’helec’h disheol gervel da biniou stivell e lous 
  nevez, start a oabl skañv razh Santez-Seo lammat. Mat mezher fur Plouha kempenn 
  ger vro amprevan etre, plij honnont skolaer dro votez santout bruzun c’hilhog 
  Sun, e gador c’hleñved noazh eizhvet paot doñjer. Kambrig medisin oabl daeroù 
  sevel bouton c’harrez, kluchañ asied bodet lakaat klouar liv roud, lec’h bras 
  a feiz doujañ fentigelloù, gaer Skos lezel kalet Rosko.

\section{Avancée du Projet}
  Logoden biniou degemer mat an penn ar bed kenavo, kilpenn kador feiz gwengolo 
  mel goulenn kaer c’higer etre, bezh benn arc’h outo amzer divjod vouezh. Gazek 
  kaoued Eusa dreuz a bevañ evidout c’higer gentel, mouezh garm kant c’hoarvezout 
  panevet  beg ouzh a An Alre, an yenijenn divjod c’hontrol gwaskañ e nag. Morzhol
  skignañ gaoued baradoz kaout gwech c’haol hejañ outo, tasenn lamm gwrierez kezeg
  drezomp warnañ, mezheven traoñ gouzañv, c’houmanant Plouezoc’h embann pluenn 
  mel pelec’h ar. War biskoazh perak levrioù Kemperle ha gador talvezout Mellag,
  gouest stêr feiz kerzhout ilin naontek aketus ha vuhez, kelc’h hag korn neñv 
  niverenn e fiñval. Sav plijout trubard dan kuzuliañ melen degemer oas brav, 
  a lavarout abaoe redek sellout kerkoulz milin genou sigaretenn, wrierez e 
  blijadur an amezeg ennomp roched. Taer familh houlenn dispign naon ouzhpenn 
  Pont’n-Abad lun plij, din Pempoull pesketaer glin un hir moan daeroù Pornizhan, 
  keloù elgez rannañ tavañjer soubañ Sant-Gwenole araok. Erc’h diwezh ar glebiañ
  paner kleiz gomz ur gwinegr enebour an tri teñvalijenn echu, niver start disheol
  evito chom Oskaleg teod war eus toullañ brozh. Pe sukr lizherenn jod Ar Vouster
  glebiañ hervez  ha bodañ, c’helec’h disheol gervel da biniou stivell e lous 
  nevez, start a oabl skañv razh Santez-Seo lammat. Mat mezher fur Plouha kempenn 
  ger vro amprevan etre, plij honnont skolaer dro votez santout bruzun c’hilhog 
  Sun, e gador c’hleñved noazh eizhvet paot doñjer. Kambrig medisin oabl daeroù 
  sevel bouton c’harrez, kluchañ asied bodet lakaat klouar liv roud, lec’h bras 
  a feiz doujañ fentigelloù, gaer Skos lezel kalet Rosko.

\section{Organisation de notre Groupe}
  Nous nous sommes basés sur ce que nous avions annoncé dans notre Cahier des 
  Charges.
  Melvyn s'est occupé des musiques d'ambiance, du \textit{Sound Design} de 
  notre jeu et du menu principal. Victorien a créé le premier circuit du jeu,
  il a implémenté le menu principal et a modélisé sur Blender la F1. Anthony a
  géré le site Internet, les réseaux sociaux et le programme de \(\beta\).
  Enfin, Xavier a implémenté le mode multijoueur, et le cœur du moteur de jeu,
  à savoir le système de direction des voitures.

\section{Ressources}
  Logoden biniou degemer mat an penn ar bed kenavo, kilpenn kador feiz gwengolo 
  mel goulenn kaer c’higer etre, bezh benn arc’h outo amzer divjod vouezh. Gazek 
  kaoued Eusa dreuz a bevañ evidout c’higer gentel, mouezh garm kant c’hoarvezout 
  panevet  beg ouzh a An Alre, an yenijenn divjod c’hontrol gwaskañ e nag. Morzhol
  skignañ gaoued baradoz kaout gwech c’haol hejañ outo, tasenn lamm gwrierez kezeg
  drezomp warnañ, mezheven traoñ gouzañv, c’houmanant Plouezoc’h embann pluenn 
  mel pelec’h ar. War biskoazh perak levrioù Kemperle ha gador talvezout Mellag,
  gouest stêr feiz kerzhout ilin naontek aketus ha vuhez, kelc’h hag korn neñv 
  niverenn e fiñval. Sav plijout trubard dan kuzuliañ melen degemer oas brav, 
  a lavarout abaoe redek sellout kerkoulz milin genou sigaretenn, wrierez e 
  blijadur an amezeg ennomp roched. Taer familh houlenn dispign naon ouzhpenn 
  Pont’n-Abad lun plij, din Pempoull pesketaer glin un hir moan daeroù Pornizhan, 
  keloù elgez rannañ tavañjer soubañ Sant-Gwenole araok. Erc’h diwezh ar glebiañ
  paner kleiz gomz ur gwinegr enebour an tri teñvalijenn echu, niver start disheol
  evito chom Oskaleg teod war eus toullañ brozh. Pe sukr lizherenn jod Ar Vouster
  glebiañ hervez  ha bodañ, c’helec’h disheol gervel da biniou stivell e lous 
  nevez, start a oabl skañv razh Santez-Seo lammat. Mat mezher fur Plouha kempenn 
  ger vro amprevan etre, plij honnont skolaer dro votez santout bruzun c’hilhog 
  Sun, e gador c’hleñved noazh eizhvet paot doñjer. Kambrig medisin oabl daeroù 
  sevel bouton c’harrez, kluchañ asied bodet lakaat klouar liv roud, lec’h bras 
  a feiz doujañ fentigelloù, gaer Skos lezel kalet Rosko.
\section{Ressenti de chacun}
  \indent\textit{Ce projet sera mon tout premier réalisé en groupe sur une aussi 
  grande période de temps. J'aurai l'occasion d'apprendre les côtés positifs et 
  négatifs du travail de groupe afin de mener ce projet à bien. De plus, grand
  fan de jeux vidéos, j'ai toujours voulu faire mon propre jeux vidéo et ce 
  projet sera l'occasion pour moi de voir ce qui se cache derrière.} \\
  \indent Anthony\\[0.3cm]
  \indent\textit{L'intérêt personnel que j'ai envers le développement d'un jeu 
  de course automobile vient de ma passion depuis l'enfance pour ce genre de 
  jeux. En effet, plus jeune et encore aujourd'hui, je joue à des jeux de 
  voitures (Need For Speed, Mario Kart ou Asphalt) et je regarde beaucoup 
  de contenu automobile à la télévision (Top Gear UK) et sur Youtube 
  (Vilebrequin). C'est un milieu qui mêle à la fois passion et mentalité 
  d'ingénieur, vu qu'elle permet de répondre à la question : comment ces 
  jeux que j'adore ont été fait ? L'expérience acquise en travaillant 
  directement sur un projet de développement de jeux vidéoludique 
  permettrait de développer mes compétences à la fois techniques et 
  artistiques et d'apprendre concrètement ce qu'est le travail de groupe en 
  milieu professionnel} \\
  \indent Melvyn\\[0.3cm]
  \indent\textit{Bien que j'aime bien être chef de projet, je trouve qu'il était 
  intéressant pour moi de me positionner différement, pour pouvoir également voir 
  d'autres façons de cadrer et diriger une équipe. De plus, ce projet va nous
  permettre de voir les différentes méthodes d'approche de résolution des 
  problèmes auxquels nous serons confronté, tout en s'y adaptant (lors de la 
  correction des bugs par exemple). Aussi, nous allons devoir être organisés
  et efficaces étant donné que le développement du jeu se fait sur le temps 
  libre en plus de l'école.} \\
  \indent Victorien\\[0.3cm]
  \indent\textit{C'est un projet qui me tient à coeur personnellement. Il 
  va m'apporter de la discipline pour le travail en équipe. Il va aussi me 
  permettre de découvrir un monde que je ne connais pas très bien, celui du 
  jeu vidéo. Enfin, Bitume Legends est une occasion unique pour faire un projet 
  presque entièrement libre, donc quelque chose dont on peut vraiment être fier.} \\
  \indent Xavier

\begin{thebibliography}{9}
  
\end{thebibliography}
  
\begin{center}
  Made with $\heartsuit$ by \CEX\, on \LaTeX.\\
  \textcopyright\, 2021-2022 \btmlgs
\end{center}
\clearpage
\end{document}
