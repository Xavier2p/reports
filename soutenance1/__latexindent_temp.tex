%%%%%%%%%%%%%%%%%%%%%%%%%%%%%%%%%%%%%%%%%%%%%%%%%%%%%%%%%%%%%%%%%%%%%%%%%%%%%%%%
% Soutenance 1
% Bitume Legends Project
% CarrEniX
% Mars 2022
%%%%%%%%%%%%%%%%%%%%%%%%%%%%%%%%%%%%%%%%%%%%%%%%%%%%%%%%%%%%%%%%%%%%%%%%%%%%%%%%

% Packages
\documentclass[12pt,a4paper]{article}
\usepackage{mathtools}
\usepackage[utf8]{inputenc}
\usepackage{graphicx}
\usepackage[french]{babel}
\usepackage[T1]{fontenc}
\usepackage{url}
\usepackage{fancyhdr}
\usepackage{longtable}
\usepackage[aboveskip=.5cm]{caption}
\usepackage{pdflscape}
\usepackage{geometry}

% Global settings of document
\setlength\parindent{20pt}
\newcommand{\btmlgs}{\textit{Bitume Legends}}
\newcommand{\AI}{Intelligence Artificielle}
\newcommand{\CEX}{CarrEniX}
\pagestyle{fancy}
\rhead{\btmlgs}
\lhead{Rapport de Soutenance 1}
\rfoot{Page \thepage}
\fancyfoot[C]{}
\setlength{\headheight}{15pt}

% Begin of the document
\begin{document}

\begin{titlepage}
  \newcommand{\HRule}{\rule{\linewidth}{0.5mm}}
  \center
  \text{\LARGE Projet \btmlgs}\\[1cm]
  add image
%   \includegraphics[scale=0.7]{../Medias/logo192.png} \\[1cm]
  \HRule \\[0.4cm]
  { \huge \bfseries Rapport de Soutenance 1\\[0.15cm] }
  \HRule \\[1.5cm]
  {\large \bfseries \CEX} \\[0.3cm]
  Anthony CARON\;--\;Melvyn DELAROQUE\\ Victorien CAMBOURIAN\;--\;Xavier de PLACE
   \\ [6cm]
  EPITA INFOSUP 2026\\Année 2021 - 2022
  \end{titlepage}

\tableofcontents
\clearpage

\section{Introduction}
  Logoden biniou degemer mat an penn ar bed kenavo, kilpenn kador feiz gwengolo 
  mel goulenn kaer c’higer etre, bezh benn arc’h outo amzer divjod vouezh. Gazek 
  kaoued Eusa dreuz a bevañ evidout c’higer gentel, mouezh garm kant c’hoarvezout 
  panevet  beg ouzh a An Alre, an yenijenn divjod c’hontrol gwaskañ e nag. Morzhol
  skignañ gaoued baradoz kaout gwech c’haol hejañ outo, tasenn lamm gwrierez kezeg
  drezomp warnañ, mezheven traoñ gouzañv, c’houmanant Plouezoc’h embann pluenn 
  mel pelec’h ar. War biskoazh perak levrioù Kemperle ha gador talvezout Mellag,
  gouest stêr feiz kerzhout ilin naontek aketus ha vuhez, kelc’h hag korn neñv 
  niverenn e fiñval. Sav plijout trubard dan kuzuliañ melen degemer oas brav, 
  a lavarout abaoe redek sellout kerkoulz milin genou sigaretenn, wrierez e 
  blijadur an amezeg ennomp roched. Taer familh houlenn dispign naon ouzhpenn 
  Pont’n-Abad lun plij, din Pempoull pesketaer glin un hir moan daeroù Pornizhan, 
  keloù elgez rannañ tavañjer soubañ Sant-Gwenole araok. Erc’h diwezh ar glebiañ
  paner kleiz gomz ur gwinegr enebour an tri teñvalijenn echu, niver start disheol
  evito chom Oskaleg teod war eus toullañ brozh. Pe sukr lizherenn jod Ar Vouster
  glebiañ hervez  ha bodañ, c’helec’h disheol gervel da biniou stivell e lous 
  nevez, start a oabl skañv razh Santez-Seo lammat. Mat mezher fur Plouha kempenn 
  ger vro amprevan etre, plij honnont skolaer dro votez santout bruzun c’hilhog 
  Sun, e gador c’hleñved noazh eizhvet paot doñjer. Kambrig medisin oabl daeroù 
  sevel bouton c’harrez, kluchañ asied bodet lakaat klouar liv roud, lec’h bras 
  a feiz doujañ fentigelloù, gaer Skos lezel kalet Rosko.

\section{Avancée du Projet}
  Logoden biniou degemer mat an penn ar bed kenavo, kilpenn kador feiz gwengolo 
  mel goulenn kaer c’higer etre, bezh benn arc’h outo amzer divjod vouezh. Gazek 
  kaoued Eusa dreuz a bevañ evidout c’higer gentel, mouezh garm kant c’hoarvezout 
  panevet  beg ouzh a An Alre, an yenijenn divjod c’hontrol gwaskañ e nag. Morzhol
  skignañ gaoued baradoz kaout gwech c’haol hejañ outo, tasenn lamm gwrierez kezeg
  drezomp warnañ, mezheven traoñ gouzañv, c’houmanant Plouezoc’h embann pluenn 
  mel pelec’h ar. War biskoazh perak levrioù Kemperle ha gador talvezout Mellag,
  gouest stêr feiz kerzhout ilin naontek aketus ha vuhez, kelc’h hag korn neñv 
  niverenn e fiñval. Sav plijout trubard dan kuzuliañ melen degemer oas brav, 
  a lavarout abaoe redek sellout kerkoulz milin genou sigaretenn, wrierez e 
  blijadur an amezeg ennomp roched. Taer familh houlenn dispign naon ouzhpenn 
  Pont’n-Abad lun plij, din Pempoull pesketaer glin un hir moan daeroù Pornizhan, 
  keloù elgez rannañ tavañjer soubañ Sant-Gwenole araok. Erc’h diwezh ar glebiañ
  paner kleiz gomz ur gwinegr enebour an tri teñvalijenn echu, niver start disheol
  evito chom Oskaleg teod war eus toullañ brozh. Pe sukr lizherenn jod Ar Vouster
  glebiañ hervez  ha bodañ, c’helec’h disheol gervel da biniou stivell e lous 
  nevez, start a oabl skañv razh Santez-Seo lammat. Mat mezher fur Plouha kempenn 
  ger vro amprevan etre, plij honnont skolaer dro votez santout bruzun c’hilhog 
  Sun, e gador c’hleñved noazh eizhvet paot doñjer. Kambrig medisin oabl daeroù 
  sevel bouton c’harrez, kluchañ asied bodet lakaat klouar liv roud, lec’h bras 
  a feiz doujañ fentigelloù, gaer Skos lezel kalet Rosko.

\section{Organisation de notre Groupe}
\subsection{Collaboration}
La création de jeux en solo ne demande pas de partage de données, contrairement à celle en équipe. Nous avions besoin d'un endroit de collaboration où nous pourrions nous échanger les fichiers relatifs au jeu. 
Nous avons donc créé un ensemble de \textit{repositeries} sur \textit{Github}, répartis dans une organisation, chacun pour un usage bien spécifique.
Le premier est donc le \textit{repo} de notre jeu, nommé \textit{game}, et qui contient le projet au format Unity et des dossiers contenants les différents assets et autres ressources, nécessaires à l'execution du jeu. 
Le second est un \textit{repo} consacré exclusivement aux différents rapports que nous devons fournir. Il est composé à 99\% de \TeX\, et 1\% de .pdf.
Et le dernier est celui dédié à notre site Internet, qui posssède une double fonction : il nous permet de collaborer sur le site mais aussi de l'héberger grâce à \textit{GitHub Pages}.\\
\indent Ainsi, nous pouvons toujours être à jour sur la bonne version du jeu, du site ou des rapports, tout en étant géographiquement à distance les uns des autres.

\subsection{Organisation pratique}
De plus, il nous a fallu mettre en place une organisation particulière entre nous, car au début nous ne savions pas spécialement par où commencer.
Chaque semaine, Nous avons mis en place % répétition mise en place
 des réunions, définissant des objectifs pour chacun, à faire durant la semaine suivante. Cela nous a permis d’avoir des buts sur le court terme et d'avancer plus efficacement. Pour les objectifs, nous nous sommes basés sur ce que nous avions annoncé dans notre Cahier des 
  Charges :
  Melvyn s'est occupé des musiques d'ambiance, du \textit{Sound Design} de 
  notre jeu et du menu principal. Victorien a créé le premier circuit du jeu,
  il a implémenté le menu principal et a modélisé sur Blender la F1. Anthony a
  géré le site Internet, les réseaux sociaux et le programme de \(\beta\).
  Enfin, Xavier a implémenté le mode multijoueur, et le cœur du moteur de jeu,
  à savoir le système de direction des voitures.\\
  {\bfseries Ajouter ici ce que chacun a fait en perso}

\clearpage
\section{Ressources}
Étant donné que nous créons un jeu de voiture, il était bien venu de créer % répétition de créer
au moins une voiture par nos propres soins. 
Victorien, ayant des connaissances dans \textit{Blender}, notre logiciel de modélisation 3D, nous avons donc décidé de créer un modèle de voiture en pour notre jeu. 
% la phrase veut rien dire, il faut mettre tt ds un autre sens.
Nous avons donc pris pour modèle une Formule 1 de 2021, pour en faire la voiture ultime, notre “Graal”. 
% il manque un truc ds la phrase, ca sonne bizarre
Les formes de la voiture ont toutes été créées à partir de formes basiques, telles que des rectangles, des cylindres, des sphères ou encore des cubes. 
Ensuite, il nous a fallu créer des points sur ces formes pour pouvoir les étirer et les transformer à notre guise et ainsi obtenir notre Formule 1 en \textit{low-poly}. % il manque pas des étapes ?
Il nous a fallu % répétition fallu
toutefois regarder une vidéo pour préciser quelques détails sur la voiture, tels que les boutons sur le volant, les suspensions ainsi que les ailerons. 

\clearpage
\section{Ressenti de chacun}

  \subsection{Anthony}
  \indent\textit{\bfseries Ce projet sera mon tout premier réalisé en groupe sur une aussi 
  grande période de temps. J'aurai l'occasion d'apprendre les côtés positifs et 
  négatifs du travail de groupe afin de mener ce projet à bien. De plus, grand
  fan de jeux vidéos, j'ai toujours voulu faire mon propre jeux vidéo et ce 
  projet sera l'occasion pour moi de voir ce qui se cache derrière.}
  \subsection{Melvyn}
  
  \indent\textit{\bfseries L'intérêt personnel que j'ai envers le développement d'un jeu 
  de course automobile vient de ma passion depuis l'enfance pour ce genre de 
  jeux. En effet, plus jeune et encore aujourd'hui, je joue à des jeux de 
  voitures (Need For Speed, Mario Kart ou Asphalt) et je regarde beaucoup 
  de contenu automobile à la télévision (Top Gear UK) et sur Youtube 
  (Vilebrequin). C'est un milieu qui mêle à la fois passion et mentalité 
  d'ingénieur, vu qu'elle permet de répondre à la question : comment ces 
  jeux que j'adore ont été fait ? L'expérience acquise en travaillant 
  directement sur un projet de développement de jeux vidéoludique 
  permettrait de développer mes compétences à la fois techniques et 
  artistiques et d'apprendre concrètement ce qu'est le travail de groupe en 
  milieu professionnel}
 
  \subsection{Victorien}
  \indent\textit{Quelques années auparavant, j’avais déjà créé un jeux-vidéo sur un autre moteur de jeux, \textit{Unreal Engine} de \textit{Epic Games}. 
  Je n’avait cependant pas eu à coder car il suffisait de créer des \textit{Blueprints} (élements de code préfabriqués qu’il faut relier entre eux).
  De plus, c’était un jeu développé en solo, pas en équipe. Je ne savais pas spécialement par quoi commencer au vu de ce que l’on avait prévu.
  Après avoir fixé des objectifs hebdomadaires grâce aux réunions, les choses étaient cadrées et j’ai pu être plus efficace sur le développement du jeu. 
  Il a fallu également faire attention à notre emploi du temps. 
  Bien qu’il y eût les vacances pour avancer, il y avait à la rentrée les Midterms et 5 jours plus tard première soutenance. 
  Il nous a fallu être organisé et voir plus loin que la semaine suivante. } % répétition de fallu
  
  \subsection{Xavier}
  \indent\textit{\bfseries C'est un projet qui me tient à coeur personnellement. Il 
  va m'apporter de la discipline pour le travail en équipe. Il va aussi me 
  permettre de découvrir un monde que je ne connais pas très bien, celui du 
  jeu vidéo. Enfin, Bitume Legends est une occasion unique pour faire un projet 
  presque entièrement libre, donc quelque chose dont on peut vraiment être fier.}
\clearpage

\section{Objectifs pour la suite}
  Logoden biniou degemer mat an penn ar bed kenavo, kilpenn kador feiz gwengolo 
  mel goulenn kaer c’higer etre, bezh benn arc’h outo amzer divjod vouezh. Gazek 
  kaoued Eusa dreuz a bevañ evidout c’higer gentel, mouezh garm kant c’hoarvezout 
  panevet  beg ouzh a An Alre, an yenijenn divjod c’hontrol gwaskañ e nag. Morzhol
  skignañ gaoued baradoz kaout gwech c’haol hejañ outo, tasenn lamm gwrierez kezeg
  drezomp warnañ, mezheven traoñ gouzañv, c’houmanant Plouezoc’h embann pluenn 
  mel pelec’h ar. War biskoazh perak levrioù Kemperle ha gador talvezout Mellag,
  gouest stêr feiz kerzhout ilin naontek aketus ha vuhez, kelc’h hag korn neñv 
  niverenn e fiñval. Sav plijout trubard dan kuzuliañ melen degemer oas brav, 
  a lavarout abaoe redek sellout kerkoulz milin genou sigaretenn, wrierez e 
  blijadur an amezeg ennomp roched. Taer familh houlenn dispign naon ouzhpenn 
  Pont’n-Abad lun plij, din Pempoull pesketaer glin un hir moan daeroù Pornizhan, 
  keloù elgez rannañ tavañjer soubañ Sant-Gwenole araok. Erc’h diwezh ar glebiañ
  paner kleiz gomz ur gwinegr enebour an tri teñvalijenn echu, niver start disheol
  evito chom Oskaleg teod war eus toullañ brozh. Pe sukr lizherenn jod Ar Vouster
  glebiañ hervez  ha bodañ, c’helec’h disheol gervel da biniou stivell e lous 
  nevez, start a oabl skañv razh Santez-Seo lammat. Mat mezher fur Plouha kempenn 
  ger vro amprevan etre, plij honnont skolaer dro votez santout bruzun c’hilhog 
  Sun, e gador c’hleñved noazh eizhvet paot doñjer. Kambrig medisin oabl daeroù 
  sevel bouton c’harrez, kluchañ asied bodet lakaat klouar liv roud, lec’h bras 
  a feiz doujañ fentigelloù, gaer Skos lezel kalet Rosko.
 
\section{Conclusion}
Logoden biniou degemer mat an penn ar bed kenavo, kilpenn kador feiz gwengolo 
  mel goulenn kaer c’higer etre, bezh benn arc’h outo amzer divjod vouezh. Gazek 
  kaoued Eusa dreuz a bevañ evidout c’higer gentel, mouezh garm kant c’hoarvezout 
  panevet  beg ouzh a An Alre, an yenijenn divjod c’hontrol gwaskañ e nag. Morzhol
  skignañ gaoued baradoz kaout gwech c’haol hejañ outo, tasenn lamm gwrierez kezeg.

\begin{thebibliography}{9}
  
\end{thebibliography}
  
\begin{center}
  Made with $\heartsuit$ by \CEX\, on \LaTeX.\\
  \textcopyright\, 2021-2022 \btmlgs
\end{center}
\clearpage
\end{document}