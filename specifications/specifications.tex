\documentclass[11pt,a4paper]{article}
\usepackage{mathtools}
\usepackage[utf8]{inputenc}
\usepackage{graphicx}
\usepackage[french]{babel}
\usepackage[T1]{fontenc}
\usepackage{url}
\setlength\parindent{15pt}
\begin{document}

\title{Cahier des Charges \\ Projet Bitume Legends}
\author{CarrEniX}
\date{Janvier 2022}
\maketitle

\begin{center}
    \includegraphics[width=0.5\textwidth]{Medias/logo192.png}
\end{center}

\tableofcontents
\clearpage

\section{Introduction}
  \begin{enumerate}
    \item donner une vue d'ensemble au Jury
    \item faire ressortir l'intérêt principal du projet
    \item mettre en caleur le but final
  \end{enumerate}
  Structure à respecter pour tout le document:
  \begin{enumerate}
    \item QUOI: ce qui est fait ou qui doit l'être
    \item COMMENT: quels seront les moyens matériels et intellectuels utilisés?
    \item COMBIEN: quels seront les aspects économiques qui devront être pris en contre?
  \end{enumerate}
\clearpage

\section{Origine et Nature du projet}
Dans notre groupe, nous aimons beaucoup les voitures. Nous regardons tous des émissions comme 
\emph{Top Gear}, \emph{The Gran Tour} ou encore \emph{Vilebrequin}. Nous avons donc décider de faire
un projet qui se rapporte à cet univers. Notre projet est donc un jeu de voitures, spécialement un jeu
de courses. 
\clearpage

\section{Objet de l'étude}

\clearpage

\section{État de l'art}
Le premier jeu de voiture à proprement parler est \textit{Gran Track 10}, un jeu d'arcade signé Atari 
sorti en 1974. Le jeu était plutôt bien pensé pour l'époque. Nous étions face à un 
circuit vu de dessus en nous pouvions contrôler notre voiture avec un volant, un levier de vitesses
et un pédalier. Il sera mis à jour par \textit{Gran Track 20}, une version deux joueurs.
En 1986, c'est Sega qui sort sa version du jeu de voiture avec \textit{OutRun}. Le premier 
jeu de en 3D était \textit{Hard Drivin}, produit par Atari. Le très connu \textit{Mario Kart},
développé par Nitendo, est sorti en 1992 sur Super NES.\\
\indent Plus récemment, simulation avec F1 ou Gran Turismo et foot avec Rocket League ou encore 
rallye avec WRC et course avec Need For Speed et Forza Horizon 5.\\
\textbf{TO BE CONTINUED...}
\clearpage

\section{Découpage du projet}
  \subsection{Répartition des tâches} 
    Nous avons découpé le projet en 4 grands pôles et nous avons désigné une personne responsable
    pour chaque pôle. Le rôle du responsable n'est pas de faire tout le travail de son pôle mais
    de gérer les \textit{deadlines} de son domaine et de coordonner le travail effectué.
    \begin{enumerate}
      \item Communication et Site Internet.\\
        Ce pôle est notre outil de communication. Il est géré par Anthony et comprend la création
        et la gestion du site Internet, le compte Instagram de notre projet \cite{insta} et la
        gestion du programme de \textit{bêta-testing} (\(\beta\)-testeurs).
      \item Sound design et interface graphique.\\
        Ce pôle est la partie créative de notre projet. Ce pôle est sous la responsabilité de Melvyn.
        Nous avons comme objectif de créer nos propres musiques et de faire un très bon \textit{sound design}.
        Nous avons aussi une charte graphique propre, etc...
      \item Moteur de jeu.\\
        Ce pôle pourrait être assimilé au \textit{gamplay} de notre projet. Il est dirigé par Victorien.
        Le but de ce pôle est de gérer tout ce que va faire / vivre le joueur. 
      \item Intelligence Artificielle et Multijoueur.\\
        Ce pôle est tout le côté technique du jeu. Il est managé par Xavier. Le but est de faire du notre projet
        un jeu complet, jouable en solo contre des adversaires imaginaires ou à plusieurs, chacun pour soi.
    \end{enumerate}
  \subsection{Découpage temporel}
    \begin{itemize}
      \item Période 1: Remise du cahier des charges \(\implies\) Première soutenance
      \item Période 2: Première soutenance \(\implies\) Seconde soutenance
      \item Période 3: Seconde soutenance \(\implies\) Soutenance finale
    \end{itemize}
\clearpage

\section{Conclusion}
\clearpage

\begin{thebibliography}{9}
  \bibitem{history}
  Game4free (2021), \emph{L'histoire des jeux de voitures et automobiles} \\
  \url{https://game-4-free.fr/lhistoire-des-jeux-de-voitures-et-automobiles/}

  \bibitem{insta}
  \url{https://www.instagram.com/bitumelegends/} (Abonnez vous !)

  \bibitem{hi2}
  Wikipédia (2021), \emph{Jeu Vidéo de Course} \\
  \url{https://fr.wikipedia.org/wiki/Jeu_video_de_course}

  \bibitem{h3}
  Blaise Huchet (2020), \emph{20 jeux de voitures qui ont marqué l'histoire}\\
  \url{https://fr.motor1.com/features/408752/top-20-jeux-video-voitures/}

  
\end{thebibliography}

\end{document}
