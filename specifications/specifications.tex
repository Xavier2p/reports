%%%%%%%%%%%%%%%%%%%%%%%%%%%%%%%%%%%%%%%%%%%%%%%%%%%%%%%%%%%%%%%%%%%%%%%%%%%%%%%%%%%%%%%%%%%%%%%%%%%%%%%%%%%%%%%%%%%
% Cahier des Charges 
% Bitume Legends Project
% CarrEniX
% Janvier 2022
%%%%%%%%%%%%%%%%%%%%%%%%%%%%%%%%%%%%%%%%%%%%%%%%%%%%%%%%%%%%%%%%%%%%%%%%%%%%%%%%%%%%%%%%%%%%%%%%%%%%%%%%%%%%%%%%%%%
% Il y a des commentaires dans le document qui vous montre là où il faut que vous notiez quelque chose.

% Packages
\documentclass[11pt,a4paper]{article}
\usepackage{mathtools}
\usepackage[utf8]{inputenc}
\usepackage{graphicx}
\usepackage[french]{babel}
\usepackage[T1]{fontenc}
\usepackage{url}
% \usepackage{fancyhdr}
% \usepackage{blindtext}

% Global settings of document
\setlength\parindent{15pt}
\newcommand{\btmlgs}{\textit{Bitume Legends}}
\newcommand{\AI}{Intelligence Artificielle}
%\pagestyle{fancy}
%\fancyhf{}
%\rhead{Bitume Legends}
%\lhead{Cahier des Charge}

% Begin of the document
\begin{document}

\title{Cahier des Charges \\ {\large Projet Bitume Legends}}
\author{CarrEniX}
\date{Janvier 2022}
\maketitle

\begin{center}
  \includegraphics[width=0.5\textwidth]{Medias/logo192.png}
\end{center}

\tableofcontents
\clearpage

\section{Introduction}
  \begin{enumerate}
    \item donner une vue d'ensemble au Jury
    \item faire ressortir l'intérêt principal du projet
    \item mettre en valeur le but final
  \end{enumerate}
  Structure à respecter pour tout le document:
  \begin{enumerate}
    \item QUOI: ce qui est fait ou qui doit l'être
    \item COMMENT: quels seront les moyens matériels et intellectuels utilisés?
    \item COMBIEN: quels seront les aspects économiques qui devront être pris en contre?
  \end{enumerate}
  % Présentation du groupe
  Nous sommes CarrEniX, le studio de développement de \btmlgs. Notre groupe est composé de 
  Anthony CARON, Victorien CAMBOURIAN, Melvyn DELAROQUE et Xavier de PLACE. Nous sommes en InfoSUP à EPITA
  Rennes. Voici le cahier des charges de notre projet de S2, nommé \btmlgs. Nous sommes en R2 et R1.
  
\clearpage

\section{Origine et Nature du projet}
  Dans notre groupe, nous aimons beaucoup les voitures. Nous regardons tous des émissions comme 
  \textit{Top Gear}, \textit{The Gran Tour} ou encore \textit{Vilebrequin}. Nous avons donc décider de faire
  un projet qui se rapporte à cet univers. Nous avons réfléchi à plusieurs possibilités de jeux, mais nous
  nous sommes mis d'accord sur un jeu de courses plutôt qu'un RPG où les joueurs incarneraient des voitures.
  De plus, un jeu de courses est facilement imaginable avec un mode multijoueur et un mode Intelligence 
  Artificielle. C'était donc un choix qui s'imposait pour nous.\\
  \indent Le nom \textit{Bitume Legends} est un jeu de mots sur le jeu de Gameloft, \textit{Asphalt 9 Legends}.
  Nous connaissons tous ce jeu et nous avons décidé de nous en inspirer grandement et de faire notre 
  spécialité : des jeux de mots pourris. Le nom de notre team CarrEniX est du même type. Nous avons décidé
  de franciser le nom du studio de jeux vidéo Square Enix. En français, cela donne Carré Enix, réduit en
  CarrEnix.
\clearpage

\section{Objet de l'étude}
  Quels sont les buts et intérêts de ce projet ?\\

  Qu'est ce qu'il peut nous apporter en groupe ou individuellement ?\\
  % Chacun met un truc que lui apporte le projet, ou qu'il attend avec ce projet
  % Remplissez le champ {[votre prénom] Lorem Ipsum....} pour conserver la mise en page
  % Hésitez pas si vous avez des trucs à dire, vous pouvez y aller.

  \textit{[Anthony] Ce projet sera mon tout premier, réalisé en groupe sur une aussi grande période de temps. J'aurai l'occasion d'apprendre les côtés positifs et négatifs du travail de groupe afin de mener ce projet a bien. De plus, grand fan de jeux vidéos, j'ai toujours voulu faire mon propre jeux vidéo et ce projet, sera l'occasion pour moi de voir ce qui se cache derrière.}\\
  \indent Anthony\\

  \textit{[Melvyn] Lorem ipsum dolor sit amet, consectetur adipiscing elit. Duis ante augue, dictum id hendrerit sed, faucibus ac odio. In pulvinar, lectus sit amet eleifend feugiat, metus lectus scelerisque felis, vel luctus nulla nibh consectetur arcu. Suspendisse potenti. Ut ut efficitur purus. Phasellus in ligula magna. Ut vestibulum lorem mi, a.}\\
  \indent Melvyn\\

  \textit{[Victorien] Lorem ipsum dolor sit amet, consectetur adipiscing elit. Maecenas vitae aliquet neque. In hac habitasse platea dictumst. Vestibulum ante ipsum primis in faucibus orci luctus et ultrices posuere cubilia curae; Cras nec nisi fringilla, dignissim metus id, facilisis velit. Fusce tempor sapien sapien, id vehicula ligula porta at. Duis dignissim.}\\
  \indent Victorien\\

  \textit{C'est un projet qui me tient à coeur personnellement. Il va m'apporter de la discipline pour le travail
    en équipe. Il va aussi me permettre de découvrir un monde que je ne connais pas très bien, celui du jeu vidéo.
    Enfin, Bitume Legends est une occasion unique pour faire un projet presque entièrement libre, donc quelque
    chose dont on peut vraiment être fier.}\\
  \indent Xavier
\clearpage

\section{État de l'art}
  Le premier jeu de voiture à proprement parler est \textit{Gran Track 10}, un jeu d'arcade signé Atari 
  sorti en 1974. Le jeu était plutôt bien pensé pour l'époque. Nous étions face à un 
  circuit vu de dessus en nous pouvions contrôler notre voiture avec un volant, un levier de vitesses
  et un pédalier. Il sera mis à jour par \textit{Gran Track 20}, une version deux joueurs.
  En 1986, c'est Sega qui sort sa version du jeu de voiture avec \textit{OutRun}. Le premier 
  jeu de en 3D était \textit{Hard Drivin}, produit par Atari. Le très connu \textit{Mario Kart},
  développé par Nitendo, est sorti en 1992 sur Super NES.\\
  \indent Plus récemment, nous pouvons remarquer plusieurs types de jeux de voitures. Il y a en premier lieu 
  les jeux de simulation comme \textit{Formula One} (un jeu de simulation de Fromule 1) ou encore 
  \textit{Gran Turismo}. Un autre style de jeu un peu moins conventionnel est le foot avec des voitures. 
  \textit{Rocket League} est un jeu où les joueurs sont en équipe  et doivent gagner des
  points en marquant des buts avec une grosse balle. Nous avons aussi des jeux qui sont moins réalistes
  que les jeux de simulations. Parmi ceux-là, nous avons \textit{WRC} (pour \textit{World Rallye Cup})
  qui, comme son nom l'indique, est un jeu où le joueur incarne un pilote de rallye. Nous avons aussi 
  \textit{Need for Speed} et \textit{Forza Horizon 5}, qui sont des jeux de courses de voitures.
  Enfin, un dernier style plutôt connu est \textit{Mario Kart}. C'est un jeu beaucoup moins réaliste
  que les précédents cités mais qui mise tout sur un \textit{gameplay} très simplifié et une cible plus
  jeune.
  \textbf{TO BE CONTINUED...}
\clearpage

\section{Découpage du projet}
  \subsection{Répartition des tâches} 
    Nous avons découpé le projet en 4 grands pôles et nous avons désigné une personne responsable
    pour chaque pôle. Le rôle du responsable n'est pas de faire tout le travail de son pôle mais
    de gérer les \textit{deadlines} de son domaine et de coordonner le travail effectué.
    \begin{enumerate}
      \item Communication et Site Internet.\\
        Ce pôle est notre outil de communication et il est géré par Anthony. Il comprend la création
        et la gestion du site Internet, la gestion du compte Instagram de notre projet \cite{insta} et la
        gestion du programme de \(\beta\)\textit{-testing} (bêta-testing). Le pôle comprend aussi le suivi
        des rapports et autres rendus à effectuer. Nous avons décidé de 
        créer un programme de \(\beta\)\textit{-testing} parce que nous savons que nous ne ferons
        jamais un jeu sans bugs et que nous n'aurons pas le temps de tous les trouver. Nous mettrons
        donc en place un système de remonté de bugs directement dans le menu des versions de développement.
      \item Sound Design et Interface Graphique.\\
        Ce pôle est la partie créative de notre projet. Il est sous la responsabilité de Melvyn.
        Nous avons comme objectif de créer nos propres musiques et de faire un très bon \textit{sound design}.
        Nous avons aussi une unité graphique propre, avec une seule police, une palette de couleurs, et notre
        design de textes.
      \item Moteur de Jeu.\\
        Ce pôle pourrait être assimilé au \textit{gamplay} de notre projet. Il est dirigé par Victorien.
        Le but de ce pôle est de gérer tout ce que va faire / vivre le joueur. Cela comprend la gestion des niveaux,
        le système de récolte d'expérience, le système de customisation des voitures, le système de gains etc. C'est le coeur de ce que nous voulons faire ressentir aux joueurs.
      \item \AI\, et Multijoueur.\\
        Ce pôle est tout le côté technique du jeu. Il est managé par Xavier. Le but est de faire de notre projet
        un jeu complet, jouable en solo contre des adversaires imaginaires ou à plusieurs, chacun pour soi. Il comprend l'implémentation et le débugage du multijoueur et de l'\AI.
    \end{enumerate}
  \clearpage

  \subsection{Découpage temporel}
    \begin{itemize}
      \item Période 1: Remise du cahier des charges \(\implies\) Première soutenance
        \begin{enumerate}
          \item Implémentation du mode Multijoueur
          \item Création du programme de \(\beta\)\textit{-testing}
          \item Création du menu principal du jeu
          \item Implémentation de la logique basique du jeu (mouvements des voitures, gestion des crash, etc.)
        \end{enumerate}
      \item Période 2: Première soutenance \(\implies\) Seconde soutenance
        \begin{enumerate}
          \item Création du site Internet
          \item Création des \textit{maps} de jeu en fonction des niveaux
          \item Implémentation de l'\AI
          \item Implémentation des différents modes de jeu (contre l'\AI, contre la montre, etc.)
        \end{enumerate}
      \item Période 3: Seconde soutenance \(\implies\) Soutenance finale
        \begin{enumerate}
          \item Finalisation des niveaux et des systèmes de gains
          \item Finalisation du site Internet
          \item \textbf{TO BE CONTINUED...}
        \end{enumerate}
    \end{itemize}
  \subsection{\AI}
    Notre vision de l'\AI\, est d'implémenter des voitures qui se conduisent toutes seules.
    Cela servira à faire des courses tout seul mais contre la machine, ou de faire des courses en plus grand
    nombre que celui des joueurs connectés. Nous allons utiliser le mode AI de Unity qui est directement intégré
    dans notre \textit{framework} de travail.
  \subsection{Multijoueur}
    % Il faut vérifier mes explications sur le type de multijoueur, je ne connais pas assez bien pour être sur de
    % ce que je dis. N'hésitez pas à me corriger !
    Notre multijoueur est un PvP (\textit{Player versus Player}), c'est à dire qu'il faut disputer la course 
    contre un autre joueur qui doit posséder le jeu sur sa machine. Le multijoueur est un mode du jeu, il peut 
    se jouer sans et utiliser par exemple le mode contre la montre ou celui contre l'Intelligence Artificielle.
    Pour implémenter notre mode multijoueur, nous avons décider d'utiliser la solution de \emph{Photon Engine}.
    C'est la solution qui nous a paru la plus complète et relativement facile à implémenter.
\clearpage

\section{Conclusion}
  Lorem ipsum dolor sit amet, consectetur adipiscing elit. Pellentesque dui ligula, gravida a vehicula et, semper eu erat. Aenean sodales a ligula vel sollicitudin. Praesent mollis sapien nec arcu molestie, ut cursus erat elementum. Donec massa metus, placerat nec elementum quis, posuere eu massa. Nullam tempus facilisis maximus. Aenean eget dictum nisi, et faucibus nibh. Nullam est urna, congue vitae risus at, imperdiet malesuada elit. Suspendisse id tortor auctor elit accumsan aliquet vel dictum elit. Suspendisse potenti. Nunc semper ligula nulla, at imperdiet nibh commodo ac. Mauris cursus finibus ligula, et posuere eros.
  Aliquam erat volutpat. Donec ac turpis congue, porta sapien accumsan, auctor nulla. Vivamus sed facilisis diam. Integer dapibus ac dui porttitor scelerisque. Duis ac orci et lectus mattis euismod in sit amet sapien. Integer vel semper urna. Curabitur ultrices urna sapien, sed mollis nunc lobortis nec. Nullam elementum eros elit, vel dignissim turpis hendrerit vel. Mauris eget.
\clearpage

\begin{thebibliography}{9}
  \bibitem{insta}
    \url{https://www.instagram.com/bitumelegends/} (Abonnez vous !)
  
  \bibitem{history}
    Game4free (2021), \emph{L'histoire des jeux de voitures et automobiles}\\
    \url{https://game-4-free.fr/lhistoire-des-jeux-de-voitures-et-automobiles/}

  \bibitem{hi2}
    Wikipédia (2021), \emph{Jeu Vidéo de Course} \\
    \url{https://fr.wikipedia.org/wiki/Jeu_video_de_course}

  \bibitem{h3}
    Blaise Huchet (2020), \emph{20 jeux de voitures qui ont marqué l'histoire}\\
    \url{https://fr.motor1.com/features/408752/top-20-jeux-video-voitures/}

\end{thebibliography}

Made with $\heartsuit$ by CarrEniX on \LaTeX.\\
\textcopyright\, 2021-2022 Bitume Legends
\end{document}