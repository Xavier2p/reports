%%%%%%%%%%%%%%%%%%%%%%%%%%%%%%%%%%%%%%%%%%%%%%%%%%%%%%%%%%%%%%%%%%%%%%%%%%%%%%%%%%%%%%%%%%%%%%%%%%%%%%%%%%%%%%%%%%%
% Cahier des Charges 
% Bitume Legends Project
% CarrEniX
% Janvier 2022
%%%%%%%%%%%%%%%%%%%%%%%%%%%%%%%%%%%%%%%%%%%%%%%%%%%%%%%%%%%%%%%%%%%%%%%%%%%%%%%%%%%%%%%%%%%%%%%%%%%%%%%%%%%%%%%%%%%

% Packages
\documentclass[12pt,a4paper]{article}
\usepackage{mathtools}
\usepackage[utf8]{inputenc}
\usepackage{graphicx}
\usepackage[french]{babel}
\usepackage[T1]{fontenc}
\usepackage{url}
\usepackage{fancyhdr}
\usepackage{longtable}
\usepackage[aboveskip=.5cm]{caption}
\usepackage{pdflscape}
\usepackage{geometry}

% Global settings of document
\setlength\parindent{20pt}
\newcommand{\btmlgs}{\textit{Bitume Legends}}
\newcommand{\AI}{Intelligence Artificielle}
\newcommand{\CEX}{CarrEniX}
\pagestyle{fancy}
\rhead{\btmlgs}
\lhead{Cahier des Charges}
\rfoot{Page \thepage}
\fancyfoot[C]{}
\setlength{\headheight}{15pt}

% Begin of the document
\begin{document}

\begin{titlepage}
  \newcommand{\HRule}{\rule{\linewidth}{0.5mm}}
  \center
  \text{\LARGE Projet \btmlgs}\\[1cm]
  \includegraphics[scale=0.7]{Medias/logo192.png} \\[1cm]
  \HRule \\[0.4cm]
  { \huge \bfseries Cahier des Charges \\[0.15cm] }
  \HRule \\[1.5cm]
  \CEX \\[0.3cm]
  Anthony CARON\;--\;Melvyn DELAROQUE\\ Victorien CAMBOURIAN\;--\;Xavier de PLACE
   \\ [6cm]
  EPITA INFOSUP 2026\\Année 2021 - 2022
  \end{titlepage}

% Former title page
% \title{Cahier des Charges \\ {\large Projet Bitume Legends}}
% \author{CarrEniX\\ Anthony CARON\;\; Melvyn DELAROQUE\\ Victorien CAMBOURIAN\;\; Xavier de PLACE}
% \date{Janvier 2022}
% \maketitle
% \begin{center}
%   \includegraphics[width=0.7\textwidth]{Medias/logo192.png}
% \end{center}
% \begin{center}
%   EPITA INFOSUP 2026\\2021 - 2022
% \end{center}
% \clearpage
\tableofcontents
\clearpage

\section{Introduction}
  % Présentation du groupe
  Nous sommes \CEX, le studio de développement de \btmlgs. Notre groupe est composé d'Anthony CARON, 
  Victorien CAMBOURIAN, Melvyn DELAROQUE et Xavier de PLACE. Voici le cahier des charges de notre projet
  de S2, nommé \btmlgs.\\
  
  \btmlgs\, est un jeu vidéo automobile, plus précisément un jeu de courses de voitures. Le but est 
  de faire un maximum de courses pour gagner de l'expérience ainsi que de l'argent afin de
  modifier et améliorer sa ou ses voitures.
  Nous aurons plusieurs modèles de voitures inspirées du monde réel adaptées en style \textit{Low Poly},
  comme dans un cartoon. De plus, trois modes de jeu seront disponibles : contre la montre, contre une \AI\,
  ou alors contre un joueur à distance en réseau. Ces courses se dérouleront sur différents circuits créés 
  par nos soins en respectant une physique réaliste.
  Les musiques seront également signées \CEX, ainsi que le \textit{sound design} des voitures.
  Notre but est de faire un jeux vidéo amusant et ludique, tout en ayant un minimum de réalisme.\\
  
  Ce projet nous permettra de faire un travail en équipe pendant une longue période tout en respectant les dates limites.
  Nous nous servirons du moteur de jeu Unity3D et le C\# en temps que langage de programmation. Nous intègrerons 
  également des assets de Unity pour gagner du temps, pour pouvoir implémenter des fonctionnalités plus poussées dans notre projet.
  Nous utiliserons GitHub pour collaborer et nous échanger facilement nos avancées techniques.
  Notre jeu possèdera une version Windows pour PC mais aussi une version Mac.
  Nous prévoyons de faire des objets dérivés de notre jeu vidéo (T-Shirts, stickers) pour avoir une unité d'équipe 
  pendant les soutenances.\\
  
  % Vous trouverez ci-dessous un approfondissement de ce que nous avons dit précédemment, détaillant l'histoire de notre jeu 
  % et celle des jeux de voitures
  % en général. Puis, nous vous présenterons la vision de chaque membre du groupe sur ce projet. Enfin, un calendrier 
  % détaillé de l'avancement de notre projet, avec les rôles de chacun vous sera proposé.
  
\clearpage

\section{Origine du projet}
  Dans notre groupe, nous aimons beaucoup les voitures. Nous regardons tous des émissions comme 
  \textit{Top Gear}, \textit{The Gran Tour} ou encore \textit{Vilebrequin}. Nous avons donc décider de faire
  un projet qui se rapporte à cet univers. Nous avons réfléchi à plusieurs possibilités de jeux, mais nous
  nous sommes mis d'accord sur un jeu de courses plutôt qu'un RPG où les joueurs incarneraient des voitures.
  De plus, un jeu de courses est facilement imaginable avec un mode multijoueur et un mode Intelligence 
  Artificielle. C'était donc un choix qui s'imposait pour nous.\\
  
  \indent Le nom \textit{Bitume Legends} est un jeu de mots sur le jeu de Gameloft, \textit{Asphalt 9 Legends}.
  Nous connaissons tous ce jeu et nous avons décidé de nous inspirer grandement de son nom tout en y intégrant
  une de nos spécialités : les jeux de mots. Le nom de notre team \CEX\, est du même type. Nous avons décidé
  de franciser le nom du studio de jeux vidéo Square Enix. En français, cela donne Carré Enix, réduit en
  \CEX.\\

\section{État de l'art}
  Le premier jeu de voiture à proprement parler est \textit{Gran Track 10}, un jeu d'arcade signé Atari 
  sorti en 1974. Le jeu était plutôt bien pensé pour l'époque. Nous étions face à un 
  circuit vu de dessus en nous pouvions contrôler notre voiture avec un volant, un levier de vitesses
  et un pédalier. Il sera mis à jour par \textit{Gran Track 20}, une version pour deux joueurs.
  En 1986, c'est Sega qui sort sa version du jeu de voiture avec \textit{OutRun}. Le premier 
  jeu de voitures en 3D était \textit{Hard Drivin}, produit par Atari. Le très connu \textit{Mario Kart},
  développé par Nintendo, est sorti en 1992 sur Super NES.\\

  \indent Plus récemment, nous pouvons remarquer plusieurs types de jeux de voitures. Il y a en premier lieu 
  les jeux de simulation comme \textit{Formula One} (un jeu de simulation de Formule 1) ou encore 
  \textit{Gran Turismo}. Il y a également des jeux moins réalistes
  que les jeux de simulations. Parmi ceux-là, nous avons \textit{WRC} (pour \textit{World Rallye Cup})
  qui, comme son nom l'indique, est un jeu où le joueur incarne un pilote de rallye. Dans la 
  catégorie des jeux de courses de voitures, nous pouvons retrouver 
  \textit{Need for Speed} et \textit{Forza Horizon 5}.
  En opposition, \textit{Rocket League} est un jeu où les joueurs sont en équipe  et doivent gagner des
  points en marquant des buts avec une grosse balle. Cela s'apparente à du \textit{car soccer}, ou "foot
  de voitures".
  Enfin, \textit{Mario Kart} est un jeu beaucoup moins réaliste
  que ceux précédemment cités mais qui mise sur un \textit{gameplay} plus simplifié et une cible plus jeune. 
 \clearpage

\section{Objet de l'étude}
  Pour arriver à finir notre projet, nous devons nous organiser.
  Ce projet va donc nous apprendre le travail de groupe, là où nous étions habitués à travailler seuls. 
  Nous allons également apprendre les bases du déroulement d'un projet dans une entreprise, 
  et pour les années futures à EPITA. Nous avons décider de donner chacun notre avis sur ce projet et sur ce 
  qu'il pourrait nous apporter individuellement.\\

  \indent\textit{Ce projet sera mon tout premier réalisé en groupe sur une aussi grande période de temps. 
    J'aurai l'occasion d'apprendre les côtés positifs et négatifs du travail de groupe afin de mener ce 
    projet à bien. De plus, grand fan de jeux vidéos, j'ai toujours voulu faire mon propre jeux vidéo et ce 
    projet sera l'occasion pour moi de voir ce qui se cache derrière.} \\
  \indent Anthony\\[0.3cm]
  \indent\textit{L'intérêt personnel que j'ai envers le développement d'un jeu de course automobile vient de ma 
    passion depuis l'enfance pour ce genre de jeux. En effet, plus jeune et encore aujourd'hui, je joue
    à des jeux de voitures (Need For Speed, Mario Kart ou Asphalt) et je regarde beaucoup de contenu automobile
    à la télévision (Top Gear UK) et sur Youtube (Vilebrequin). C'est un milieu qui mêle à la fois passion et 
    mentalité d'ingénieur, vu qu'elle permet de répondre à la question : comment ces jeux que j'adore ont été 
    fait ? L'expérience acquise en travaillant directement sur un projet de développement de jeux vidéoludique 
    permettrait de développer mes compétences à la fois techniques et artistiques et d'apprendre concrètement 
    ce qu'est le travail de groupe en milieu professionnel} \\
  \indent Melvyn\\[0.3cm]
  \indent\textit{Bien que j'aime bien être chef de projet, je trouve qu'il était intéressant 
    pour moi de me positionner différement, pour pouvoir également voir d'autres façons 
    de cadrer et diriger une équipe. De plus, ce projet va nous permettre de voir les différentes
    méthodes d'approche de résolution des problèmes auxquels nous serons confronté, tout en s'y
    adaptant (lors de la correction des bugs par exemple). Aussi, nous allons devoir être organisés
    et efficaces étant donné que le développement du jeu se fait sur le temps libre en plus de l'école.} \\
  \indent Victorien\\[0.3cm]
  \indent\textit{C'est un projet qui me tient à coeur personnellement. Il va m'apporter de la discipline pour le travail
    en équipe. Il va aussi me permettre de découvrir un monde que je ne connais pas très bien, celui du jeu vidéo.
    Enfin, Bitume Legends est une occasion unique pour faire un projet presque entièrement libre, donc quelque
    chose dont on peut vraiment être fier.} \\
  \indent Xavier
\clearpage

\section{Comment allons-nous réussir ce projet ?}
  \subsection{Répartition des tâches} 
    Nous avons découpé le projet en 4 grands pôles et nous avons désigné une personne responsable
    pour chaque pôle. Le rôle du responsable n'est pas de faire tout le travail de son pôle seul mais
    de gérer les \textit{deadlines} de son domaine et de coordonner le travail effectué.\\
    \begin{enumerate}
      \item Communication et Site Internet.\\
        Ce pôle est notre outil de communication et il est géré par Anthony. Il comprend la création
        et la gestion du site Internet, la gestion du compte Instagram de notre projet \footnote{\url{https://www.instagram.com/bitumelegends} (Abonnez vous !)}et la
        gestion du programme de \(\beta\)\textit{-testing} (bêta-testing). Le pôle comprend aussi le suivi
        des rapports et autres rendus à effectuer.
        \\
      \item \textit{Sound Design} et Interface Graphique.\\
        Ce pôle est la partie créative de notre projet. Il est sous la responsabilité de Melvyn.
        Nous avons comme objectif de créer nos propres musiques et de faire un bon \textit{sound design}.
        Nous avons aussi une unité graphique propre, avec une seule police, une palette de couleurs, et notre
        design de textes.
\\
      \item \textit{Gameplay}.\\
        Ce pôle est dirigé par Victorien.
        Son but est de gérer tout ce que va faire et vivre le joueur. Cela comprend la gestion des niveaux,
        le système de récolte d'expérience, de customisation des voitures, de gains etc. C'est le coeur 
        de ce que nous voulons faire ressentir aux joueurs.
        \\
      \item \AI\, et Multijoueur.\\
        Ce pôle est tout le côté technique du jeu. Il est managé par Xavier. Le but est de faire de notre projet
        un jeu complet, jouable en solo contre des adversaires imaginaires, ou à plusieurs, chacun pour soi. Il 
        comprend l'implémentation et le débugage du multijoueur et de l'\AI.
    \end{enumerate}
  \clearpage

  \subsection{Découpage temporel}
    \begin{description}
      \item Période 1: Remise du cahier des charges \(\implies\) Première soutenance
        \begin{enumerate}
          \item Implémentation du mode Multijoueur
          \item Création du menu principal du jeu
          \item Implémentation de la logique basique du jeu (mouvements des voitures, gestion des crash, etc.)
          \item Création du programme de \(\beta\)\textit{-testing}
          \\
        \end{enumerate}
      \item Période 2: Première soutenance \(\implies\) Seconde soutenance
        \begin{enumerate}
          \item Création et mise en ligne du site Internet
          \item Création des \textit{maps} de jeu en fonction des niveaux
          \item Implémentation de l'\AI
          \item Implémentation des différents modes de jeu (contre l'\AI, contre la montre, etc.)
          \item Correction des bugs
          \\
        \end{enumerate}
      \item Période 3: Seconde soutenance \(\implies\) Soutenance finale
        \begin{enumerate}
          \item Finalisation des niveaux et des systèmes de gains
          \item Mise en place des musiques et du \textit{sound design} (menu, en course, multijoueur)
          \item Finalisation du site Internet
          \item Correction des bugs
          \\
          \\
          \\
        \end{enumerate}
    \end{description}

  \subsection{Moyens techniques}
    \subsubsection{Moteur de jeu}
      Nous avons décidé d'utiliser le moteur de jeu Unity 3D, qui nous convient très bien. Il possède plusieurs 
      \textit{features} très bien pensées, et il est relativement simple à prendre en main. Notre langage est 
      donc le C\#, et nous développerons sur l'environnement Rider de JetBrains. Pour les visuels, nous
      utiliserons Blender pour la partie 3D et GIMP pour le 2D.
    \subsubsection{\AI}
      Notre vision de l'\AI\, est d'implémenter des voitures qui se conduisent toutes seules.
      Cela servira à faire des courses seul contre la machine ou ajouter des \textit{bots} dans le mode 
      multijoueur. Nous nous baserons sur le mode AI de Unity qui est directement intégré
      dans notre \textit{framework} de travail. Nous rajouternons nos propres fonctions pour obtenir
      le résultat que nous voulons.
    \subsubsection{Communication}
      Nous utilisons \LaTeX\, pour effectuer nos rapports et notre cahier des charges. Pour notre site web, 
      nous utiliserons Bootstrap Studio. Nous avons un compte Instagram (\texttt{@bitumelegends}) 
      ainsi qu'un serveur Discord pour communiquer entre nous et pouvoir proposer le jeu en avance.
      Nous avons décidé de créer un programme de \(\beta\)\textit{-testing} parce que nous savons que nous 
      ne ferons jamais un jeu sans bugs et que nous n'aurons pas le temps de tous les trouver. Nous mettrons
      donc en place un système de remontée de bugs directement dans le menu des versions de développement, 
      dans le site web et dans le serveur Discord.
    \subsubsection{Multijoueur}
      Notre multijoueur est un PvP (\textit{Player versus Player}), c'est à dire qu'il faut disputer la course 
      contre un autre joueur qui doit posséder le jeu sur sa machine. Le multijoueur est un mode de notre jeu. 
      Cependant Bitume Legends peut être joué en solo en utilisant par exemple le mode contre la montre ou 
      celui contre l'Intelligence Artificielle. Pour implémenter notre mode multijoueur, nous avons décider 
      d'utiliser la solution de \emph{Photon Engine}. C'est la solution qui nous a paru la plus complète et 
      elle est relativement simple à implémenter.
    \subsubsection{Musique et \textit{Sound Design}}
      Nous utiliserons FL Studio 20 pour composer l'OST (\textit{Original Sound Track}) dans un style Phonk.
      Nous prévoyons une dizaines de musiques originales au total, avec différentes ambiances pour
      différents moments du jeu (menu, courses, victoires, etc.). Pour le sound design, un bruit de 
      moteur adaptatif en fonction de la vitesse sera créé à partir de bruits de moteurs déjà 
      existants et retravaillés pour le jeu.
\clearpage

\section{Conclusion}
    Nous vous remercions de l'attention que vous portez à \btmlgs.
    Ce projet nous tient vraiment à coeur et nous espérons vous avoir transmis notre enthousiasme.
    Le studio \CEX\, a hâte de vous présenter de vive voix ce projet lors de sa première soutenance le 7 mars 2022.


\section{Annexe : Tableau récapitulatif du projet}
\renewcommand{\arraystretch}{1.2}
\setlength{\LTleft}{-1cm plus 1 fill}
\setlength{\LTright}{-1cm plus 1 fill}
\begin{longtable}{| p{4.5cm} || c | c | c | c | c |}
  \hline
  Tâches & Période 1 & Période 2 & Période 3 & Responsable & Suppléant\\\hline\hline
  Multijoueur & 100\% & 100\% & 100\% & Xavier & Victorien \\\hline
  Menu principal & 85\% & 95\% & 100\% & Melvyn & Xavier \\\hline
  Logique du jeu & 100\% & 100\%  & 100\% & Victorien & Anthony\\\hline
  Programme \(\beta\) & 50\% & 75\% & 100\% & Anthony & Melvyn \\\hline
  Site Internet & 50\% & 85\%  & 100\% & Anthony & Victorien\\\hline
  Maps et niveaux & 0\% & 95\%  & 100\% & Victorien & Melvyn\\\hline
  \AI & 0\% & 100\%  & 100\% & Xavier & Anthony\\\hline
  Implémentations modes de jeu & 0\% & 100\%  & 100\% & Victorien & Xavier\\\hline
  Systèmes de gains & 0\% & 10\%  & 100\% & Anthony & Xavier\\\hline
  \textit{Sound Design} & 10\% & 45\% & 100\% & Melvyn & Victorien\\\hline
  Musiques & 10\% & 45\%  & 100\% & Melvyn & Anthony \\\hline
  \caption{Tableau récapitulatif de notre projet}
\end{longtable}

\begin{thebibliography}{9}
  Articles utilisés pour l'histoire du jeu vidéo de voitures :\\

  Game4free (2021), \emph{L'histoire des jeux de voitures et automobiles}\\
  \url{https://game-4-free.fr/lhistoire-des-jeux-de-voitures-et-automobiles/}\\

  Wikipédia (2021), \emph{Jeu Vidéo de Course} \\
  \url{https://fr.wikipedia.org/wiki/Jeu_video_de_course}\\

\end{thebibliography}

\begin{center}
  Made with $\heartsuit$ by \CEX\, on \LaTeX.\\
\textcopyright\, 2021-2022 \btmlgs
\end{center}

\end{document}